
\documentclass[conference]{IEEEtran}


\usepackage{listings}
\usepackage{epsfig}
\usepackage{url}
\usepackage{cite}
\usepackage{fancybox}
\usepackage{amsthm}
\usepackage{tikz}

\newcommand{\tana}{\textsc{TANA}}

\begin{document}

\title{TANA: Fined-grained Side-channel Inforamtion Leakage Quantification in Binaries}
\author{Anonymous}

\maketitle

\begin{abstract}
Side-channel attacks allow attackers to infer some sensitive information based on 
non-functional characteristics. Existing works on address-base side-channel detection 
can provide a list of potential side-channel leakage sites. We observe 
 the following limitations in the previous work: 1) 
Some vulnerabilities could be more severe than others, but the existing work 
cannot tell precisely the difference between those leakages. 2)  An attacker usually exploits multiple 
leakages at one time. However, no existing tool can precisely report
the sum of the information leakage from multiple sites given the information dependency among them.

---FIXME: EMPHASIZE NOVELTY AND METHOD---
To overcome the above limitations, we propose a novel method
to more precisely quantify the information leakage. Previous work only considers
``average'' information leakage given that the event, e.g., a branch condition, has not happened yet and
it could either happen or not happen in the future. For example, given a crypto key, we can assume the key is fixed
in the attack scenario because the attacker can fix the key and do profiling. The key insight is ...
By fixing with more constraints, we are able to obtain more precise information leakage.
Our results are superisingly different compared to previous results and muhc more useful in practice.

---FIXME: NEED A BETTER NAME---
We have developed a tool called TANA, which can not only 
find the side-channel vulnerabilities but can estimate how many bits are actually leaked 
through the leakage. TANA works in three steps. First, the application is executed to record the 
trace. Second, TANA runs the instruction level symbolic execution on the top of the 
execution trace. TANA will find side-channel information leakages and model each leakage 
as one unique math constraint. Finally, TANA will classify those constraints into 
independent multiple groups and run the multiple step Monte Carlo to estimate the 
information leakage. TANA can report a very fined-grained vulnerability result 
compared to the existing tools.
We apply the tool on OpenSSL, MbedTLS and libjpeg and find several serious side channel 
vulnerabilities. We also evaluate the vulnerabilities from previous research. The result 
confirms our intuition; it 
indicates most of the reported vulnerabilities are actually hard to exploit in practice.

\end{abstract}

\IEEEpeerreviewmaketitle
\pagenumbering{arabic}

\section{Introduction}
%% side channels are important
Side channels are inevitable in modern computer systems as the sensitive information 
may be leaked by many kinds of inadvertent behaviors, 
such as power, electromagnetic radiation and even sound~\cite{xxx}. 
Among them, software-based side channels, such as cache attacks, memory page attacks,
and controlled-channel attacks, are especially common 
and have been studied for years~\cite{xxx}. 
These vulnerabilities result from vulnerable software and shared hardware components.
By observing the outputs or hardware behaviors, attackers can
infer the program execution flow that manipulate secrets and 
guess the secrets such as encryption keys~\cite{xxx}.

%% to deal with side channels, we can protect or detect them and detection is better
Various countermeasures have been proposed to defend against 
software-based side-channel attacks. Hardware level solutions, 
including reducing shared resources, adopting oblivious RAM, and using
transnational memory~\cite{182946,203878,217537} need new hardware features or changes
to modern complex computer systems, which is impractical and hard to adopt in 
reality. Therefore, a more promising and universal direction is software countermeasures, 
detecting and eliminating side channel vulnerabilities from code.

Regarding the root cause of software-based side channels, 
many of them are caused by the following two specific types: 
data flow from secrets to load addresses and data flow from secrets to branch conditions.
We call them secret-dependent control-flow and memory-access correspondingly.
Therefore, a central problem is identifying those two code patterns automatically.
Recent works~\cite{203878,data,caches} adopt static and dynamic analysis
to detect side-channels.
They can find many potential leak sites in real-world software, 
but fail to report how severe a potential leakage could be. 
Many of the reported vulnerabilities are typically hard to exploit
and leak very little information. For example, DATA~\cite{xxx} reports
2,246 potential leakage site for the RSA implementation in OpenSSL\@.
After some inspectations, 1,510 are dismissed, but it still
leaves 278 control-flow and 460 data-access patterns. For software
developers, it is hard for them to fix all those vulnerabilities,
let alone the majority of them are negligible.
While some vulnerabilities can be used to recover the full secret
keys~\cite{xxx}, many other vulnerabilities prove to be less serious in reality.

To assess the sensitive level of side-channel vulnerabilities, we need a proper 
quantification metric.
Static methods, usually with abstract interpretation, can give a leakage upper bound, 
which is useful to justify the implementation is secure when they report zero or little leakage. 
However, they cannot indicate how serious the leakage is because of over-approximation~\cite{xxx}. 
For example, CacheAudit~\cite{xxx} reports that the upper bound leakage of AES-128 exceeds 
the original key size! The dynamic methods take another approach with a concrete input and 
run the program in real environment. Although they are very precise in term of true leakages, 
no existing tool can precisely assess the severity of the vulnerabilities they discover.

To overcome these limitations, we propose a novel method
to quantify information leakage more precisely. 
Different from previous works, which only consider the
``average'' information leakage, we study the problem based on real attack scenarios.
The average information assumes that the target program will have \emph{variable} sensitive 
information when an attack is launched.
However, for real-world attacks, an adversary may run the target problem again and over again 
with \emph{fixed} unknown sensitive information such as the key. 
Therefore, the previous threat model cannot catch real attack scenarios.
In contrast, our method is more precise and fine-grained. 
We quantify the amount of leaked information as the cardinality of the set of 
possible inputs based on attackers' observations. 

Before an attack, an adversary has a big but finite input space.
Every time when the adversary observes a leakage site, he can eliminate some 
potential inputs and reduce the size of the input space. 
The smaller the input space is, the more information is actually gained. 
In an extreme case, if the size of the input space reduces to one, 
the adversary can determine the input information uniquely, which means all the secret information
(e.g., the whole secret key) is leaked. By counting the number of distinct inputs, 
we can quantify the information leakage more precisely. 

We use constraints to model the relation between the original sensitive input and
each leakage site. We run the instruction level symbolic execution on the whole
execution trace to generate the constraints. Symbolic execution can provide the fine-grained
information but is usually believed to be an expensive operation in terms of performance. 
Therefore, existing dynamic symbolic execution based works~\cite{xxx} either only analyze 
small programs or apply some domain knowledges to simplify the execution. We systematically
analyze the bottleneck of the symbolic execution and optimzie it scalable to
real-world cryptosystems. 

We apply the above technique and build a tool called \tool{},\footnote{\tool\ is a horse that can ``count''.
Our tool uses an advanced method to count the number of leaked bits from side channels.}
which could discover potential information leakage sites 
as well as estimating how many bits they can leak for each leakage site. 
We assume that adversaries can exploit secret-dependnet control-flow transfers and 
data-access patterns when the program processes different sensitive data. 
%We refer them as the potential information leakage sites. 
First, we collect the dynamic execution trace for each input of the target libraries 
and then run symbolic execution on the traces. 
In this way, we model each side-channel leakage as a math formula. 
The sensitive input is divided into several independent bytes and each byte is regarded as 
a unique symbol. Those formulas can precisely model side-channel vulnerabilities.
Then we extend the problem to multiple leakages and related leakages
and introduce a monte carlo sampling method to estimate the single and combined information leakage.
In fact, if an application has a different sensitive input but still satisfies the formula, 
the code can still leak the same information. 


%Based on the fixed attack target, we classify the software-based side-channel 
%vulnerabilities into two categories: 1.\textit{secret-dependent control-flow transfers} 
%and 2.\textit{secret-dependent data accesses} and model them with math formulas which
%constrain the value of sensitive information.
%We quantify the amount of leaked information as the number of possible solutions that are
%reduced after applying each constrains.


%Our method can identify and quantify address-based
%sensitive information leakage sites in real-world applications automatically. 
%Adversaries can exploit different control-flow transfers and data-access patterns when 
%the program processes different sensitive data. We refer them as the potential information
%leakage sites. Our tool can discover and estimate those potential information leakage sites 
%as well as how many bits they can leak. We are also able to report precisely how many bits
%can be leaked in total if an attacker observes more than one site.
%We run symbolic execution on execution traces. We model each side-channel leakage as a math formula. 
%The sensitive input is divided into several independent bytes and each byte is regarded as 
%a unique symbol. Those formulas can precisely model every the side-channel vulnerability. 
%In other words, if the application has a different sensitive input but still satisfies the formula, 
%the code can still leak the same information.  
%Those information leakage sites may spread in the whole program 
%and their leakages may not be dependent. Simply adding them up can only get a coarse upper bound 
%estimate. In order to accurately calculate the total information leakage, we must know the 
%dependent relationships among those multiple leakages sites. Therefore, we introduce a 
%monte carlo sampling method to estimate the total information leakage.

We apply \tool{} on both symmetric and asymmetric ciphers from real-world crypto libraries including OpenSSL and
mbedTLS. The experimental result confirms that \tool{} can precisely identify the previous known vulnerabilities,
reporting how much information is leaked and which byte in the original sensitive buffer is leaked. 
Although some of the analyzed crypto libraries have a number of side-channels, they actually
leak very little information. Also, we perform the analysis of widely deployed software countermeasures
against side channels.
Finally, we present new vulnerabilities. With the help of \tool{}, we confirm those
vulnerabilities are easily to be exploited. Our results are superisingly different compared to previous results
and much more useful in practice.

In summary, we make the following contributions:

\begin{itemize}
	\item We propose a novel method that can quantify fine-grained leaked information from side-channel
        vulnerabilities. We model each side-channel vulnerabilities as math formulas and 
        mutiple side-channel vulnerabilities can be seen as the conjunction of those formulas, which
        precisely models the program semantics.
        \item We transfer the information quantification problem into a probabilty distribution problem and 
        use the Monte Carlo sampling method to estimate the information leakage. Some initial results indicate the 
        the sampling method suffers from the curse of dimensionality problem. We therefore design a guided
        sampling method and provide the corresponding error esitimate.
	\item We implement the proposed method into a practical tool and apply it on several real-world software. \tool{} 
        successfully identifies the address-based side-channel vulnerabilities and provides the corresponding
        information leakge. The information leakage result provides the detailed information that help developers to
        fix the reported vulnerabilities.
\end{itemize}

\section{Background}
In this section, we first present a basic introduction about the 
memory-based side-channel attack. Those attacks 
are exactly what we attempt to study in the paper. After that we 
will present existing work on side-channel detection and quantification.
We will also analyze strengths and limitations of those quantification 
methods.

\subsection{Address-based Side-Channels}
Address-based side-channels are information channels that can leak sensitive information unintended
through the different behaviors when the program accesses different memory addrsss. Fundamentally,
those difrences were caused by the memory hierarchy design in modern computer systems. When the 
CPU fetches the data, it will first search the cache, which stores copies of the data from 
the frequently used main memory. If the data doesn't exist in the cache, the CPU will read
the data from the main memory (RAM). Classified by the layer caused the side-channel, we 
introduce two kinds of commom side-channels: cache-based side-channel attacks and memory-based
side-channel attacks.

\subsubsection{Cache-based Attack}
In general, the cached-based side-channel attacks seek information 
rely on the time differences between the cache miss
and cache hit. Here we introduce two types of cache attacks:
PRIME+PROBE, FLUSH+RELOAD.

\textbf{PRIME+PROBE} targets a single cache set. It has two phases. During the
"prime" phase, the attacker fills the cache set will his own data.
In the second "probe" phase, the attacker accesses the cache set
again. If the victim accesses the cache set and evicts part of 
the data, the attacker will experience a slow mesurement. If not, 
the mesurement will be fast.

\textbf{FLUSH+RELOAD} targets a single cache line. 
It requires the attacker and victim share the same memory.
It also have two phases. During the "flush" phase, the attacker 
will flush the "monitered memory" from the cache. Then the attacker
wait for the victim to access the memory. In the third phase, the 
attacker reload the "monitered memory". If the time is short, which
indicates there is a cache hit and the victim reolads the memory before. 
On the other hand, the time will be longer since the CPU need to reolad
the memory into the cache line. 

\subsubsection{Memory-based Attack}
Memory-based side-channel attack\cite{} exploits the different behaviors when the
program accesses different page tables. The controlled-channel attack\cite{7163052},
which works in the kernel space, can infer the sensitive data in the shielding systems by
observing the page fault sequences by restricting some code and
data pages. 

After examing the memory-based side-channels attack. We find the fundamentally
reason of those attacks are due to secret-dependent memory access and control
flow transfers.
\lstinputlisting[language=c, 
                 numbers=right,
                 caption={Sample code shows secret-dependent memory access and 
                          secret-dependent control-flow transfer.},
                 captionpos=b,
                 label={code:background},
                 basicstyle=\fontsize{7}{9}\selectfont\ttfamily]
                 {sample_code/background.c}

For exampls, the above code~\ref{code:background} show an simple encryption function that
has the two kinds of side-channels. At the line 11, dependending on the value of key,
the code will access the different entry in the predefined table /textbf{Table}. At the
line 13, the code will do a series of computation and determine if the code in the if
branch is executed or not. Such vulnerabilities could leak to the memory-based 
side-channles. We identify and quantify the leakage of the two kinds of vulnerabilities 
in the paper.

\subsection{Information Leakage Quantification}
Given an event e which occurs with the probability $P(e)$, if the event e happens, 
then we receive
\begin{equation}
    I = - log(P(e))
\end{equation}
bits of information by knowing the event e.

The above definition is obvious. Suppose a char variable \textit{a} in C program has the size
of one byte (8 bits), so the value in the variable can range from 0 - 255. We assume
the \textit{a} has the uniform distribution. If at one time we observe the \textit{a}
equals to 1, the probability will be 1/256. So the information we get is 
$-log(1/256) = 8 bits$, which is exactly the size of the char variable in C program.

Existing works on information leakage quantification are based on mutual information or 
min-entropy \cite{10.1007/978-3-642-00596-1_21}.
In their frameworks, the input sensitive
information $K$ is viewed as random variables. Let $k_i$ be one of the possible
value of $K$. The Shannon entropy $H(K)$ is defined by
\begin{equation}
    H(K) = - \sum_{k_i {\in} K}P(k_i)log(P(k_i))
\end{equation}

The Shannon entropy can be used to quantify the initial uncertainty about the sensitive
information. Suppose a program (P) with the $K$ as
the sensitive input, an adversary has some observations (O) through the side-channels.
In this work, the observations are referred to the secret-dependent control-flows and
secret-dependent data-accesses patterns. The conditional entropy $H(K|O)$ is
\begin{equation}
    H(K|O) = - \sum_{o_j {\in} O} {P(o_j) \sum_{k_i {\in} K}{P(k_i|o_j)log(P(k_i|o_j))}}
\end{equation}
Intuitively, the conditional information marks the uncertainty about $K$ after the adversary
has gained some observations (O). 

Many previous works use the mutual information $I(K; O)$ to quantify the leakage which is defined 
as follows:
\begin{equation}
    Leakage = I(K;O) = \sum_{k_i {\in} K}{\sum_{o_j {\in} O}{P(k_i, o_j)log(\frac{P(k_i, o_j)}{P(k_i)P(o_j)})}}
\end{equation}
where $P(k_i, o_i)$ is the joint discrete distribution of $K$ and $O$.
Alternatively, the mutual information can also be computed with the following equation:
\begin{equation}
    Leakage = I(K;O) = H(K) - H(K|O) = H(O) - H(O|K)
\end{equation} 
For a deterministic program, once the input $K$ is fixed, the program will have the same
control-flow transfers and data-access patterns. As a result, $P(k_i, o_j)$ will always
equals to 1. So the conditional entropty $H(O|K)$ will equal to zero. Now the leakage defined
by the mutual information can be simplified into:
\begin{equation}
    Leakage = I(K;O) = H(O)
\end{equation}
In other words, once we know the distribution of those memory-access patterns. We can 
calculate how much information is actually leaked.

Another common method is based on maximal leakage ~\cite{10.1007/978-3-642-00596-1_21,10.1007/978-3-642-31424-7_40,182946}.

\begin{equation}
    Leakage = log(C(O))
\end{equation}
$C(O)$ represents the number of different observations that an attacker can have.

Now we provide a concrete example to show how the two types of quantification definition works.
\lstinputlisting[language=c, 
                 numbers=right,
                 caption={A simple program},
                 captionpos=b,
                 label={code::entropy},
                 basicstyle=\fontsize{7}{9}\selectfont\ttfamily]
                 {sample_code/dependent.c}

\textbf{Maximal leakage} 
Dependenting on the value of key, the code can run four different branches which corrosponding to 
four different observations. Therefore, by the maximal leakage definition, the leakage equals to 
$log4 = 2$ bits.

\textbf{Mutual Information} If the key satisfies the uniform distribution, the probability of the code runs each branch
can be computed with the following result: 
\begin{table}[h]
\centering
\begin{tabular}{|c|c|c|c|c|}
\hline
Branch & A     & B      & C      & D       \\ \hline
P      & 1/256 & 64/256 & 64/256 & 127/256 \\ \hline
\end{tabular}
\caption{The distribution of observations}
\end{table}
Therefore, the leakage equals to 
$\frac{1}{256}log\frac{1}{256} + \frac{1}{4}log\frac{1}{4}*2 + \frac{127}{256}log\frac{127}{256} = 1.7$ bits.

\section{Threat Model}
We consider an attacker who shares the same hardware resource with the victim. 
The attacker attempt to retrieve sensitive information via memory-based side-channel attack. 
The attacker has no direct access to the memory or cache, but can probe the memory 
or cache at each program point. Similar to DATA, the attacker will face many 
noisy observations or can only observe a limited of memory or caches in practice. 
For the project, we assume the attacker can have noise-free observations. 
This threat model captures most of the cache-based and memory-based side channel attacks.

\section{Overview}
The shortcomings of existing work inspire us to design a new tool to detect
and quantify information leakage vulnerabilities in binaries. We capture the 
fined-grained semantics of each secret-dependent control-flow transfers 
and data-accesses. 

\begin{enumerate}
    \item \textit{Execution trace generation.} The design goal of \tana\ is to
    estimate the total information leakage as precisely as possible. 
    \item \textit{Instruction level symbolic execution.}
    \item \textit{Monte Carlo sampling.}
\end{enumerate}
\section{Challenges}

In this section, we articulate several challenges and existing problems
in quantifying the side-channel vulnerability leakages. We describe the
challenges and then briefly present the corresponding solution.

\subsection{Challenge I: Information Leakage Definition}

Existing static-based side-channel quantification works~\cite{182946} define information leakage
as the mutual information or the maximal leakage. These definitions provide a strong security guarantee
when trying to prove a program is secure enough if their methods calculate the program 
leaks zero bits of information.


\begin{figure}[h!]
\centering
    % \vspace{-5mm}
\begin{lstlisting}[xleftmargin=.03\textwidth,xrightmargin=.01\textwidth]
unsigned char key = input();
// key = [0 ... 255]
if(key = 128)
    A(); // branch 1
else if (key < 64)
    B(); // branch 2
else if (key < 128)
    C(); // branch 3
else
    D(); // branch 4
\end{lstlisting}
\caption{Side-channel leakage}
\label{code::entropy}
    % \vspace{-5mm}
\end{figure}

However, the above definition is less useful to justify the sensitive level of leakage sites. 
Considering the example code~\ref{code::entropy}, if an attacker knows the
code executes branch A by some observations, the attacker can know the key actually equals to 128. 
Suppose it is a dummy password checker, in which case the attacker can fully retrieve the password.
Therefore, the total information leakage should be 8 bits, which equals to the size
of unsigned char. 
According to the mutual definition, however, the leakage will be 1.7 bits. \fixme{use footnote or list the calculation process}. The maximal information
leakage is 2 bits. \fixme{the same with previous}. Both approaches fail to tell how 
much information is leaked during the execution precisely.

The problem with the existing methods is that they are static-based and the 
input values are neglected by the previous definition. 
They assume the attacker runs the program multiple times with many different sensitive 
information as the input. Both the mutual information and the max-leakage give an ``average" 
estimate of the information leakage. However, it isn't the typical scenario for an adversary to 
launch an side-channel attack. When a side-channel attack happens, the adversary wants 
to retrieve the sensitive information, in which case the sensitive information is fixed (e.g. AES keys). 
The adversary will run the attack over and over again and guess the value bit by bit. Like the 
previous example, the existing static method does not work well in those situations.

\begin{figure}
  \centering
   \includegraphics[width=.9\columnwidth]{./figures/RA.pdf}
   \caption{some caption here. \fixme{redraw the figure. i will give you the instruction.}}
\end{figure}

\textbf{Our Solution to Challenge I:}
In the project, we hope to give a very precise definition of information leakages. 
Suppose an attacker run the target program multiple times with one fixed input, we
want to know how much information he can infer by observing the memory access patterns.
We come to the simple slogan ~\cite{10.1007/978-3-642-00596-1_21} %% where the information
%% leakage equals:
%% \textbf{Initial uncertainty - remaining uncertainty}
that
\begin{align*}
 & \mathit{Information\ leakage} = \\
 & ~~~~~~ \mathit{Initial}\ \mathit{uncertainty} - \mathit{Remaining\ uncertainty}. 
\end{align*}


If an adversary has zero knowledge about the input before the attack. The initial uncertainty
equals to the size the input. As for the remaining uncertainty, we come to the original definition
of the information content.
We quantify the information leakage with the following definition. 

\newtheorem{mydef}{Definition}

\begin{mydef}
\label{def}
Given a program $P$ with the input set $K$, 
an adversary has the observation $o$ when the input $k{\in}K$. 
We denote it as
    $$P(k) = o.$$
The leakage $L_{P(k)=o}$ based on the oberservation is
    $$L_{P(k)=o} = \log_2{|K|} - \log_2{|K^o|}$$
    where
    $$K^o = \{k^{'} | k^{'}{\in}K \ \text{and} \ P(k^{'}) = o \}$$
\end{mydef}

With the new definition, if the attacker observes that the code~\ref{code::entropy} runs the branch 1, 
then the $K^{o^{1}} = \{128\}$. Therefore, the information leakage $L_{P(k)=o^{1}} = log_2{256} - log_2{1} = 8$
bits, which means the key is totally leaked. If the attacker observes the code runs other
branches, the leaked information is shown in the following table.

\begin{table}[h]
    \centering
    \resizebox{.7\columnwidth}{!}{
    \begin{tabular}{|c|c|c|c|c|}
    \hline
    Branch & 1 & 2  & 3  & 4   \\ \hline
    $K^o$   & 1 & 64 & 64 & 127 \\ \hline
    $L_{P(k)=o}$(bits)   & 8 & 2  & 2  & 1   \\ \hline
    \end{tabular}
    }
    \caption{The leaked information by the definition~\ref{def}}
\end{table}

With the same definition~\ref{def}, if the attacker observes that the code run branch 2, the information
leakage will be 2 bits. The conclusion is consistent with the intuition. Because if the branch 2 was
executed, we can know the key is less than 64. So we know the most and the second significant digits of 
the value key equals to $128$.

\subsection{Challenge II: How to Combine the Leakage Information from Multiple Leak Sites}
Real-world software can have various side-channel vulnerabilities. Those vulnerabilities 
may spread in the whole program. An adversary may exploit more than one side-channel vulnerabilities 
to gain more information~\cite{7163052, 191010}. In order to precisely quantify the
total information leakage, we need to know the relation of those leakage sites. 


\lstinputlisting[language=c, 
                 numbers=left,
                 numbersep=5pt,                   % how far the line-numbers are from the code
                 caption={Multiple leakages},
                 frame = single,
                 captionpos=b,
                 label={code::multiple},
                 basicstyle=\fontsize{7}{9}\selectfont\ttfamily]
                 {sample_code/motivation_multiple.c}

                

Consider the running example in ~\ref{code::multiple}, in which $k1$, $k2$ and $k3$ are
the sensitive key. The code has six different leakage. Leakage 1, 2, 3 are the secret-dependent
data accesses and leakage 4, 5, 6 are the secret-dependent control-flow transfers.  
The attacker can infer the last three digits of
$k1$, $k2$, $k3$ from leakage 1, 2, 3. So those leakages are independent. For leakage 1, 4, 6, however,
we have no idea about the total information leakage.


Suppose one program has two side-channel vulnerabilities A and B, which can leak $L_A$ and $L_B$ bits respectively
according to the definition~\ref{def}. 
Depending on the relation between A and B, the total leaked information $L_{\mathit{total}}$ will be:

\subsubsection{Independent Leakages}
If A and B are independent leakages, the total information leakage will be:
$$L_{\mathit{total}} = L_A + L_B $$

\subsubsection{Dependent Leakages}
If A and B are dependent leakages, the total information leakage will be:
$$\max{\{L_A, L_B\}}  \leq L_{\mathit{total}} < L_A + L_B$$

\subsubsection{Mutual Exclusive Leakages}
If A and B are mutual exclusive leakages, then only A or B can be observed for one fixed input.
$$L_{\mathit{total}} = 
\begin{cases}
L_A, & \text{only} ~ A \\
L_B, & \text{only} ~ B
\end{cases}$$

According to above definition, leakage 1, 2, 3 are independent leakages. Leakage 4, 5
are mutual exclusive leakages. 
For real-world applications, it is hard to estimate the total leaked information for the following reasons.
First, the real-world applications have more than thousands of lines of code. One leakage site leaks the temporary value. 
But the value contains some information about the original buffer. It is hard to know how the 
the sensitive value affects the temporary value. Second, some leakages sites may be
dependent. The occurrence of the first affects the occurrence of the second sites. We 
can't simply add them up. Third, leakage sites are in the different blocks of the 
control-flow graph, which means that only one of the two leakages site may be executed
during the exectution.

\textbf{Our Solution to Challenge II:}
Given a program $P$ with $k$ as the sensitive input, 
we use $k_i$ to denote the sensitive information, where $i$ is the index of the byte in the original buffer.  
We can represent each temporal
values with a formula. There are two types of values in the formula: the concrete value and
the symbolic value. We use the runtime information to simplify the formula. In other words,
we only use symbolic values to represent the sensitive input. For other values that are
independent from the sensitive input, we use the concrete value from the runtime information. 

After that, we model each leakage sites as a math formulas.
The attacker can retrieve the sensitive information by observing the different patterns in 
control-flows and data access when the program process different sensitive information. 
We refer them as the secret-dependent control flow and secret-dependent data access accordingly.
For secret-dependent control transfers, we model the leakage using the path conditions that cause the control
transfer. For secret-dependent memory accesses, we use a symbolic formula $F(\vec{K})$ to
represent the memory address and check if different sensitive inputs can lead to different
memory accesses. As long as we model each leakage with a formula. We can regard multiple leakges as the conjunction of
those formulas. 

\subsection{Challenge III: Scalability and Performance}

After we transfer each potential leaks sites into formula. We can group several formulas together
to estimate the total information leakage. One naive way is to use the Monte Carlo sampling  estimate the
number of input keys. With the definition ~\ref{def}, we can estimate the total information leakage.

However, some pre-experiments show that above approach suffers from the unberable cost, which impede its usage
to detect and quantify side-channel leakages in real-world applications. 
We systematically analyze the performance bottlenecks of the whole process. In general, the performance suffers
from the two following reasons. 
\begin{itemize}
    \item Symbolic Execution (Challenge III(a))
    \item The Monte Carlo Sampling  (Challenge III(b))
\end{itemize}

\subsubsection{Symbolic Execution}
Symbolic execution interprets each instruction and update the memory cells and registers with a 
formula that captured the semantics of the execution. Unfortunately, the number of machine instructions 
are huge and the semantics of each instruction is complex. For example, the Intel Developer Manual~\cite{intelsys}
introduces more than 1000 different X86 instructions. It is tedious to manually implement the
rules for every instructions.

Therefore, existing binary analysis tools ~\cite{shoshitaishvili2016state, 10.1007/978-3-642-22110-1_37} 
will translate machine instructions into intermediate languages (IR). The IR typically has fewer 
instructions compared to the original machine instructions. The IR layer designs, which significantly
simplify the implementations, also introduce significant overhead as well~\cite{217563}.

\textbf{Our Solution to Challenge III(a):}
We adopt the similar approach from~\cite{217563} and implement the symbolic execution 
directly on the top X86 instructions.

\subsubsection{Monte Carlo Sampling}
\label{MCreasons}
For an application with $m$ bytes secret, there are total $2^{8m}$ possible inputs. Of the
$2^{8m}$ possible inputs, we want to estimate the number of inputs that satisfy those formulas.
Then we can use the definition ~/ref{def} to calculate the information leakage.

A Monte Carlo method for approximating the number of $|K_o|$ is to pick up 
$M$ random values and check how many of them satisfy those constrains. If $l$ values
satisfy those constrains, then the approximate result is $\frac{l*2^{8m}}{M}$.

However, the number of satisfying values could be exponentially small. Consider the formula
$F={k_1} = 1\land{k_2} = 2\land{k_3} = 3\land{k_4} = 4$, $k_1$, $k_2$, $k_3$ and $k_4$ each represents
one byte in the orginal sensitive input, there is only one possible solution of $2^{32}$ possible
values, which requres exponentially many samples to get a tight bound. 
The naive Monte Carlo Method also suffers from the curse of dimensionality. For example, 
the libjpeg libraries can transfer the image from one format into another format. One image could
be 1kMB. If we take each byte in the original buffer as symbols, the formula can have at most
1024 symbols. 

\textbf{Our Solution to Challenge III(b):}
We adopt the Markov Chain Monte Carlo to estimate the number of possible input
that satisfies the logic formula groups. The key idea is that we have one group of input that satisfies
the logic formula constrains.  We will
introduce the method in the following sections.

\section{Design}
In this section, we will explain the design decisions to realize A. 

Binary code vs source code:

Many existing works find side-channels vulnerabilities from source code level or intermediate languages (e.g., LLVM IR). Those approaches will have the following questions. First, many compilers can translate the operator into branches. For example, the GCC compiler will translate the ! operator into conditional branches. If the branch is secret-dependent, the attacker could learn some sensitive information. But those source-based methods will fail to detect those vulnerabilities. Second, some if else in the source code can be converted into single conditional instructions (e.g., cmov). Source-based methods will still regard them as potential leakages, which will introduce false positives.

Intermediate Language vs X86 


The tool takes in an unmodified binary and the marked sensitive information as the input. The tool will first start with logging the execution trace. Then the tool will symbolizes the sensitive information into multiple symbols and start with the symbolic execution. During the symbolic execution, the tool will find any potentials leakage sites as well as the path constraints. For each leakage site, whether it is  the tool will also generate the constraint. Then, the tool will split the constraints into multiple group. After that, the tool will run monte carlo sampling to estimate the total information leakages.

\subsection{Trace Logging}
The trace information can be logged via some emulators (e.g., QEMU) or Binary Instrumentation Tools. For our project, we write an Intel Pin Tool to record the execution traces. The trace data has the following information:
Each instruction mnemonics and its memory address
The operands of each instruction and their values
The memory address of sensitive information and its length
The value of eflags register
Most software developers stores sensitive information in an array, a variable or a buffer, which means that those data is stored in a contiguous area in the memory. We use the symbol information in the binary to track the address in the memory.

\subsection{Instruction Level Symbolic Execution}
The main purpose of the this step is to generate constraints of the input sensitive information for the execution trace. If we give the target program a new input which is different from the origin input that was used to generate the execution trace but still satisfies those constraints, the new execution trace will still have the same control flow and data access patterns. 

The tool runs the symbolic execution on the top of the execution traces. At the beginning of the symbolic execution, the tool creates fresh symbols for each byte in the sensitive buffer. For other data in the register or memory at the beginning of the symbolic execution, we use the concrete value from the runtime information collected in the previous step. During the symbolic execution, the tool will maintain a symbol and a concrete value for every variables in the memory and registers. The formula is made up with concrete values and the input key as the symbols calculated through the symbolic execution. For each formula, the tool will check weather it can be reduced into a concrete values. If so, the tool will only use the concrete values in the following symbolic execution.

\subsection{Verification and Optimizayion}
We run the symbolic execution on the top of x86 instructions to achieve the better performance and accuracy of the memory model. In other words, we don’t rely on any intermediate languages to simplify the symbolic execution. While the implementation itself has a lot of benefits, we need to implement the symbolic execution rules for more than one thousand x86 instructions. However, due to the complexity of X86, it is inevitable to make mistakes. Therefore, we verify the correctness of the symbolic execution. The tool will collect the runtime information (Register values, memory values) and compare them with the values generated from the symbolic execution. Whenever the tool finishes the symbolic execution of each instruction, the tool will compare the formula for each symbol and its actual value. If the two values don’t match, we check the code and fix the error. Also, if the formula doesn’t contain the any symbols, the tool will use the concrete value instead of symbolic execution.

\subsection{Secret-dependent control-flows}
An adversary can infer sensitive information from secret dependent control-flows. 
There are two kinds of control-transfer instructions: the unconditional control transfer instructions and the conditional transfer instructions. The unconditional instructions, like CALL, JUMP, RET transfer control from one code segment location to another. Since the transfer is independent from the input sensitive information, an attacker was not able to infer any sensitive information from the control-flow. So the unconditional control transfer doesn’t leak any information based on our threat model. During the symbolic execution, we just update the register information and memory cells with the new formulas.

The conditional transfer instructions, like conditional jump, may or may not transfer control, depending the CPU state. For any conditional jump, the CPU will test if certain condition flag (e.g., CF = 0, ZF =1) is true or false and jump to the certain branches respectively. So the symbolic engine will compute the flag and represent the flag in a symbol formula. Because we are running on a symbolic execution on a execution trace, we know which branch executes. If a conditional jump uses the CPU status flag, we will generate the constraint accordingly.

For examples,

\begin{lstlisting}
...
0x0000e781      add dword [local_14h], 1
0x0000e785      cmp dword [local_14h], 4
0x0000e789      jne 0xe7df
0x0000e78b      mov dword [local_14h], 0

\end{lstlisting}

At the beginning of the instruction segment, the value at the address of local14h can be written as F(K). At the address e785, the value will be updated with F(K)+1. Then the code compare the value with 4 and use the result as a conditional jump. Based on the result, we can have the following constrain:

F(K) + 1 = 4
 
\subsection{Secret-dependent data access}
Like control-flows, an adversary can also infer sensitive information from the data access pattern as well. We try to find this kind of leakages by checking every memory operand of the instruction. We generate the memory addressing formula. As discussed before, every symbols in the formula is the input key. If the formula doesn’t contain any symbols, the memory access is independent from the input sensitive information and won’t leak any sensitive information according to our threat model. Otherwise, we will generate the constraint for the memory addressing. 


Monte Carlo Volume Sampling
From the above steps, we already have the constraints from the execution trace. The only variables in those constraints is the sensitive data. An adversary who wants to infer the sensitive data based on side-channel attacks can’t observe the sensitive information directly. The adversary, however, can observe the memory access pattern of the software.  

We use the Monte Carlo Sampling to calculate the ratio of input that satisfies those constraints.
We will estimate how much information is actually leaked from each input key.

Normalizations
The goal of the normalizations is to simplify the constraints. Each constraint will be evaluated multiple times during the following sampling, we would like to make those formula simpler to reduce the whole execution time.  Each formula is implemented as a abstract syntax tree. We apply a series normalization rules (e.g. key1 xor key1 = 0) to simplify the formula.


Key1 + 1 + 2 + key3 < key2 => key1 + 3 < key2

Split the independent constraints into multiple groups
Each constraints may have multiple symbols as the input. If each two formulas have complete different inputs, then those leakages modeled by the constraints are independent. 

Multiple Stage Monte Carlo Sampling

\section{Discussions and Limitations}
In the section, we discuss \tool's limitations, usages, and
some future works.

\tool{} works on the native x86 execution traces. The design,
which is very precise in terms of true leakages compared to other
static source code method~\cite{197207,BacelarAlmeida:2013:FVS:2483313.2483334},
also suffers from limitations of dynamic approaches as well.
\tool{} is not sound and has coverage problem. Each time we only
get one single execution trace. Therefore, we may neglect
some side-channel vulnerabilities on other traces. However,
we argue that it is not a crucial problems for analyzing crypto
libraries. Because crypto libraries are designed to have the same
code coverage for various inputs. Our evaluation also confirms
the above point. For symmetric encryptions during our evaluation,
there is no secret-dependent control-flow transfers. RSA implementations
have several secret-dependent control-flow transfers. But after we
manually check those leakages cites. We find most of them are useful
for bound checking, which do not leak much information and have
negligible effects on the whole code coverage as well.

One of the motivations of \tool{} is that while recent works
have reported lots of tentative side-channel vulnerabilities,
most of them are unpatched by developers. Our evaluation result
also confirms it. For RSA, the latest OpenSSL\@ only has one leakage
site that can leak more than 3 bits while there are 22 leakage sites
according to \tool{}. DES implementation of OpenSSL\@ has several
sensitive leakages. But given the end life status of DES, it is
still unpatched for the worth of engineering effort. At the early
stage of the project, we hope to find some sensitive leakages but are
neglected by communities for years. But somehow every sensitive leakages
identified by \tool{} are known to people before. We think the main reason is
that we only test famous crypto algorithms in well-known crypto libraries.
Those code bases have been studied for years. So it is unlikely to have
unknown sensitive leakages. We leave the task of applying
\tool{} on other libraries and non-crypto libraries in the future.

%\tool{} works on the native x86 instructions, while
%some existing works~\cite{197207,BacelarAlmeida:2013:FVS:2483313.2483334} 
%find side-channels vulnerabilities from source code level 
%or intermediate languages (e.g., VEX, REIL). Apart from the scalability
%issues for IR implementations, \tool{} is designed to work on the
%native x86 instructions for the following considerations. First, compliers can 
%introduce or mitigate side channels vulnerabilities. 
%For example, the GCC compiler may or may not translate the $!$ operator into conditional branches. 
%If the branch is secret-dependent, the attacker could learn some sensitive information.
%However, source-based methods fail to know how those the machine code looks like, which
%leads to false positives or false negatives.  
%Second, we find many crypto libraries have lots of inlined assembly code. In general, 
%it is hard to convert assembly code into source code
%or IR.



\section{Implementation}
We implement the \tana\ with 12K lines of code in C++11. It has three components, Intel
Pin tool that can collect the execution trace, the intruction-level symbolic execution
engine and the backend that can estimate the information leakage. 

\section{Evaluation}
\label{res_overview}
\begin{table*}[]
    \begin{tabular}{c c c c c c c l}
    \hline
    Program                & Start Function             & Input Size (bits)   & CF   & DA     & Number of Instructions & Process Time (s) \\ \hline
    AES OpenSSL-0.9.7      & AES\_set\_encrypt\_key     &     128             &      &        &                        &                  \\
    DES OpenSSL-0.9.7      & DES\_string\_to\_key       &     64              &      &        &                        &                  \\
    AES OpenSSL-1.1.1      & AES\_set\_encrypt\_key     &     128             &      &        &                        &                  \\
    DES OpenSSL-1.1.1      & DES\_string\_to\_key       &     64              &      &        &                        &                  \\
    AES mbedTLS-2.5        & mbedtls\_aes\_setkey\_enc	&     128             &      &        &                        &                  \\
    DES mbedTLS-2.5        & mbedtls\_des\_setkey\_enc  &     64              &      &        &                        &                  \\
    RSA mbedTLS-2.5        & mbedtls\_pk\_parse\_key    &     1024            &      &        &                        &                  \\
    AES mbedTLS-2.15       & mbedtls\_aes\_setkey\_enc  &     128             &      &        &                        &                  \\
    DES mbedTLS-2.15       & mbedtls\_des\_setkey\_enc  &     64              &      &        &                        &                  \\
    wget-1.18              & sock\_read                 &     494             &      &        &                        &                  \\
    GTK                    &                            &                     &      &        &                        &                  \\
    \hline
    \end{tabular}
\end{table*}
We evaluate \tool{} with the real-world crypto libraries and non-crypto libraries. 
For crypto libraries, we choose OpenSSL, mbedTLS and NaCl. 
OpenSSL and mbedTLS are the two most commonly used
crypto libraries in today's software. NaCl (pronounced "salt") is a 
new software libreary for encryption, decryption and signatures, etc.
NaCl is designed to have no data flow from secrets to load address and no data 
flow from secrets to branch conditions. Therefore, NaCl should have no leakages
under our attack model. 

For non-crypto libraries, we study libjepg, GTK, and wget.
JPEG is a commonly used lossy image c ompression standard, and
libjpeg is a popular library for handling the JPEG image data
format. Previously, researchers have introduced controlled-channel
attacks, which allow attackers to ret rieve outlines of JPEG images
from applications. We also study GTK and wget with \tool{}. GTK 
is a widely used cross-platform toolkit for creating graphical user
interfaces. And wget is free software that can retrieve information
via HTTPS, HTTP, and FTP.

We build the source code into 32-bit x86 Linux executables with the 
GCC 8.0 on Ubuntu 14.04. Although our tool can
work on stripped binaries, we use symbol information to track
back leakage sites in the source code. We use Intel Pin version 3.7 
to record the execution trace. We run our experiments on a 2.90GHz
Intel Xeon(R) E5-2690 CPU with 128GB memory.
During the evaluation, we are interested in the following two
aspects:
\begin{enumerate}
    
    \item  Can \tool{} precisely
    report the number of leaked bits in open source libraries?
    \item  Recent work has reported a number
    of side-channel vulnerabilities in open source libraries. 
    Is the number of leaked bits reported by \tool{} useful to justify 
    the sensitive level of side-channel vulnerabilities?
   
\end{enumerate}

\subsection{Evaluation Result Overview}
In this section, we present an overview of the evaluation result. 
\tool{} find xx leakages in total from real-world cryptosystems source libraries.
 Among the xx leak points, xx of them are leaked due
to secret-dependent control-flow transfers and xx of them are leaked 
due to secret-dependent memory accesses. 

For crypto libraries, \tool{} finds that secret-dependent memory accesses 
cause most leakages. 
\tool{} also identifies that most side-channel vulnerabilities 
leak very little information in practice, which confirms our initial
assumptions. 
However, we do find some sensitive leakages. 
Some of them have been confirmed by existing research that those 
vulnerabilities can be exploited to realize real attacks. 

All the symmetric key implementations in OpenSSL and mbedTLS all yield
a significant leakage due to the implementation of the lookup table
to speed up the computation. Every leakages found during the evaluation
belongs to the type of secret-dependent memory accesses. We believe that
the secret-dependent control-flow transfers have been widely studied in
the past few years, and developers have patched most of those leakages. 

\tool{} find eight leakage sites for the implementation of DES in OpenSSL.
Each leakage can leak one bit of information from one byte. Also, our tool
confirms those leakages are independent, so the total information leakage
is 8 bits. We check the source code and find the least significant
bit in each byte is used for the parity. Therefore, \tool{} can confirm
that even the key length of DES is 64 bits. However, only 56 bits of the 
DES key is valid.


\subsection{Information Leakage Quantification}
\subsection{Analysis of Software Countermeasures}
\subsubsection{Bit-slicing}
\subsubsection{Scatter and Gather}
\subsection{Case Studies}
\subsubsection{libjpeg}


\section{Related Work}

There is a vast amount of work on 
side channel 
detection~\cite{182946, 236338, Brotzman19Casym, 203878,217537,Wichelmann:2018:MFF:3274694.3274741,langley2010ctgrind}, 
  %side channel
mitigation~\cite{Page2005PartitionedCA,
Wang:2007:NCD:1250662.1250723,Zhang:2015:HDL:2775054.2694372,Li:2014:SLH:2541940.2541947,
236344,shih2017t,Coppens:2009:PMT:1607723.1608124,
brickell2006software,crane2015thwarting}, 
and information
quantification~\cite{10.1007/978-3-642-31424-7_40,McCamantE2008,5207642,Phan:2012:SQI:2382756.2382791,Chattopadhyay:2017:QIL:3127041.3127044}, 
Here we only present the closely related work to ours.
Due to space limit, we do not include side channel attack work.

\subsection{Detection}

There are a large number of works on side-channel vulnerability detections in
recent years.  CacheAudit~\cite{182946} uses abstract domains to compute the
over approximation of cache-based side-channel information leakage upper bound.
However, due to over approximation they make, CacheAudit can indicate the
program is side-channel free if the program has zero leakage.  However, it is
less useful to judge the sensitive level of the side-channel leakage based on
the leakage provided by CacheAudit. CacheS~\cite{236338} improves the work of
CacheAudit by proposing the novel abstract domains, which only track
secret-related code. Like CacheAudit, CacheS cannot provide the information to
indicate the sensitive level of side-channel vulnerabilities.
CacheSym~\cite{Brotzman19Casym} introduces a static cache-aware symbolic
reasoning technique to cover multiple paths for the target program. Still, their
approaches cannot assess the sensitive level for each side-channel
vulnerability.

The dynamic approach, usually comes with taint analysis and symbolic execution,
can perform a very precise analysis. CacheD~\cite{203878} takes a concrete
program execution trace and run the symbolic execution on the top of the trace
to get the formula of each memory address. During the symbolic execution, every
value except the sensitive key uses the concrete value. Therefore, CacheD is
quite precise in term of false positives. We adopted a similar idea to model the
secret-dependent memory accesses. DATA~\cite{217537} detects address-based
side-channel vulnerabilities by comparing different execution traces under
various test inputs. MicroWalk~\cite{Wichelmann:2018:MFF:3274694.3274741} uses
mutual information (MI) between sensitive input and execution state to detect
side-channels. They can only detect control-flow channels and MI scores are less
meaningful for dynamic analysis.

\subsection{Mitigation}
Both hardware~\cite{Page2005PartitionedCA,
Wang:2007:NCD:1250662.1250723,Zhang:2015:HDL:2775054.2694372,Li:2014:SLH:2541940.2541947,
236344} and software~\cite{shih2017t,Coppens:2009:PMT:1607723.1608124,
brickell2006software,crane2015thwarting} side-channels mitigation methods have
been proposed recently. Hardware countermeasures, including parting the hardware
computing resource~\cite{Page2005PartitionedCA}, randomizing cache
accesses~\cite{Wang:2007:NCD:1250662.1250723, 236344}, and designing new
architecture~\cite{tiwari2011crafting}, which need to change the hardware and is
usually hard to adopt in reality. On the contrary, software approaches are
usually easy to implement. Coppens et
al.~\cite{Coppens:2009:PMT:1607723.1608124} introduced a compiler-based approach
to eliminate key-dependent control-flow transfers. Crane et
al.~\cite{crane2015thwarting} mitigated side-channels by randomizing software.
As for crypto libraries, the basic idea is to eliminate key-dependent
control-flow transfers and data accesses. Common approaches include
bit-slicing~\cite{konighofer2008fast,rebeiro2006bitslice}, unifying
control-flows~\cite{Coppens:2009:PMT:1607723.1608124}.

\subsection{Quantification}

Quantitative Information Flow (QIF) aims at providing an information leakage
estimation for the sensitive information given the public output. If zero bit
of the information is leaked, the program is called non-interference. McCamant
and Ernst~\cite{McCamantE2008} quantify the information leakage as the network
flow capacity. Backes et al.~\cite{5207642} proposes an automated method for QIF
by computing an equivalence relation on the set of input keys. But the approach
cannot handle real-world programs with bitwise operations. Phan et
al.~\cite{Phan:2012:SQI:2382756.2382791} propose symbolic QIF. The goal of their
work is to ensure the program is non-interference. They adopt an over
approximation way of estimating the total information leakage and their method
does not work for secret-dependent memory access side-channels.
CHALICE~\cite{Chattopadhyay:2017:QIL:3127041.3127044} quantifies the leaked
information for a given cache behavior. CHALICE symbolically reason about cache
behavior and estimate the amount of leaked information based on cache miss/hit.
Their approach can only scale to small programs, which limits its usage in
real-world applications. On the contrary, \tool{} can assess the sensitive level
of side-channels with different granularities. It can also analyze side-channels
in real-world crypto libraries.





\bibliographystyle{IEEEtran}
\bibliography{refs}

\end{document}


