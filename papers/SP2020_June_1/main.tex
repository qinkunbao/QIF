
\documentclass[conference]{IEEEtran}


\usepackage{listings}
\usepackage{epsfig}
\usepackage{url}
\usepackage{cite}
\usepackage{fancybox}
\usepackage{amsthm}

\newcommand{\tana}{\textsc{TANA}}

\begin{document}

\title{TANA: Fined-grained Side-channel Inforamtion Leakage Quantification in Binaries}
\author{Anonymous}

\maketitle

\begin{abstract}
Side-channel attacks allow attackers to infer some sensitive information based on 
non-functional characteristics. Existing works on address-base side-channel detection 
can provide a list of potential side-channels leakage sites. We observe that those 
works still have the following limitations: 1) Many software may have multiple information 
leakage sites. Some vulnerabilities could be more severe than others. But existing work 
couldn’t tell the difference between those leakages. 2)  An attacker usually exploits multiple 
leakages at one time. However, no existing tool can report how much information is leaked 
in total.

To overcome the above limitations: we proposed a tool called TANA, which can not only 
find the side-channel vulnerabilities but can estimate how many bits are actually leaked 
through the leakage. TANA works in three steps. First, the application is executed to record the 
trace. Second, TANA runs the instruction level symbolic execution on the top of the 
execution trace. TANA will find side-channel information leakages and model each leakage 
as one unique math constraints. Finally, TANA will classify those constraints into 
independent multiple groups and run the multiple step monte carlo to estimate the 
information leakage. TANA can report a very fined-grained vulnerability result 
compared to existing tools. 

We apply the tool on OpenSSL, MbedTLS and libjpeg and find several serious side channel 
vulnerabilities. We also evaluate the vulnerabilities from previous research. The result 
confirms our intuition: 
indicates most of the reported vulnerabilities are actually hard to exploit in practice.

\end{abstract}

\IEEEpeerreviewmaketitle
\pagenumbering{arabic}

\section{Introduction}
%% side channels are important
Side channels are inevitable in modern computer systems as the sensitive information 
may be leaked by many kinds of inadvertent behaviors, 
such as power, electromagnetic radiation and even sound~\cite{xxx}. 
Among them, software-based side channels, such as cache attacks, memory page attacks,
and controlled-channel attacks, are especially common 
and have been studied for years~\cite{xxx}. 
These vulnerabilities result from vulnerable software and shared hardware components.
By observing the outputs or hardware behaviors, attackers can
infer the program execution flow that manipulate secrets and 
guess the secrets such as encryption keys~\cite{xxx}.

%% to deal with side channels, we can protect or detect them and detection is better
Various countermeasures have been proposed to defend against 
software-based side-channel attacks. Hardware level solutions, 
including reducing shared resources, adopting oblivious RAM, and using
transnational memory~\cite{182946,203878,217537} need new hardware features or changes
to modern complex computer systems, which is impractical and hard to adopt in 
reality. Therefore, a more promising and universal direction is software countermeasures, 
detecting and eliminating side channel vulnerabilities from code.

Regarding the root cause of software-based side channels, 
many of them are caused by the following two specific types: 
data flow from secrets to load addresses and data flow from secrets to branch conditions.
We call them secret-dependent control-flow and memory-access correspondingly.
Therefore, a central problem is identifying those two code patterns automatically.
Recent works~\cite{203878,data,caches} adopt static and dynamic analysis
to detect side-channels.
They can find many potential leak sites in real-world software, 
but fail to report how severe a potential leakage could be. 
Many of the reported vulnerabilities are typically hard to exploit
and leak very little information. For example, DATA~\cite{xxx} reports
2,246 potential leakage site for the RSA implementation in OpenSSL\@.
After some inspectations, 1,510 are dismissed, but it still
leaves 278 control-flow and 460 data-access patterns. For software
developers, it is hard for them to fix all those vulnerabilities,
let alone the majority of them are negligible.
While some vulnerabilities can be used to recover the full secret
keys~\cite{xxx}, many other vulnerabilities prove to be less serious in reality.

To assess the sensitive level of side-channel vulnerabilities, we need a proper 
quantification metric.
Static methods, usually with abstract interpretation, can give a leakage upper bound, 
which is useful to justify the implementation is secure when they report zero or little leakage. 
However, they cannot indicate how serious the leakage is because of over-approximation~\cite{xxx}. 
For example, CacheAudit~\cite{xxx} reports that the upper bound leakage of AES-128 exceeds 
the original key size! The dynamic methods take another approach with a concrete input and 
run the program in real environment. Although they are very precise in term of true leakages, 
no existing tool can precisely assess the severity of the vulnerabilities they discover.

To overcome these limitations, we propose a novel method
to quantify information leakage more precisely. 
Different from previous works, which only consider the
``average'' information leakage, we study the problem based on real attack scenarios.
The average information assumes that the target program will have \emph{variable} sensitive 
information when an attack is launched.
However, for real-world attacks, an adversary may run the target problem again and over again 
with \emph{fixed} unknown sensitive information such as the key. 
Therefore, the previous threat model cannot catch real attack scenarios.
In contrast, our method is more precise and fine-grained. 
We quantify the amount of leaked information as the cardinality of the set of 
possible inputs based on attackers' observations. 

Before an attack, an adversary has a big but finite input space.
Every time when the adversary observes a leakage site, he can eliminate some 
potential inputs and reduce the size of the input space. 
The smaller the input space is, the more information is actually gained. 
In an extreme case, if the size of the input space reduces to one, 
the adversary can determine the input information uniquely, which means all the secret information
(e.g., the whole secret key) is leaked. By counting the number of distinct inputs, 
we can quantify the information leakage more precisely. 

We use constraints to model the relation between the original sensitive input and
each leakage site. We run the instruction level symbolic execution on the whole
execution trace to generate the constraints. Symbolic execution can provide the fine-grained
information but is usually believed to be an expensive operation in terms of performance. 
Therefore, existing dynamic symbolic execution based works~\cite{xxx} either only analyze 
small programs or apply some domain knowledges to simplify the execution. We systematically
analyze the bottleneck of the symbolic execution and optimzie it scalable to
real-world cryptosystems. 

We apply the above technique and build a tool called \tool{},\footnote{\tool\ is a horse that can ``count''.
Our tool uses an advanced method to count the number of leaked bits from side channels.}
which could discover potential information leakage sites 
as well as estimating how many bits they can leak for each leakage site. 
We assume that adversaries can exploit secret-dependnet control-flow transfers and 
data-access patterns when the program processes different sensitive data. 
%We refer them as the potential information leakage sites. 
First, we collect the dynamic execution trace for each input of the target libraries 
and then run symbolic execution on the traces. 
In this way, we model each side-channel leakage as a math formula. 
The sensitive input is divided into several independent bytes and each byte is regarded as 
a unique symbol. Those formulas can precisely model side-channel vulnerabilities.
Then we extend the problem to multiple leakages and related leakages
and introduce a monte carlo sampling method to estimate the single and combined information leakage.
In fact, if an application has a different sensitive input but still satisfies the formula, 
the code can still leak the same information. 


%Based on the fixed attack target, we classify the software-based side-channel 
%vulnerabilities into two categories: 1.\textit{secret-dependent control-flow transfers} 
%and 2.\textit{secret-dependent data accesses} and model them with math formulas which
%constrain the value of sensitive information.
%We quantify the amount of leaked information as the number of possible solutions that are
%reduced after applying each constrains.


%Our method can identify and quantify address-based
%sensitive information leakage sites in real-world applications automatically. 
%Adversaries can exploit different control-flow transfers and data-access patterns when 
%the program processes different sensitive data. We refer them as the potential information
%leakage sites. Our tool can discover and estimate those potential information leakage sites 
%as well as how many bits they can leak. We are also able to report precisely how many bits
%can be leaked in total if an attacker observes more than one site.
%We run symbolic execution on execution traces. We model each side-channel leakage as a math formula. 
%The sensitive input is divided into several independent bytes and each byte is regarded as 
%a unique symbol. Those formulas can precisely model every the side-channel vulnerability. 
%In other words, if the application has a different sensitive input but still satisfies the formula, 
%the code can still leak the same information.  
%Those information leakage sites may spread in the whole program 
%and their leakages may not be dependent. Simply adding them up can only get a coarse upper bound 
%estimate. In order to accurately calculate the total information leakage, we must know the 
%dependent relationships among those multiple leakages sites. Therefore, we introduce a 
%monte carlo sampling method to estimate the total information leakage.

We apply \tool{} on both symmetric and asymmetric ciphers from real-world crypto libraries including OpenSSL and
mbedTLS. The experimental result confirms that \tool{} can precisely identify the previous known vulnerabilities,
reporting how much information is leaked and which byte in the original sensitive buffer is leaked. 
Although some of the analyzed crypto libraries have a number of side-channels, they actually
leak very little information. Also, we perform the analysis of widely deployed software countermeasures
against side channels.
Finally, we present new vulnerabilities. With the help of \tool{}, we confirm those
vulnerabilities are easily to be exploited. Our results are superisingly different compared to previous results
and much more useful in practice.

In summary, we make the following contributions:

\begin{itemize}
	\item We propose a novel method that can quantify fine-grained leaked information from side-channel
        vulnerabilities. We model each side-channel vulnerabilities as math formulas and 
        mutiple side-channel vulnerabilities can be seen as the conjunction of those formulas, which
        precisely models the program semantics.
        \item We transfer the information quantification problem into a probabilty distribution problem and 
        use the Monte Carlo sampling method to estimate the information leakage. Some initial results indicate the 
        the sampling method suffers from the curse of dimensionality problem. We therefore design a guided
        sampling method and provide the corresponding error esitimate.
	\item We implement the proposed method into a practical tool and apply it on several real-world software. \tool{} 
        successfully identifies the address-based side-channel vulnerabilities and provides the corresponding
        information leakge. The information leakage result provides the detailed information that help developers to
        fix the reported vulnerabilities.
\end{itemize}

\section{Background}
In this section, we first present a basic introduction about the 
memory-based side-channel attack. Those attacks 
are exactly what we attempt to study in the paper. After that we 
will present existing work on side-channel detection and quantification.
We will also analyze strengths and limitations of those quantification 
methods.

\subsection{Address-based Side-Channels}
Address-based side-channels are information channels that can leak sensitive information unintended
through the different behaviors when the program accesses different memory addrsss. Fundamentally,
those difrences were caused by the memory hierarchy design in modern computer systems. When the 
CPU fetches the data, it will first search the cache, which stores copies of the data from 
the frequently used main memory. If the data doesn't exist in the cache, the CPU will read
the data from the main memory (RAM). Classified by the layer caused the side-channel, we 
introduce two kinds of commom side-channels: cache-based side-channel attacks and memory-based
side-channel attacks.

\subsubsection{Cache-based Attack}
In general, the cached-based side-channel attacks seek information 
rely on the time differences between the cache miss
and cache hit. Here we introduce two types of cache attacks:
PRIME+PROBE, FLUSH+RELOAD.

\textbf{PRIME+PROBE} targets a single cache set. It has two phases. During the
"prime" phase, the attacker fills the cache set will his own data.
In the second "probe" phase, the attacker accesses the cache set
again. If the victim accesses the cache set and evicts part of 
the data, the attacker will experience a slow mesurement. If not, 
the mesurement will be fast.

\textbf{FLUSH+RELOAD} targets a single cache line. 
It requires the attacker and victim share the same memory.
It also have two phases. During the "flush" phase, the attacker 
will flush the "monitered memory" from the cache. Then the attacker
wait for the victim to access the memory. In the third phase, the 
attacker reload the "monitered memory". If the time is short, which
indicates there is a cache hit and the victim reolads the memory before. 
On the other hand, the time will be longer since the CPU need to reolad
the memory into the cache line. 

\subsubsection{Memory-based Attack}
Memory-based side-channel attack\cite{} exploits the different behaviors when the
program accesses different page tables. The controlled-channel attack\cite{7163052},
which works in the kernel space, can infer the sensitive data in the shielding systems by
observing the page fault sequences by restricting some code and
data pages. 

After examing the memory-based side-channels attack. We find the fundamentally
reason of those attacks are due to secret-dependent memory access and control
flow transfers.
\lstinputlisting[language=c, 
                 numbers=right,
                 caption={Sample code shows secret-dependent memory access and 
                          secret-dependent control-flow transfer.},
                 captionpos=b,
                 label={code:background},
                 basicstyle=\fontsize{7}{9}\selectfont\ttfamily]
                 {sample_code/background.c}

For exampls, the above code~\ref{code:background} show an simple encryption function that
has the two kinds of side-channels. At the line 11, dependending on the value of key,
the code will access the different entry in the predefined table /textbf{Table}. At the
line 13, the code will do a series of computation and determine if the code in the if
branch is executed or not. Such vulnerabilities could leak to the memory-based 
side-channles. We identify and quantify the leakage of the two kinds of vulnerabilities 
in the paper.

\subsection{Information Leakage Quantification}
Given an event e which occurs with the probability $P(e)$, if the event e happens, 
then we receive
\begin{equation}
    I = - log(P(e))
\end{equation}
bits of information by knowing the event e.

The above definition is obvious. Suppose a char variable \textit{a} in C program has the size
of one byte (8 bits), so the value in the variable can range from 0 - 255. We assume
the \textit{a} has the uniform distribution. If at one time we observe the \textit{a}
equals to 1, the probability will be 1/256. So the information we get is 
$-log(1/256) = 8 bits$, which is exactly the size of the char variable in C program.

Existing works on information leakage quantification are based on mutual information or 
min-entropy \cite{10.1007/978-3-642-00596-1_21}.
In their frameworks, the input sensitive
information $K$ is viewed as random variables. Let $k_i$ be one of the possible
value of $K$. The Shannon entropy $H(K)$ is defined by
\begin{equation}
    H(K) = - \sum_{k_i {\in} K}P(k_i)log(P(k_i))
\end{equation}

The Shannon entropy can be used to quantify the initial uncertainty about the sensitive
information. Suppose a program (P) with the $K$ as
the sensitive input, an adversary has some observations (O) through the side-channels.
In this work, the observations are referred to the secret-dependent control-flows and
secret-dependent data-accesses patterns. The conditional entropy $H(K|O)$ is
\begin{equation}
    H(K|O) = - \sum_{o_j {\in} O} {P(o_j) \sum_{k_i {\in} K}{P(k_i|o_j)log(P(k_i|o_j))}}
\end{equation}
Intuitively, the conditional information marks the uncertainty about $K$ after the adversary
has gained some observations (O). 

Many previous works use the mutual information $I(K; O)$ to quantify the leakage which is defined 
as follows:
\begin{equation}
    Leakage = I(K;O) = \sum_{k_i {\in} K}{\sum_{o_j {\in} O}{P(k_i, o_j)log(\frac{P(k_i, o_j)}{P(k_i)P(o_j)})}}
\end{equation}
where $P(k_i, o_i)$ is the joint discrete distribution of $K$ and $O$.
Alternatively, the mutual information can also be computed with the following equation:
\begin{equation}
    Leakage = I(K;O) = H(K) - H(K|O) = H(O) - H(O|K)
\end{equation} 
For a deterministic program, once the input $K$ is fixed, the program will have the same
control-flow transfers and data-access patterns. As a result, $P(k_i, o_j)$ will always
equals to 1. So the conditional entropty $H(O|K)$ will equal to zero. Now the leakage defined
by the mutual information can be simplified into:
\begin{equation}
    Leakage = I(K;O) = H(O)
\end{equation}
In other words, once we know the distribution of those memory-access patterns. We can 
calculate how much information is actually leaked.

Another common method is based on maximal leakage ~\cite{10.1007/978-3-642-00596-1_21,10.1007/978-3-642-31424-7_40,182946}.

\begin{equation}
    Leakage = log(C(O))
\end{equation}
$C(O)$ represents the number of different observations that an attacker can have.

Now we provide a concrete example to show how the two types of quantification definition works.
\lstinputlisting[language=c, 
                 numbers=right,
                 caption={A simple program},
                 captionpos=b,
                 label={code::entropy},
                 basicstyle=\fontsize{7}{9}\selectfont\ttfamily]
                 {sample_code/dependent.c}

\textbf{Maximal leakage} 
Dependenting on the value of key, the code can run four different branches which corrosponding to 
four different observations. Therefore, by the maximal leakage definition, the leakage equals to 
$log4 = 2$ bits.

\textbf{Mutual Information} If the key satisfies the uniform distribution, the probability of the code runs each branch
can be computed with the following result: 
\begin{table}[h]
\centering
\begin{tabular}{|c|c|c|c|c|}
\hline
Branch & A     & B      & C      & D       \\ \hline
P      & 1/256 & 64/256 & 64/256 & 127/256 \\ \hline
\end{tabular}
\caption{The distribution of observations}
\end{table}
Therefore, the leakage equals to 
$\frac{1}{256}log\frac{1}{256} + \frac{1}{4}log\frac{1}{4}*2 + \frac{127}{256}log\frac{127}{256} = 1.7$ bits.

\section{Threat Model}
We consider an attacker who shares the same hardware resource with the victim. 
The attacker attempt to retrieve sensitive information via memory-based side-channel attack. 
The attacker has no direct access to the memory or cache, but can probe the memory 
or cache at each program point. Similar to DATA, the attacker will face many 
noisy observations or can only observe a limited of memory or caches in practice. 
For the project, we assume the attacker can have noise-free observations. 
This threat model captures most of the cache-based and memory-based side channel attacks.

\section{Motivation}
\subsection{Techinical Chanllenges}
In this section, we articulate several chanllenges and existing problems
in quantifying the side-channel vulnerability leakages. We briefly describe the
chanllenges and then present the corresponding solutions.

\subsubsection{Information Leakage Definition}
Existing static-based side-channel quantification works defined information leakage
as the mutual information or the max leakage. Those definitions provide strong security guarantee
when trying to show a program is secure if the their methods say the program leaks zero bits of
information.
However, the above definition is less useful when if the program has some leakages. 
Considering the example in section ~\ref{code::entropy}, if an attacker observes the
code runs branch A, the attacker can know the key actually equals to 128. Suppose it is 
a dummy password checker, in which case the attacker can fully retrieve the password.

The problem with the existing method is that they are static based. 

We want to have a way that can 
very precisely estimate how much information are leaked for each execution. 
Apparently,the existing method fails to achieve our requirement.
Suppose an attacker can observe which branch the program executes (e.g., flush reload, controlled attack), 
the above code may leak different amount of information depending on the value of input.
For example, if the attacker knows the code executes the branch 0, 
then he will know the key equals 11*2 = 22. 
As the key has 4 bits information in total, we think the amount of information leakage here is 4 bits. 
On the other hand, if the code executes the branch 2, 
the only thing that an attacker can know is that the key doesn’t equal 22, 
which leaks very little information in this case.  

Solution: We model each leakage as one unique formula. 
We come to the original information content definition and use the probability to
mark the information leakage.   

\subsection{Multiple Leakage Sites}
Real-world software usually can have multiple side-channel vulnerabilities. Those vulnerabilities 
may spread in the whole program. An adversary may exploit more than one side-channel vulnerabilities 
to achieve the attack. For example, the controled-side channel attack \cite{7163052} \cite{191010}, the author 
demonstrate an attack against a popular spell checking tool, Hunspell. By observing four sets 
of secret-dependent memory accesses sites in two functions $HashMgr::addword$ and $HashMgr::lookup$, 
the author can recover the word that Hunspell checks.

For the Hunspell, the attacker manually study the source code of Hunspell, figure out
the relation of those vulnerabilities and launch the attack. In order to precisely quantify the
total information leakage, we need to know the relation of those leakage sites. 

Suppose one program have two side-channel vulnerabilities A and B, which leaks $L_A$ and $L_A$ bits
during the exectution. The total information leakage is noted as $L_{Total}$. The relation between
A and B has the following two cases.

\subsubsection{Independent Leakages}
If A and B are independent leakages, the total information leakage will be:
\begin{equation}
\label{independent leakage}
    L_{total} = L_A + L_B
\end{equation} 

\subsubsection{Dependent Leakages}
If A and B are dependent leakages, the total information leakage will be:
\begin{equation}
\label{dependent leakage}
    \max{\{L_A, L_B\}}  <= L_{total} < L_A + L_B
\end{equation}

\subsection{Scalability}



We use the mutual information (MI) to quantify the information leakages. 
For a program P with sensitive information K as the input, the attacker may have some observations O during the execution. 
The information leakage is defined as the mutual information I(O; K) between O and K.
\begin{equation}
I(O; K) = H(O) - H(O|K)
\end{equation}

I(O; K) represent how much uncertainty about K can be reduced if the attacker has the observation O.
For a deterministic program, the program will have the same memory access behavior as long as the input is fixed. 
As the observation of the attacker correlate to the memory access behavior, 
we can have the following formula.
\begin{equation}
O = f(K)
\end{equation}

The function f is determined by the program P. For our project, we can calculate the function f via symbolic execution.
\begin{equation}
I(O; K) = H(O) - H(f(K)|K) = H(O) = H(f(K))
\end{equation}

So the mutual information between O and S equals to the self information of O. 
\begin{equation}
H(O) = Σp(Oi)log(p(Oi))
\end{equation}

For a determinist program, we can calculate the distribution of O as long as we know the distribution of input K. So we can calculate how much information is leaked.

For examples, given a program P, we have the sensitive input K. The K should be a value in a memory cell or a sequential buffer (e.g., an array). We use ki to denote the sensitive information, where i is the index of the byte in the original buffer.  We can have the following equations. The t1, t2, t3, is the temporary values during the execution.
\begin{equation}
t_1 = f1(k1, k2, k3 ... kn)\\
t_2 = f2(k1, k2, k3 ... kn)\\
t_3 = f3(k1, k2, k3 ... kn)\\
tm = fm(k1, k2, k3 ... kn)
\end{equation}

The attacker can retrieve the sensitive information by observing the different patterns in control-flows and data access when the program process different sensitive information. We refer them as the secret-dependent control flow and secret-dependent data access accordingly.

\subsection{Secret-dependent Control Flow}
Here is an example of the secret-dependent control-flows. Consider the code snippet in List 1. Here the key is the confidential data. The code will have different behaviours (time, cache access) dependenting on which branch is actually executing. By observing the behaviour, the attacker can infer which branch actually executed and know some of the sensitive information. One of the famous leakage example is the square and multiply in many RSA implementations. 

For example, the attacker knows the key equals to zero if he observes the code run the branch1. Because key has 256 different possibilities. The original key has lg256 = 8 bits information. If the attacker can observe the code run branch 1. Then he will knows the key equals to zero. If the code run branch 2, the attacker can infer the key doesn’t equal to zero. 

Branch 1
temp = 0xb;
0 =< key <= 256;
temp = key/2;

Information Leakage = -log(1/p) = -log(1/256) = 8 bits

Branch 2
temp != 0; 
0 =< key <= 256;
temp = key/2;

Information Leakage = -log(255/256) bits

\subsection{Seret-dependent Memory Access}

\begin{lstlisting}

T[64]; // Lookup tables with 64 entries
index = key % 63;
temp = T[index]; 
// Secret-dependent memory access       

\end{lstlisting}

The simple program above is an example of secret dependent memory access. Here T is a precomputed tables with sixty-four entries. Depending on the values of key, the program may access any values in the array. Those kind of code patterns may wildly exist in many crypto and media libraries. 

Suppose the attackers observe the code accesses the first entry of the lookup tables. We can have the following formulas.

key mod 63 ≡ 1
0 =< key <= 256

So the key can be one of the following values:
1 64 127 190 253

Information leakages = -log(5/256) =  5.6 bits


\section{Design}
In this section, we will explain the design decisions to realize A. 

Binary code vs source code:

Many existing works find side-channels vulnerabilities from source code level or intermediate languages (e.g., LLVM IR). Those approaches will have the following questions. First, many compilers can translate the operator into branches. For example, the GCC compiler will translate the ! operator into conditional branches. If the branch is secret-dependent, the attacker could learn some sensitive information. But those source-based methods will fail to detect those vulnerabilities. Second, some if else in the source code can be converted into single conditional instructions (e.g., cmov). Source-based methods will still regard them as potential leakages, which will introduce false positives.

Intermediate Language vs X86 


The tool takes in an unmodified binary and the marked sensitive information as the input. The tool will first start with logging the execution trace. Then the tool will symbolizes the sensitive information into multiple symbols and start with the symbolic execution. During the symbolic execution, the tool will find any potentials leakage sites as well as the path constraints. For each leakage site, whether it is  the tool will also generate the constraint. Then, the tool will split the constraints into multiple group. After that, the tool will run monte carlo sampling to estimate the total information leakages.

\subsection{Trace Logging}
The trace information can be logged via some emulators (e.g., QEMU) or Binary Instrumentation Tools. For our project, we write an Intel Pin Tool to record the execution traces. The trace data has the following information:
Each instruction mnemonics and its memory address
The operands of each instruction and their values
The memory address of sensitive information and its length
The value of eflags register
Most software developers stores sensitive information in an array, a variable or a buffer, which means that those data is stored in a contiguous area in the memory. We use the symbol information in the binary to track the address in the memory.

\subsection{Instruction Level Symbolic Execution}
The main purpose of the this step is to generate constraints of the input sensitive information for the execution trace. If we give the target program a new input which is different from the origin input that was used to generate the execution trace but still satisfies those constraints, the new execution trace will still have the same control flow and data access patterns. 

The tool runs the symbolic execution on the top of the execution traces. At the beginning of the symbolic execution, the tool creates fresh symbols for each byte in the sensitive buffer. For other data in the register or memory at the beginning of the symbolic execution, we use the concrete value from the runtime information collected in the previous step. During the symbolic execution, the tool will maintain a symbol and a concrete value for every variables in the memory and registers. The formula is made up with concrete values and the input key as the symbols calculated through the symbolic execution. For each formula, the tool will check weather it can be reduced into a concrete values. If so, the tool will only use the concrete values in the following symbolic execution.

\subsection{Verification and Optimizayion}
We run the symbolic execution on the top of x86 instructions to achieve the better performance and accuracy of the memory model. In other words, we don’t rely on any intermediate languages to simplify the symbolic execution. While the implementation itself has a lot of benefits, we need to implement the symbolic execution rules for more than one thousand x86 instructions. However, due to the complexity of X86, it is inevitable to make mistakes. Therefore, we verify the correctness of the symbolic execution. The tool will collect the runtime information (Register values, memory values) and compare them with the values generated from the symbolic execution. Whenever the tool finishes the symbolic execution of each instruction, the tool will compare the formula for each symbol and its actual value. If the two values don’t match, we check the code and fix the error. Also, if the formula doesn’t contain the any symbols, the tool will use the concrete value instead of symbolic execution.

\subsection{Secret-dependent control-flows}
An adversary can infer sensitive information from secret dependent control-flows. 
There are two kinds of control-transfer instructions: the unconditional control transfer instructions and the conditional transfer instructions. The unconditional instructions, like CALL, JUMP, RET transfer control from one code segment location to another. Since the transfer is independent from the input sensitive information, an attacker was not able to infer any sensitive information from the control-flow. So the unconditional control transfer doesn’t leak any information based on our threat model. During the symbolic execution, we just update the register information and memory cells with the new formulas.

The conditional transfer instructions, like conditional jump, may or may not transfer control, depending the CPU state. For any conditional jump, the CPU will test if certain condition flag (e.g., CF = 0, ZF =1) is true or false and jump to the certain branches respectively. So the symbolic engine will compute the flag and represent the flag in a symbol formula. Because we are running on a symbolic execution on a execution trace, we know which branch executes. If a conditional jump uses the CPU status flag, we will generate the constraint accordingly.

For examples,

\begin{lstlisting}
...
0x0000e781      add dword [local_14h], 1
0x0000e785      cmp dword [local_14h], 4
0x0000e789      jne 0xe7df
0x0000e78b      mov dword [local_14h], 0

\end{lstlisting}

At the beginning of the instruction segment, the value at the address of local14h can be written as F(K). At the address e785, the value will be updated with F(K)+1. Then the code compare the value with 4 and use the result as a conditional jump. Based on the result, we can have the following constrain:

F(K) + 1 = 4
 
\subsection{Secret-dependent data access}
Like control-flows, an adversary can also infer sensitive information from the data access pattern as well. We try to find this kind of leakages by checking every memory operand of the instruction. We generate the memory addressing formula. As discussed before, every symbols in the formula is the input key. If the formula doesn’t contain any symbols, the memory access is independent from the input sensitive information and won’t leak any sensitive information according to our threat model. Otherwise, we will generate the constraint for the memory addressing. 


Monte Carlo Volume Sampling
From the above steps, we already have the constraints from the execution trace. The only variables in those constraints is the sensitive data. An adversary who wants to infer the sensitive data based on side-channel attacks can’t observe the sensitive information directly. The adversary, however, can observe the memory access pattern of the software.  

We use the Monte Carlo Sampling to calculate the ratio of input that satisfies those constraints.
We will estimate how much information is actually leaked from each input key.

Normalizations
The goal of the normalizations is to simplify the constraints. Each constraint will be evaluated multiple times during the following sampling, we would like to make those formula simpler to reduce the whole execution time.  Each formula is implemented as a abstract syntax tree. We apply a series normalization rules (e.g. key1 xor key1 = 0) to simplify the formula.


Key1 + 1 + 2 + key3 < key2 => key1 + 3 < key2

Split the independent constraints into multiple groups
Each constraints may have multiple symbols as the input. If each two formulas have complete different inputs, then those leakages modeled by the constraints are independent. 

Multiple Stage Monte Carlo Sampling

\section{Discussions and Limitations}
In the section, we discuss \tool's limitations, usages, and
some future works.

\tool{} works on the native x86 execution traces. The design,
which is very precise in terms of true leakages compared to other
static source code method~\cite{197207,BacelarAlmeida:2013:FVS:2483313.2483334},
also suffers from limitations of dynamic approaches as well.
\tool{} is not sound and has coverage problem. Each time we only
get one single execution trace. Therefore, we may neglect
some side-channel vulnerabilities on other traces. However,
we argue that it is not a crucial problems for analyzing crypto
libraries. Because crypto libraries are designed to have the same
code coverage for various inputs. Our evaluation also confirms
the above point. For symmetric encryptions during our evaluation,
there is no secret-dependent control-flow transfers. RSA implementations
have several secret-dependent control-flow transfers. But after we
manually check those leakages cites. We find most of them are useful
for bound checking, which do not leak much information and have
negligible effects on the whole code coverage as well.

One of the motivations of \tool{} is that while recent works
have reported lots of tentative side-channel vulnerabilities,
most of them are unpatched by developers. Our evaluation result
also confirms it. For RSA, the latest OpenSSL\@ only has one leakage
site that can leak more than 3 bits while there are 22 leakage sites
according to \tool{}. DES implementation of OpenSSL\@ has several
sensitive leakages. But given the end life status of DES, it is
still unpatched for the worth of engineering effort. At the early
stage of the project, we hope to find some sensitive leakages but are
neglected by communities for years. But somehow every sensitive leakages
identified by \tool{} are known to people before. We think the main reason is
that we only test famous crypto algorithms in well-known crypto libraries.
Those code bases have been studied for years. So it is unlikely to have
unknown sensitive leakages. We leave the task of applying
\tool{} on other libraries and non-crypto libraries in the future.

%\tool{} works on the native x86 instructions, while
%some existing works~\cite{197207,BacelarAlmeida:2013:FVS:2483313.2483334} 
%find side-channels vulnerabilities from source code level 
%or intermediate languages (e.g., VEX, REIL). Apart from the scalability
%issues for IR implementations, \tool{} is designed to work on the
%native x86 instructions for the following considerations. First, compliers can 
%introduce or mitigate side channels vulnerabilities. 
%For example, the GCC compiler may or may not translate the $!$ operator into conditional branches. 
%If the branch is secret-dependent, the attacker could learn some sensitive information.
%However, source-based methods fail to know how those the machine code looks like, which
%leads to false positives or false negatives.  
%Second, we find many crypto libraries have lots of inlined assembly code. In general, 
%it is hard to convert assembly code into source code
%or IR.



\section{Implementation}
We implement the \tana\ with 12K lines of code in C++11. It has three components, Intel
Pin tool that can collect the execution trace, the intruction-level symbolic execution
engine and the backend that can estimate the information leakage. 


\bibliographystyle{IEEEtran}
\bibliography{refs}

\end{document}


