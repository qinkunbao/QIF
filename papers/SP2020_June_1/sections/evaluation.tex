\section{Evaluation}
\label{res_overview}
 
\begin{table*}
\centering
\caption{Evaluation results overview. We evaluate two versions of mbedTLS and five
versions of OpenSSL\@. CF represents secret-dependent control-flow transfers and
DF represents secret-dependent data-flow transfers. Side-channel leakagas can
be found by symbolic execution and we run Monte Carlo to estimate the amount 
of leakage information. A summary of all vulnerabilities with the amount of
leak information can be found in the appendix.
}\label{fig:Testtt}
\begin{tabular}{clrrrrrrr}
\hline
\textbf{Algorithm} & \textbf{Implementation} & \textbf{Lekage Sites} & \textbf{CF} & \textbf{DF}
& \textbf{\# Instructions} & \textbf{Max Leakeage} & \textbf{Sym.\ Exe.} & \textbf{Monte Carlo}\\\hline
&&&&&& bits & ms & ms\\\cline{7-9}
AES & mbed TLS 2.5   & 68 & 0 & 68 & 39,855 & 8 & 570 ~~&   850 ~~\\
AES & mbed TLS 2.15  & 68 & 0 & 68 & 39,855 & 8 & 550 ~~&   829 ~~\\
AES & openssl 0.9.7  & 75 & 0 & 75 & 1,704 & 10 & 319 ~~& 7,720 ~~\\
AES & openssl 1.0.2f & 88 & 0 & 88 & 1,350 & 12 &  72 ~~& 1,500 ~~\\
AES & openssl 1.0.2k & 88 & 0 & 88 & 1,350 & 11 &  83 ~~& 1,441 ~~\\
AES & openssl 1.1.0f & 88 & 0 & 88 & 1,420 & 12 &  87 ~~& 1,454 ~~\\
AES & openssl 1.1.1  & 88 & 0 & 88 & 1,586 & 8 &   91 ~~& 1,250 ~~\\
DES & mbed TLS 2.5   & 15 & 0 & 15 & 4,596 & 1 &  114 ~~&   144 ~~\\
DES & mbed TLS 2.15  & 15 & 0 & 15 & 4,596 & 1 &  106 ~~&   137 ~~\\
DES & openssl 0.9.7  & 6 & 0 & 6 & 2,976 & 7 & 149 ~~& 4,193       ~~\\
DES & openssl 1.0.2f & 8 & 0 & 8 & 2,593 & 9 & 239 ~~& 5,311       ~~\\
DES & openssl 1.0.2k & 8 & 0 & 8 & 2,593 & 9 & 235 ~~& 5,080        ~~\\
DES & openssl 1.1.0f & 8 & 0 & 8 & 4,260 & 9 & 256 ~~& 5,027        ~~\\
DES & openssl 1.1.1  & 6 & 0 & 6 & 8,272 & 7 & 235 ~~& 4,584       ~~\\
&&&&&&& minutes & minutes\\\cline{8-9}
RSA & mbed TLS 2.5   & 6 & 6 & 0 & 22,109,246 & 9      & 38 ~~& 20  ~~\\
RSA & mbed TLS 2.15  & 12 & 0 & 12 & 24,484,441 & 9    & 39 ~~& 241  ~~\\
RSA & openssl 0.9.7  & 105 & 103 & 2 & 16,980,109 & 13 & 28 ~~& 266 ~~\\
RSA & openssl 1.0.2f & 38 & 27 & 11 & 14,468,307 & 10  & 28 ~~& 160  ~~\\
RSA & openssl 1.0.2k & 36 & 27 & 9 & 15,285,210 & 12   & 39 ~~& 282   ~~\\
RSA & openssl 1.1.0f & 31 & 22 & 9 & 16,390,750 & 13   & 32 ~~& 262 ~~\\
RSA & openssl 1.1.1  & 26 & 20 & 6 & 18,207,020 & 12   &  7 ~~& 455 ~~\\\hline
Total &              & 883 &205& 678& 128,042,089&     & 213m    ~~& 1,688m ~~\\\hline
\end{tabular}
\end{table*}

We evaluate \tool{} with the real-world crypto libraries and non-crypto libraries. 
For crypto libraries, we choose OpenSSL, mbedTLS and NaCl. 
OpenSSL and mbedTLS are the two most commonly used
crypto libraries in today's software. NaCl (pronounced "salt") is a 
new software library for encryption, decryption and signatures, etc.
NaCl is designed to have no data flow from secrets to load address and no data 
flow from secrets to branch conditions. Therefore, NaCl should have no leakage
under our attack model. 

We build the source code into 32-bit x86 Linux executables with the 
GCC 8.0 on Ubuntu 14.04. Although we use use symbol information to track
back leakage sites in the source code, our tool can
work on stripped binaries as well. We use Intel Pin version 3.7 
to record the execution trace. We run our experiments on a 2.90GHz
Intel Xeon(R) E5-2690 CPU with 128GB RAM memory.
During our evaluation process, we are interested in the following two
aspects:
\begin{enumerate}
    \item  Is \tool{} effective to detect side-channels in real-world
    crypto systems?
    \item  Can \tool{} precisely
    report the number of leaked bits in open source libraries?
    \item  Recent work has reported a number
    of side-channel vulnerabilities in open source libraries. 
    Is the number of leaked bits reported by \tool{} useful to justify 
    the sensitive level of side-channel vulnerabilities?
   
\end{enumerate}

\subsection{Evaluation Result Overview}
In this section, we present an overview of the evaluation result. 
\tool{} find 883 leakages in total from real-world crypto system libraries.
Among those 883 leak points, 205 of them are leaked due
to secret-dependent control-flow transfers and 678 of them are leaked 
due to secret-dependent memory accesses. 

For crypto libraries, \tool{} finds that secret-dependent memory accesses 
cause most leakages. 
\tool{} also identifies that most side-channel vulnerabilities 
leak very little information in practice, which confirms our initial
assumptions. 
However, we do find some sensitive leakages. 
Some of them have been confirmed by existing research that those 
vulnerabilities can be exploited to realize real attacks. 

All the symmetric key implementations in OpenSSL and mbedTLS all yield
significant leakages due to the implementation of the lookup table
to speed up the computation. Every leakages found during the evaluation
belongs to the type of secret-dependent memory accesses. We believe that
the secret-dependent control-flow transfers have been widely studied in
the past few years, and developers have patched most of those leakages. 
One method to address the leakage is to use bit-slicing. We will analyse
the corresponding countermeasure in the following sections.

\tool{} find several leakage sites for both the implementation of DES and AES
in OpenSSL and mbedTLS. \tool{} confirm that all those leakages come from
table lookups implementations. mbed TLS 2.15 and 2.5 have the same implementation
of DES and AES so they have the same leakage report. One proper fix would be 
a scalar bit-sliced implementations. However, we don't see the bit-sliced 
implementation of AES and DES in various versions of OpenSSL and mbedTLS.  
However, we find the new implementation of OpenSSL instead use typical four 1K
tables. It only uses 1K of tables. This implementation is rather easy but is
still vulnerable to a side channel attack. However, the countermeasures do
somehow decrease the total amount of leaked information.

\subsection{Information Leakage Quantification}
\subsection{Overview}
\subsection{Analysis of Software Countermeasures}
\subsubsection{Bit-slicing}
\subsubsection{Scatter and Gather}
