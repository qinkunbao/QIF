\section{Evaluation}

\begin{table*}[t]
    \begin{tabular}{c c c c c c c c l}
    \hline
    Program                   & Implementation                     & Start Function         & Key Size (bits)     & CF                    & DA                    & Total Leakage     & Number of Instructions & Process Time (s) \\ \hline
    AES                       & OpenSSL 0.9.7                      & AES\_set\_encrypt\_key &                       &                       &                       &                       &                        &                  \\
    DES                       & OpenSSL 0.9.7                      & DES\_string\_to\_key   &                       &                       &                       &                       &                        &                  \\
    RSA                       & OpenSSL 0.9.7                      &                        &                       &                       &                       &                       &                        &                  \\
    AES                       & OpenSSL 1.1.1                      &                        &                       &                       &                       &                       &                        &                  \\
    DES                       & OpenSSL 1.1.1                      &                        &                       &                       &                       &                       &                        &                  \\
    RSA                       & OpenSSL 1.1.1                      &                        &                       &                       &                       &                       &                        &                  \\
    AES                       & mbed TLS 2.5                       &                        &                       &                       &                       &                       &                        &                  \\
    DES                       & mbed TLS 2.5                       &                        &                       &                       &                       &                       &                        &                  \\
    RSA                       & mbed TLS 2.5                       &                        &                       &                       &                       &                       &                        &                  \\
    AES                       & mbed TLS 2.15                      &                        &                       &                       &                       &                       &                        &                  \\
    DES                       & mbed TLS 2.15                      &                        &                       &                       &                       &                       &                        &                  \\
    \hline
    \end{tabular}
\end{table*}
We evaluate \tool{} with the real-world crypto libraries and non-crypto. 
For crypto libraries, we choose OpenSSL and mbedTLS, two most commonly used
crypto libraries in today's software. For OpenSSL, we compile it with the option \textit{no-asm} 
to disable the assembly language routines. For non-crypto libraries, 
we study the libjepg, a commonly used lossy library implemented by 
the Independent JPEG Group. We build the source code into a 32 bit 
ELF binary with the GCC 8.0 on Ubuntu 14.04. Although our tool can
work on stripped binaries, we use the symbol information to trace
back leakage sites into the source code. We use Intel Pin version 3.7 
to record the execution trace.


During the evaluate, we are interested in the following aspects:
\begin{enumerate}
    \item \textbf{Finding Leakage Sites.} How effective is \tool{} in 
    identifying the memory-based side-channels vulnerabilities?
    How much performance overhead does the tool need to find the 
    leakage?
    \item \textbf{Leakage Sites Quantification.} Can \tool{} precisely
    report the information leakage? Is the number of leaked bits reported 
    by \tool{} useful to justify the sensitive level of the side-channel
    vulnerability?
    \item \textbf{Exsiting Leakage Sites.} Recent work has report a number
    of leakage sites in open source crypto libraries. But the crypto
    library authors don't fix them for some reasons. We use \tool{} to
    study those vulnerabilities and try to explain the reason why the 
    author fix or not fix the side channel vulnerabilities.
\end{enumerate}


\subsection{Finding Leakage Sites}
\subsection{Leakage Sites Quantifications}
\subsection{Exsiting Leakage Sites}
