\section{Evaluation}
\label{res_overview}
\begin{table*}[h]
    \begin{tabular}{c c c c c c c c l}
    \hline
    Program                & Start Function             & Input Size (bits)   & CF   & DA     & Maximum Leakage (bits)& Number of Instructions & Process Time (s) \\ \hline
    AES OpenSSL-0.9.7      & AES\_set\_encrypt\_key     &     128             &      &        &                       &                        &                  \\
    DES OpenSSL-0.9.7      & DES\_string\_to\_key       &     64              &      &        &                       &                        &                  \\
    AES OpenSSL-1.1.1      & AES\_set\_encrypt\_key     &     128             &      &        &                       &                        &                  \\
    DES OpenSSL-1.1.1      & DES\_string\_to\_key       &     64              &      &        &                       &                        &                  \\
    AES mbedTLS-2.5        & mbedtls\_aes\_setkey\_enc	&     128             &      &        &                       &                        &                  \\
    DES mbedTLS-2.5        & mbedtls\_des\_setkey\_enc  &     64              &      &        &                       &                        &                  \\
    RSA mbedTLS-2.5        & mbedtls\_pk\_parse\_key    &     1024            &      &        &                       &                        &                  \\
    AES mbedTLS-2.15       & mbedtls\_aes\_setkey\_enc  &     128             &      &        &                       &                        &                  \\
    DES mbedTLS-2.15       & mbedtls\_des\_setkey\_enc  &     64              &      &        &                       &                        &                  \\
    wget-1.18              & sock\_read                 &     494             &      &        &                       &                        &                  \\
    GTK                    &                            &                     &      &        &                       &                        &                  \\
    \hline
    \end{tabular}
\end{table*}
We evaluate \tool{} with the real-world crypto libraries and non-crypto libraries. 
For crypto libraries, we choose OpenSSL, mbedTLS and NaCl. 
OpenSSL and mbedTLS are the two most commonly used
crypto libraries in today's software. NaCl (pronounced "salt") is a 
new software libreary for encryption, decryption and signatures, etc.
NaCl is designed to have no data flow from secrets to load address and no data 
flow from secrets to branch conditions. Therefore, NaCl should have no leakages
under our attack model. We build OpenSSL with the option \textit{no-asm} 
to disable the assembly language routines. 

For non-crypto libraries, we study libjepg, GTK, and wget.
JPEG is a commonly used lossy image compression standard, and
libjpeg is a popular library for handling the JPEG image data
format. Previously, researchers have introduced controlled-channel
attacks, which allow attackers to retrieve outlines of JPEG images
from applications. We also study GTK and wget with \tool{}. GTK 
is a widely used cross-platform toolkit for creating graphical user
interfaces. And wget is free software that can retrieve information
via HTTPS, HTTP, and FTP.

We build the source code into 32-bit x86 Linux executables with the 
GCC 8.0 on Ubuntu 14.04. Although our tool can
work on stripped binaries, we use symbol information to track
back leakage sites in the source code. We use Intel Pin version 3.7 
to record the execution trace. We run our experiments on a 2.90GHz
Intel Xeon(R) E5-2690 CPU with 128GB memory.
During the evaluation, we are interested in the following two
aspects:
\begin{enumerate}
    
    \item \textbf{Leaked Sites Quantification.} Can \tool{} precisely
    report the number of leaked bits in open source libraries?
    \item \textbf{Existing Leakage Sites.} Recent work has reported a number
    of side-channel vulnerabilities in open source libraries. 
    Is the number of leaked bits reported by \tool{} useful to justify 
    the sensitive level of side-channel vulnerabilities?
   
\end{enumerate}

\subsection{Evaluation Result Overview}
In this section, we present an overview of the evaluation result. 
\tool find xx leakages in total from real-world cryptosystems and open
source libraries. Among the xx leak points, xx of them are leaked due
to secret-dependent control-flow transfers and xx of them are leaked 
due to secret-dependent memory accesses. 

For crypto libraries, \tool{} finds that secret-dependent memory accesses 
cause most leakages. We think that secret-dependent
control-flow transfers have been widely studied in recent years, 
and crypto authors have patched those leakages.
\tool{} also identifies that most side-channel vulnerabilities 
leak very little information in practice, which confirms our initial
assumptions. 
However, we do find some sensitive leakages. 
Some of them have been confirmed by existing research that those 
vulnerabilities can be exploited
to realize real attacks. 

\tool{} find 8 leakage sites for both implementations of DES. Our tool
indentifies Those sites can leak 8 bits from the total 64 bits 
input key. We check the location of the leakage, it is turned out that
those 8 bits are used for parity. Therefore, \tool{} can confirm that 
the effective key lenfth of DES is 56 bits.

\subsection{Information Leakage Quantification}
\subsection{Analysis of Software Countermeasures}
\subsubsection{Bit-slicing}
\subsubsection{Scatter and Gather}
\subsection{Case Studies}

