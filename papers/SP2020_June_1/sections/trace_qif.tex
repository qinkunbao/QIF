\section{\tool{} Leakage Definition}
In the section, we discuss how \tool{} quantifies the amount of
leaked information. \tool{} is a dynamic-based approach to 
quantify the information leakage. We will first introduce 
the limitation of existing quantification metric. After
that, we introduce the abstract and notation for the paper 
and propose our method.

\subsection{Problem Setting}
\subsection{Notations}
\subsection{Theoretical Analysis}
Suppose a program with some
sensitive input $K$, an adversary has some observations ($O$) by side channels.  
 %In this work, the observations are referred as the secret-dependent control-flows and
 %secret-dependent data-access patterns. 
  The conditional entropy $H(K|O)$ is
\begin{displaymath}
    H(K|O) = - \sum_{o_j {\in} O} {P(o_j) \sum_{k_i {\in} K}{P(k_i|o_j)\log_2P(k_i|o_j)}}
\end{displaymath}
Intuitively, the conditional information marks the uncertainty about $K$ after the adversary
has gained some observations ($O$). 

Some previous work uses the mutual information $I(K; O)$ to quantify the leakage which is defined 
as follows:
\begin{displaymath}
    \mathit{Leakage} = I(K;O) = \sum_{k_i {\in} K}{\sum_{o_j {\in} O}{P(k_i, o_j)\log_2\frac{P(k_i, o_j)}{P(k_i)P(o_j)}}}
\end{displaymath}
where $P(k_i, o_i)$ is the joint discrete distribution of $K$ and $O$.
Alternatively, the mutual information can also be computed with the following equation:
\begin{displaymath}
    \mathit{Leakage} = I(K;O) = H(K) - H(K|O) = H(O) - H(O|K)
\end{displaymath}

For a deterministic program, once the input $K$ is fixed, the program will have the same
control-flow transfers and data-access patterns. As a result, $P(k_i, o_j)$ will always
equals to 1 or 0. So the conditional entropy $H(O|K)$ will equal to zero. So the leakage defined
by the mutual information can be simplified into:
\begin{displaymath}
\label{mutual:information}
    \mathit{Leakage} = I(K;O) = H(O)
\end{displaymath}
In other words, once we know the distribution of those memory-access patterns. We can 
calculate how much information is actually leaked.

Another common method is based on the maximal leakage~\cite{10.1007/978-3-642-00596-1_21,10.1007/978-3-642-31424-7_40,182946}.
The formal information theory can prove that the maximal leakage is the upper bound of the mutual 
information (Channel Capacity).

\begin{displaymath}
    \mathit{Leakage} = \log(C(O))
\end{displaymath}
Here $C(O)$ represents the number of different observations that an attacker can have. 
