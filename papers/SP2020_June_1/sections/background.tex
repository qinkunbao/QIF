\section{Background}
In this section, we first present a basic introduction about the several kinds of memory-based side-channel attack.
Those attacks are exactly what we attempt to study in the paper. Then we present existing work on side-channel 
detection and quantification.  
\subsection{Address-based Side-Channels}
Address-based side-channels are information channels that can leak sensitive information unintended
through the different behaviors when the program accesses different memory addrsss. Fundamentally,
those difrences were caused by the memory hierarchy design in modern computer systems. When the 
CPU fetches the data, it will first search the cache, which stores copies of the data from 
the frequently used main memory. If the data doesn't exist in the cache, the CPU will read
the data from the main memory (RAM). Classified by the layer caused the side-channel, we 
introduce two kinds of commom side-channels: cache-based side-channel attacks and memory-based
side-channel attacks.

\subsubsection{Cache-based Attack}
In general, the cached-based side-channel attacks seek information 
rely on the time differences between the cache miss
and cache hit. Ge et al \cite{DBLP:journals/jce/GeYCH18} classified
cached-based attacks into three catogories:PRIME+PROBE, FLUSH+RELOAD,
and EVICT+TIME.


\subsubsection{Memory-based Attack}

\subsection{Information Leakage Quantification}

