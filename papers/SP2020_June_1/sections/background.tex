\section{Background}
In this section, we first present a basic introduction about the 
several kinds of memory-based side-channel attack. Those attacks 
are exactly what we attempt to study in the paper. Then we present
existing work on side-channel detection and quantification.  
\subsection{Address-based Side-Channels}
Address-based side-channels are information channels that can leak sensitive information unintended
through the different behaviors when the program accesses different memory addrsss. Fundamentally,
those difrences were caused by the memory hierarchy design in modern computer systems. When the 
CPU fetches the data, it will first search the cache, which stores copies of the data from 
the frequently used main memory. If the data doesn't exist in the cache, the CPU will read
the data from the main memory (RAM). Classified by the layer caused the side-channel, we 
introduce two kinds of commom side-channels: cache-based side-channel attacks and memory-based
side-channel attacks.

\subsubsection{Cache-based Attack}
In general, the cached-based side-channel attacks seek information 
rely on the time differences between the cache miss
and cache hit. Here we introduce two types of cache attacks:
PRIME+PROBE, FLUSH+RELOAD.

\textbf{PRIME+PROBE} targets a single cache set. It has two phases. During the
"prime" phase, the attacker fills the cache set will his own data.
In the second "probe" phase, the attacker accesses the cache set
again. If the victim accesses the cache set and evicts part of 
the data, the attacker will experience a slow mesurement. If not, 
the mesurement will be fast.

\textbf{FLUSH+RELOAD}, on the other hand, targets a single cache line. 
It requires the attacker and victim share the same memory.
It also have two phases. During the "flush" phase, the attacker 
will flush the "monitered memory" from the cache. Then the attacker
wait for the victim to access the memory. In the third phase, the 
attacker reload the "monitered memory". If the time is short, which
indicates there is a cache hit and the victim reolads the memory before. 
On the other hand, the time will be longer since the CPU need to reolad
the memory into the cache line.



\subsubsection{Memory-based Attack}

\subsection{Information Leakage Quantification}

