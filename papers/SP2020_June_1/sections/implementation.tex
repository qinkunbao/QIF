\subsection{Implementation}
We implement \tool{} with 14320 lines of code in C++11. 
It has three components, Intel
Pin tool that can collect the execution trace, 
the instruction-level symbolic execution
engine, and the backend that can estimate the information leakage. 
The tool can also report the memory address of the leakage site. 
To assist the developers fix the bugs, we also have several Python 
scripts that can report the leakage location in the source code with the 
debug information and the symbol information. A sample report can be found
 in the appendix.

\begin{table}[h]
    \centering
%    \resizebox{.8\columnwidth}{!}{
    \caption{\tool{}' main components and sizes}
    \begin{tabular}{lr@{~}@{}l}
    \hline
    Component                            & \multicolumn{2}{c}{Lines of Code (LOC)}    \\ \hline
    Trace Logging                        & 501 lines &of C++       \\ 
    Symbolic Execution                   & 14,963 lines &of C++    \\ 
    Data Flow                            & 451 lines &of C++       \\
    Monte Carlo Sampling                 & 603 lines &of C++       \\ 
    Others                               & 211 lines &of Python    \\ \hline
    Total    & 16,729 lines & \\\hline
    \end{tabular}
%    }
\end{table}

Our current implementation supports part of the Intel 32-bit instructions, 
including bitwise operations, control transfer, data movement, and logic 
instructions, which are essential in finding memory-based side-channel vulnerabilities. For other 
instructions the current implementation does not support, 
the tool will use the real values to update the registers and memory cells.
Therefore, the tool may miss some leakages but will not give us any new
false positives with the implementation.
