\section{Implementation}
We implement the \tool{} with 12K lines of code in C++11. It has three components, Intel
Pin tool that can collect the execution trace, the intruction-level symbolic execution
engine and the backend that can estimate the information leakage. The tool can also report 
the memory address of the leakage site. To assist the developers fix the bugs more easiliy,
we also have several Python scripts that can report the leakage location in the source code 
with the debug information and the symbol information. A sample report can be found at the 
appendix.

\begin{table}[h]
    \centering
    \resizebox{.8\columnwidth}{!}{
    \begin{tabular}{ll}
    \hline
    Component                            & Lines of code    \\ \hline
    Trace Logging                        & 101 LoC of C++   \\ 
    Symbolic Execution & 11547 LoC of C++ \\ 
    Monte Carlo Sampling                 & 603 LoC of C++   \\ 
    Others                               & LoC of Python    \\ \hline
    \end{tabular}
    }
    \caption{\tool{}'s main components and their LoC}
\end{table}

Our current implementation can support part of the Intel 32-bit intructions that are important 
in finding memory-based side-channels vulnerabilities, which include bitwise operations, 
control transfer, data movement and logic instructions. For other 
instructions the current implementation doesn't support, 
the tool will use the concrete values to update the registers and memory cell.
Therefore, the tool may miss some leakages but won't give us any new false positives with the
implementation.