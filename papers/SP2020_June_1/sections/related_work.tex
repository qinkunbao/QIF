\section{Related Work}
\subsection{Side-channel vulnerabilities detection}

There are a large number of works on side-channel vulnerability detections in recent years.
CacheAudit~\cite{182946} uses abstract domains to compute the overapproximation of cache-based
side-channel information leakage upper bound. However, due to over approximation, the leakage
given by CacheAudit can indicate the program is side-channel free if the program has zero leakage. 
However, it is less useful to judge the sensitive level of the side-channel leakage based on the
leakage provided by CacheAudit. CacheS~\cite{236338} imporves the work of CacheAudit by proposing 
the novel abstract domains, which only precisely track secret related code. Like CacheAudit, CacheS
can't provide the information to indicate the sensitive level of side-channel vulnerabilities.

The dynamic approach, usually comes with taint analysis and symbolic execution, can perform a very 
precise analysis. CacheD~\cite{203878} takes a concrete program execution trace and run the symbolic
execution on the top of the trace to get the formula of each memory address. During the symbolic
execution, every value except the sensitive key uses the concrete value. Therefore, CacheD is quite 
precise in term of false positives. We adopted the similar idea to model the  secret-dependent memory 
accesses.  DATA~\cite{217537} detects address-based side-channel vulnerabilities by comparing 
execution traces under various tests. 

\subsection{Quantification}
Quantitative Information Flow (QIF) aims at providing an information leakage estimate for the
sensitive information through program analysis. 