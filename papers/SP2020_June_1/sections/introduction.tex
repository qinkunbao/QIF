\section{Introduction}
%% side channels are important
Side channels are inevitable in modern computer systems as the sensitive information 
may be leaked by many kinds of inadvertent behaviors, 
such as power, electromagnetic radiation and even 
sound~\cite{agrawal2002side,kar20178,chari1999towards,217605,genkin2014rsa}. 
Among them, software-based side channels, such as cache attacks, memory page attacks,
and controlled-channel attacks, are especially common 
and have been studied for years~\cite{7163052,217543,217589,lee2017inferring,191010,liu2015last}. 
These vulnerabilities result from vulnerable software and shared hardware components.
By observing the outputs or hardware behaviors, attackers can
infer the program execution flow that manipulate secrets and 
guess the secrets such as encryption keys~\cite{Osvik2006,Gullasch:2011:CGB:2006077.2006784,203878,10.1007/978-3-540-45238-6_6}.

%% to deal with side channels, we can protect or detect them and detection is better
Various countermeasures have been proposed to defend against 
software-based side-channel attacks. Hardware level solutions, 
including reducing shared resources, adopting oblivious RAM, and using
transnational memory~\cite{203878,217537,shih2017t,Zhang:2015:HDL:2775054.2694372} need new hardware features or changes
to modern complex computer systems, which is impractical and hard to adopt in 
reality. Therefore, a more promising and universal direction is software countermeasures, 
detecting and eliminating side channel vulnerabilities from code.

Regarding the root cause of software-based side channels, 
many of them are caused by the following two specific types: 
data flow from secrets to load addresses and data flow from secrets to branch conditions.
We call them secret-dependent control-flow and memory-access correspondingly.
Therefore, a central problem is identifying those two code patterns automatically.
Recent works~\cite{203878,217537,Wichelmann:2018:MFF:3274694.3274741,Brotzman19Casym,236338,182946} 
adopt static and dynamic analysis
to detect side-channels.
They can find many potential leak sites in real-world software, 
but fail to report how severe a potential leakage could be. 
Many of the reported vulnerabilities are typically hard to exploit
and leak very little information. For example, DATA~\cite{217537} reports
2,246 potential leakage site for the RSA implementation in OpenSSL\@.
After some inspections, 1,510 are dismissed, but it still
leaves 278 control-flow and 460 data-access patterns. For software
developers, it is hard for them to fix all those vulnerabilities,
let alone the majority of them are negligible.
While some vulnerabilities can be used to recover the full secret
keys~\cite{184415}, many other vulnerabilities prove to be less serious in reality.

To assess the sensitive level of side-channel vulnerabilities, we need a proper 
quantification metric.
Static methods~\cite{182946,5207642}, usually with abstract interpretations, can give a leakage upper bound, 
which is useful to justify the implementation is secure when they report zero or little leakage. 
However, they cannot indicate how serious the leakage is because of over-approximation. 
For example, CacheAudit~\cite{182946} reports that the upper bound leakage of AES-128 exceeds 
the original key size! The dynamic methods take another approach with a concrete input and 
run the program in real environment. Although they are very precise in term of true leakages, 
no existing tool can precisely assess the severity of the vulnerabilities they discover.

To overcome these limitations, we propose a novel method
to quantify information leakage more precisely. 
Different from previous works, which only consider the
``average'' information leakage, we study the problem based on real attack scenarios.
The average information assumes that the target program will have \emph{variable}
or \emph{random} sensitive 
information when an attack is launched.
However, for real-world attacks, an adversary may run the target problem again and over again 
with \emph{fixed} unknown sensitive information such as the key. 
Therefore, the previous threat model cannot model real attack scenarios.
In contrast, our method is more precise and fine-grained. 
We quantify the amount of leaked information as the cardinality of the set of 
possible inputs based on attackers' observations. 

Before an attack, an adversary has a big but finite input space.
Every time when the adversary observes a leakage site, he can eliminate some 
potential inputs and reduce the size of the input space. 
The smaller the input space is, the more information is actually gained. 
In an extreme case, if the size of the input space reduces to one, 
the adversary can determine the input information uniquely, which means all the secret information
(e.g., the whole secret key) is leaked. By counting the number of distinct inputs, 
we can quantify the information leakage more precisely. 

We use constraints to model the relation between the original sensitive input and
each leakage site. We run the instruction level symbolic execution on the whole
execution trace to generate the constraints. Symbolic execution can provide the fine-grained
information but is usually believed to be an expensive operation in terms of performance. 
Therefore, existing dynamic symbolic execution based works~\cite{203878,236338,Brotzman19Casym} 
either only analyze 
small programs or apply some domain knowledge to simplify the execution. We systematically
analyze the bottleneck of the symbolic execution and optimize it scalable to
real-world crypto systems. 

We apply the above technique and build a tool called \tool{}, 
  %\footnote{CleverHans is a horse that can ``count''.
  %Our tool uses an advanced method to count the number of leaked bits from side channels.}
which can discover potential information leakage sites 
as well as estimating how many bits they can leak for each leakage site. 
We assume that adversaries can exploit secret-dependent control-flow transfers and 
data-access patterns when the program processes different sensitive data. 
%We refer them as the potential information leakage sites. 
First, we collect the dynamic execution trace for each input of the target libraries 
and then run symbolic execution on the traces. 
In this way, we model each side-channel leakage as a math formula. 
The sensitive input is divided into several independent bytes and each byte is regarded as 
a unique symbol. Those formulas can precisely model side-channel vulnerabilities.
Then we extend the problem to multiple leakages and related leakages
and introduce a monte carlo sampling method to estimate the single and combined information leakage.
In fact, if an application has a different sensitive input but still satisfies the formula, 
the code can still leak the same information. 


%Based on the fixed attack target, we classify the software-based side-channel 
%vulnerabilities into two categories: 1.\textit{secret-dependent control-flow transfers} 
%and 2.\textit{secret-dependent data accesses} and model them with math formulas which
%constrain the value of sensitive information.
%We quantify the amount of leaked information as the number of possible solutions that are
%reduced after applying each constrains.


%Our method can identify and quantify address-based
%sensitive information leakage sites in real-world applications automatically. 
%Adversaries can exploit different control-flow transfers and data-access patterns when 
%the program processes different sensitive data. We refer them as the potential information
%leakage sites. Our tool can discover and estimate those potential information leakage sites 
%as well as how many bits they can leak. We are also able to report precisely how many bits
%can be leaked in total if an attacker observes more than one site.
%We run symbolic execution on execution traces. We model each side-channel leakage as a math formula. 
%The sensitive input is divided into several independent bytes and each byte is regarded as 
%a unique symbol. Those formulas can precisely model every the side-channel vulnerability. 
%In other words, if the application has a different sensitive input but still satisfies the formula, 
%the code can still leak the same information.  
%Those information leakage sites may spread in the whole program 
%and their leakages may not be dependent. Simply adding them up can only get a coarse upper bound 
%estimate. In order to accurately calculate the total information leakage, we must know the 
%dependent relationships among those multiple leakages sites. Therefore, we introduce a 
%monte carlo sampling method to estimate the total information leakage.

We apply \tool{} on both symmetric and asymmetric ciphers from
real-world crypto libraries including OpenSSL and mbed TLS\@. The
experimental result confirms that \tool{} can precisely identify the
previous known vulnerabilities, report how much information is leaked
and which byte in the original sensitive buffer is leaked.  Although
some of the analyzed crypto libraries have a number of side-channels,
they actually leak very little information. Also, we perform the
analysis of widely deployed software countermeasures against side
channels.  \tool\ also discovers new vulnerabilities. With the help of
\tool{}, we confirm those vulnerabilities can be exploited.

In summary, we make the following contributions:

\begin{itemize}
  \item We propose a novel method that can quantify fine-grained
    leaked information from side-channel vulnerabilities to match real
    attack scenarios.  Our method is different from previous ones in
    that we model real attack scenarios more precisely while the
    previous research only models the ``average'' or ``random'' case.
    Our results are surprisingly different,
    % compared to previous results and
    much more useful in practice.
    %%   We model each side-channel vulnerabilities as math formulas
    %% and mutiple side-channel vulnerabilities can be seen as the
    %% conjunction of those formulas, which precisely models the
    %% program semantics.
        
  \item We transfer the information quantification problem into a
    counting problem and use the Monte Carlo sampling
    method to estimate the information leakage. Some initial results
    indicate the the sampling method suffers from the curse of
    dimensionality problem. We therefore design a guided sampling
    method and provide the corresponding error estimate.
        
  \item We implement the proposed method into a practical tool and
    apply it on several real-world software. \tool{} successfully
    identifies memory-related side-channel vulnerabilities and
    provides the corresponding information leakage. The information
    leakage result provides the detailed information that can help
    developers to fix the reported vulnerabilities.
\end{itemize}
