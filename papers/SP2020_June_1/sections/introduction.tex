\section{Introduction}
%% side channels are important
Side channels are inevitable in modern computer systems as the sensitive information 
may be leaked by many kinds of inadvertent behaviors, 
such as power, electromagnetic radiation and even sound~\cite{}. 
Among them, software-based side-channel attacks are especially common 
and have been studied for years~\cite{}. 
Those vulnerabilities result from vulnerable software and shared hardware components.
By observing the outputs or hardware behaviors, attackers can
infer the program execution flow that manipulate secrets and 
guess the secrets such as encryption keys~\cite{}.

%% to deal with side channels, we can protect or detect them and detection is better
Various countermeasures have been proposed to defend against 
software-based side-channel attacks. In term of hardware level,
those methods includes reducing shared resources, adopting oblivious RAM and 
transnational memory~\cite{182946,203878,217537}. 
However, in addition to runtime overheads, those methods are limited to specific hardwares. 
Therefore, a more promising and universal direction are software countermeasures, which
detect and eliminate secret-dependent control-flow transfers and data accesses.

%% There are some detection methods but need a good assess method 
In general, recent work~\cite{203878} adopts both static and dynamic analysis
to detect side-channels.
They can find a list of potential leak sites in real-world software, 
but fail to report how severe a potential leakage could be. 
Many of the reported vulnerabilities are typically hard to exploit
and leak very little information (e.g., one bit of the length of the key~\cite{203878}).  
For example, many vulnerabilities in OpenSSL library are assigned with a low severity 
and not considered 
in the current threat model~\cite{https://www.openssl.org/policies/secpolicy.html}. 
It would be valuable to have a tool to tell how much information is actually leaked 
at one vulnerability.

To assess side-channel vulnerabilities, we need a proper quantification metric.
Tools based on static analysis with abstract interpretation can provide an upper bound, 
which is useful to justify whether the implementation is secure enough. 
However, they cannot indicate the leakage is severe due to the over-approximation~\cite{}. 
The ``average'' information leakage assumes that the target program will have 
\textbf{different} sensitive 
input when an attacker launches the attack.
But for real-world attacks, an adversary may run the target problem again and over again 
with \textbf{fixed} unknown sensitive information such as the input.
The dynamic methods take another way with a concrete input and run the program in real
environment. 
Although they are very precise in term of true leakages, no existing tool can
assess the vulnerabilities they discover.
\fixme{why DATA example is commented out below.}
%For example, DATA~\cite{217537} reports more than 2000 potential leakage sites for the RSA implementation of OpenSSL.
%But most of them were dismissed by the author after some manual inspections.

Additionally, open source libraries may have multiple leakage sites, which can be exploited for attackers
at once~\cite{191010,7163052,hornby2011side}. 
The attacker may retrieve part of the sensitive information from one site, part of the sensitive
information from another site and combine them.
It is necessary to know how much information is actually leaked in total, 
but no existing tools can practically estimate multiple leakage sites in real-world libraries.
Simply adding the results for each leakage sites together gets a very rough upper bound of 
the total information leakage if those 
leakages are not independent. 

To overcome the above limitations, we propose a novel method
to quantify information leakage more precisely. 
%Different from previous works, which only consider the
%``average'' information leakage, we study the problem from real attack scenario.
%The average information assumes that the target program will have \textbf{variable} sensitive 
%information when the attacker launches the attack.
%However, for real-world attacks, an adversary may run the target problem again and over again 
%with \textbf{fixed} unknown sensitive information such as the key. 
%Therefore, the previous threat model can't catch real attack scenarios.
%In contrast, our method is more precise and fine-grained. 
%For our analysis, the input key is fixed. 
In general, we quantify the leaked information as the uncertain input space based on attackers' observation.
%It is interesting to notice that the definition is different from that in previous static analysis tools. 
Before an attack, an adversary has a big but finite input space.
Every time the adversary observes one leakage site, he can eliminate some potential inputs and
reduce the size of the input space. 
%The smaller the input space is, the more information is actually gained. 
In extreme cases, if the size of the input space reduces to one, 
the adversary can determine the input information uniquely, which means all information
(e.g., the whole secret key) is leaked. 
Further, we turn the quantification into a model counting problem~\cite{bibid} 
and solve it by Markov Chain Monte Carlo approximate counting~\cite{bibid}. 
In this way, we quantify the information leakage more precisely.


More specifically, we build a tool called \tool{},\footnote{Clever Hans is a horse that can ``count''.
Our tool uses an advanced method to count the number of leaked bits through side channel.}
which could discover and estimate those potential information leakage sites 
as well as how many bits they can leak. 
We assume that adversaries can exploit different control-flow transfers and data-access patterns when 
the program processes different sensitive data. 
%We refer them as the potential information leakage sites. 
First, we collect the dynamic execution trace for each input of the target libraries 
and then run symbolic execution on the traces. 
In this way, we model each side-channel leakage as a math formula. 
The sensitive input is divided into several independent bytes and each byte is regarded as 
a unique symbol. Those formulas can precisely model every the side-channel vulnerability.
Then we extend the problem to multiple leakages and related leakages
and introduce a monte carlo sampling method to estimate the total information leakage.
In fact, if the application has a different sensitive input but still satisfies the formula, 
the code can still leak the same information. 


%Based on the fixed attack target, we classify the software-based side-channel 
%vulnerabilities into two categories: 1.\textit{secret-dependent control-flow transfers} 
%and 2.\textit{secret-dependent data accesses} and model them with math formulas which
%constrain the value of sensitive information.
%We quantify the amount of leaked information as the number of possible solutions that are
%reduced after applying each constrains.


%Our method can identify and quantify address-based
%sensitive information leakage sites in real-world applications automatically. 
%Adversaries can exploit different control-flow transfers and data-access patterns when 
%the program processes different sensitive data. We refer them as the potential information
%leakage sites. Our tool can discover and estimate those potential information leakage sites 
%as well as how many bits they can leak. We are also able to report precisely how many bits
%can be leaked in total if an attacker observes more than one site.
%We run symbolic execution on execution traces. We model each side-channel leakage as a math formula. 
%The sensitive input is divided into several independent bytes and each byte is regarded as 
%a unique symbol. Those formulas can precisely model every the side-channel vulnerability. 
%In other words, if the application has a different sensitive input but still satisfies the formula, 
%the code can still leak the same information.  
%Those information leakage sites may spread in the whole program 
%and their leakages may not be dependent. Simply adding them up can only get a coarse upper bound 
%estimate. In order to accurately calculate the total information leakage, we must know the 
%dependent relationships among those multiple leakages sites. Therefore, we introduce a 
%monte carlo sampling method to estimate the total information leakage.

We apply \tool{} on several crypto and non-crypto libraries including OpenSSL,
MbedTLS and libjpeg. The experiment result confirms that \tool{} can precisely identify the pre-known vulnerabilities,
reports how much information is leaked and which byte in the original sensitive buffer is leaked. 
Although some of the analyzed crypto libraries have a number of side-channels, they actually
leak very little information. Also, we also use the tool to analysis the two reported side-channel attack 
in the libjpeg library. Finally, we present new vulnerabilities. With the help of \tool{}, we confirm those
vulnerabilities are easily to be exploited. Our results are superisingly different compared to previous results
and much more useful in practice.

In summary, we make the following contributions:

\begin{itemize}
	\item We propose a novel method that can quantify fine-grained side-channel
        information leakages. We model each information leakage vulnerability as math formulas and 
        mutiple side-channel vulnerabilities can be seen as the disjunctions of those formulas, which
        precisely models the program semantics.
        \item We transfer the information quantification problem into a probabilty distribution problem and 
        use the Monte Carlo sampling method to estimate the information leakage. Some initial results indicate the 
        the sampling method suffers from the curse of dimensionality problem. We therefore design a guided
        sampling method and provide the corresponding error esitimate.
	\item We implement the proposed method into a practical tool and apply it on several real-world software. \tool{} 
        successfully identify the address-based side-channel vulnerabilities and provide the corresponding
        information leakge. The information lekage result provide the detailed information that help developers
        to fix the vulnerability.
\end{itemize}



