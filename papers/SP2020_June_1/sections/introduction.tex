\section{Introduction}

Side channels are inevitable in computer systems. The attacker can infer the sensitive information by observing execution behaviours. The software-based attackers(e.g., Cache-based side channels, controlled side attacks) are especially common because these attacks typically don’t need any physical access. We observe that the root cause is that the implementation of the program may have different memory access patterns during the execution. Attackers can observe the pattern and infer the sensitive information based on source code.

Various countermeasures have been proposed to defend the address-based side-channel attacks. The basic idea is to identify and eliminate secret-dependent memory access patterns. It is very tedious and hard to manually find every the leakage site. To address the problem, a lot of methods have been proposed to automatically detect information leakage. While those tools can typically report a list of potential leakages sites, they still face a couple of limitations.

First, the existing tools can only report a list of potential leakage sites. But they fail to report how severe each potential leakage site could be. Most of leakages found by those tools are typically hard to exploit or leak too little information.  It is worthwhile to have a tool to tell how much information is actually leaks. For example, many secret dependent memory access patterns in OpenSSL have been known and studied for years. But the authors still doesn’t fix them all because they think they aren’t not severe. The tool based on static analysis with abstract interpretation, can give leakage leakage upper bound, which is useful to justify the implementation is secure enough. But they can’t indicate the leakage site is severe enough due to over approximation. The dynamic method will take a concrete input and run the actual program. Those tools are typically very precise in term of true leakage. But none of those tools can actually report how many bits are actually leaked. For example, DATA reports more than 2000 potential leakage sites for the RSA implementation of OpenSSL. But most of them were dismissed by the author after some manual inspections.

Second, many open source libraries may have multiple information leakage sites. A side-channel attacker usually exploits multiple leakage sites at one time. The attacker may retrieve some information from one site and some other information from another site. It is hard to model how much information is actually leaked in totoel. Adding those leakage simply can get only get an upper bound estimate of total information leakage if those leakages aren’t independent. No existing tools can practically estimate the total information leakage from multiple leakage sites in open source libraries.

In the paper, we presented a tool that can automatically identify sensitive information leakage sites in real-world applications. Compared to existing tools, our tool can provide a list of information leakage sites as well as how many bits they can leak if the attacker can observe one site, two sites or multiple sites. We defined the amount of leaked information as how much uncertainty about the sensitive information the attacker could reduce after observing the side-channel. It is interesting to mention that the definition here is different than the definition in several static analysis tools. We will explain the reason in the following sections. After that, we run the symbolic execution on the execution trace. We model each side-channel leakage as several constraints. The only symbols in those constraints are sensitive input. Those constraints can unique model the side-channels. In other words, if the application has an different sensitive input, the code will still leak the same information about the input.  Those information leakage sites may spread the whole program and their leakages aren’t dependent. Simply adding them up can only give the upper bound estimate. In order to accurately calculate the total information leakage, we must know the dependent relationships among those multiple leakages sites. Therefore, we introduce a monte carlo sampling method to estimate the total information leakage from the constraints generated by previous symbolic execution. 

We implemented the proposed method as a tool and applied it on a series of open source libraries. First, we found that most leakages reported in the previous research actually leaked very little information. Second, we also discovered several sensitive vulnerabilities in the media libraries. 
