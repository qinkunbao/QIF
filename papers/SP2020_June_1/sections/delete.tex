\subsection{Challenge II: How to Combine the Leakage Information from Multiple Leak Sites}
Real-world software can have various side-channel vulnerabilities. Those vulnerabilities 
may spread in the whole program. An adversary may exploit more than one side-channel vulnerabilities 
to gain more information~\cite{7163052, 191010}. In order to precisely quantify the
total information leakage, we need to know the relation of those leakage sites. 


\lstinputlisting[language=c, 
                 numbers=left,
                 numbersep=5pt,                   % how far the line-numbers are from the code
                 caption={Multiple leakages},
                 frame = single,
                 captionpos=b,
                 label={code::multiple},
                 basicstyle=\fontsize{7}{9}\selectfont\ttfamily]
                 {sample_code/motivation_multiple.c}

                

Consider the running example in ~\ref{code::multiple}, in which $k1$, $k2$ and $k3$ are
the sensitive key. The code has six different leakage. Leakage 1, 2, 3 are the secret-dependent
data accesses and leakage 4, 5, 6 are the secret-dependent control-flow transfers.  
The attacker can infer the last three digits of
$k1$, $k2$, $k3$ from leakage 1, 2, 3. So those leakages are independent. For leakage 1, 4, 6, however,
we have no idea about the total information leakage.


Suppose one program has two side-channel vulnerabilities A and B, which can leak $L_A$ and $L_B$ bits respectively
according to the definition~\ref{def}. 
Depending on the relation between A and B, the total leaked information $L_{\mathit{total}}$ will be:

\subsubsection{Independent Leakages}
If A and B are independent leakages, the total information leakage will be:
$$L_{\mathit{total}} = L_A + L_B $$

\subsubsection{Dependent Leakages}
If A and B are dependent leakages, the total information leakage will be:
$$\max{\{L_A, L_B\}}  \leq L_{\mathit{total}} < L_A + L_B$$

\subsubsection{Mutual Exclusive Leakages}
If A and B are mutual exclusive leakages, then only A or B can be observed for one fixed input.
$$L_{\mathit{total}} = 
\begin{cases}
L_A, & \text{only} ~ A \\
L_B, & \text{only} ~ B
\end{cases}$$

According to above definition, leakage 1, 2, 3 are independent leakages. Leakage 4, 5
are mutual exclusive leakages. 
For real-world applications, it is hard to estimate the total leaked information for the following reasons.
First, the real-world applications have more than thousands of lines of code. One leakage site leaks the temporary value. 
But the value contains some information about the original buffer. It is hard to know how the 
the sensitive value affects the temporary value. Second, some leakages sites may be
dependent. The occurrence of the first affects the occurrence of the second sites. We 
can't simply add them up. Third, leakage sites are in the different blocks of the 
control-flow graph, which means that only one of the two leakages site may be executed
during the execution.

\vspace*{6pt}
\textbf{Our Solution to Challenge II:}
Given a program $P$ with $k$ as the sensitive input, 
we use $k_i$ to denote the sensitive information, where $i$ is the index of the byte in the original buffer.  
We can represent each temporal
values with a formula. There are two types of values in the formula: the concrete value and
the symbolic value. We use the runtime information to simplify the formula. In other words,
we only use symbolic values to represent the sensitive input. For other values that are
independent from the sensitive input, we use the concrete value from the runtime information. 

After that, we model each leakage sites as a math formulas.
The attacker can retrieve the sensitive information by observing the different patterns in 
control-flows and data access when the program process different sensitive information. 
We refer them as the secret-dependent control flow and secret-dependent data access accordingly.
For secret-dependent control transfers, we model the leakage using the path conditions that cause the control
transfer. For secret-dependent memory accesses, we use a symbolic formula $F(\vec{K})$ to
represent the memory address and check if different sensitive inputs can lead to different
memory accesses. As long as we model each leakage with a formula. We can regard multiple leakges as the conjunction of
those formulas. 

%% the algorithm

