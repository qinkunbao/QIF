
\section{Algorithm to Compute the Maximum Independent Partition}
\label{appendix:partition}

{\small
\IncMargin{1em}
\begin{algorithm}[h]
    \DontPrintSemicolon
    \SetKwInOut{Input}{input}\SetKwInOut{Output}{output}
    \Input{$c_t(\addr{1},\addr{2},\ldots,\addr{n}) = c_{\addr{1}} \land c_{\addr{2}} \land \ldots \land c_{\addr{m}}$}
    \Output{The Maximum Independent Partition of $G = \{g_{1}, g_{2}  , \ldots,  g_{m} \}$ }
    \For{$i\leftarrow 1$ \KwTo $n$}
    {
        $S_{c_{\addr{i}}}$ $\leftarrow$ $\pi(c_{\addr{i}})$ \;
        \For{$g_{i} \in G$}
        {
            $S_{g_j}$ $\leftarrow$ $\pi(g_{j})$ \;
            $S$ $\leftarrow$ $S_{c_{\addr{i}}} \cap S_{g_j}$  \;
            \If{$S \neq \emptyset$}
            {
                $g_{j} \leftarrow g_{i} \land g_{\addr{i}}$ \;
                \textbf{break} \;
            }
            Insert $c_{\addr{i}}$ to $G$
        }
    }
    \caption{The Maximum Independent Partition}
    \label{algo:max-inde}
\end{algorithm}
\DecMargin{1em}
}

\newpage
\section{Algorithm to Compute the Number of Satisfying Assignments}
\label{appendix:montecarlo}
~
{\small
\IncMargin{1em}
\begin{algorithm}
    \SetAlgoLined
    \DontPrintSemicolon

    \KwIn{{The constraint $g_{i}= c_{i_1} \land c_{i_2}
                    \land \ldots \land c_{i_m}$}}
    \KwOut{{The number of assignments that satisfy $g_{i}$ $|K_{g_{i}}|$}}

    $n$: the number of sampling times \;
    $S_{c_i}$: the set contains input variables for $c_{i}$ \;
    $n_{s}$: the number of satisfying assignments \;
    $N_{c_t}$: the set contains all solution for $c_t$ \;
    $r$: times of reducing $g$\;
    $k$: the input variable \;
    $R$: a function that produces a random point from $S_{c_i}$\;
    %$\#k$: the satisfying number of k \fixme{this number is not used syntactically} \;
    %Initialization: \;
    $r$ $\leftarrow$ $1$,
    $n$ $\leftarrow$ $0$ \;
    \For{$t$ $\leftarrow$ $1$ \KwTo $m$} {
        $S_{c_t}$ $\leftarrow$ $\pi(c_t)$ \;
        \If{$|S_{c_t}| = 1$}
        {
            $N_{c_t}$ $\leftarrow$ Compute all solutions of $c_i$ \;
            $N_{c_t} = \{n_1, \ldots, n_m\},\ S_{c_t} = \{k\}  $ \;
            $g_{i} = $ $g_i(k=n_1) \land \ldots \land g_i(k=n_m)$ \;
            $r \leftarrow r+1$ \;

        }
    }
    \While{$n \leq \frac{6p}{1-p}$} {
        $S_{g_i}$ $\leftarrow$ $\pi(g_i)$ \;
        $v \leftarrow R(S_{g_i})$ 
        \If{$v$ satisfies $g_i$}
        {
           $n_s \leftarrow n_s + 1$
        }
        $n \leftarrow n +1,\ p = \frac{n_s}{n}$
    }

    $|K_{g_{i}}|$ $\leftarrow$ $n_s|K| / (n * r * range(k))$
    \caption{Multiple Step Monte Carlo Sampling}
\end{algorithm}
\DecMargin{1em}
}


%\clearpage
\section{SBOX Example of Bit-slicing}
\label{appendix:SBOX}
~
\begin{figure}[h!]
    \centering
    \begin{lstlisting}[xleftmargin=.02\textwidth,xrightmargin=.01\textwidth]
uint8_t password = input();

a = *password & 0b001;
b = (*password & 0b010) >> 1;
c = (*password & 0b100) >> 2;

na = ~a & 1;
nb = ~b & 1;
nc = ~c & 1;

t0 = (b & nc);
t1 = (b | nc);

l = (a & nb) | t0;
r = (na & t1) | t0;

ret = l << 1 + r;
      \end{lstlisting}
    \caption{SBOX with BitSlicing}
    \label{fig:SBOX_bitslicing}
\end{figure}

~

\begin{figure}[h!]
    \centering
    \begin{lstlisting}[xleftmargin=.02\textwidth,xrightmargin=.01\textwidth]
uint8_t password = input();

uint8_t SBOX[] = {1, 0, 3, 1, 2, 2, 3, 0};

if (password <= 0b111)      \\Leaks 5 bits of password
    ret = SBOX[password];   \\Leaks 4 bits of password
      \end{lstlisting}
    \caption{SBOX without BitSlicing}
    \label{fig:SBOX_da}
\end{figure}

\newpage

\section{Example for Lookup Table}
~
\begin{figure}[h!]
    \centering
    \begin{lstlisting}[xleftmargin=.02\textwidth,xrightmargin=.01\textwidth]
static const uint8_t T[16] = {
      0x63U, 0x7cU, 0x77U, 0x7bU, 0xf2U, 0x6bU, 0x6fU, 0xc5U,
      0x30U, 0x01U, 0x67U, 0x2bU, 0xfeU, 0xd7U, 0xabU, 0x76U,
};
      \end{lstlisting}
    \caption{One Byte-wide Lookup Table}
    \label{fig:one_byte_table}
\end{figure}
~
\begin{figure}[h!]
    \centering
    \begin{lstlisting}[xleftmargin=.02\textwidth,xrightmargin=.01\textwidth]
static const uint32_t T[16] = {
    0xc66363a5U, 0xf87c7c84U, 0xee777799U, 0xf67b7b8dU,
    0xfff2f20dU, 0xd66b6bbdU, 0xde6f6fb1U, 0x91c5c554U,
    0x60303050U, 0x02010103U, 0xce6767a9U, 0x562b2b7dU,
    0xe7fefe19U, 0xb5d7d762U, 0x4dababe6U, 0xec76769aU,
};
    \end{lstlisting}
    \caption{Four Byte-wide Lookup Table}
    \label{fig:four_byte_table}
\end{figure}

~

\begin{figure}[h!]
    \centering
    \begin{lstlisting}[xleftmargin=.02\textwidth,xrightmargin=.01\textwidth]
void encrypt_one(uint32_t *o, uint32_t *key, uint32_t l)
{
    for (int i = 0; i < l; i+=4)
        output[i] = (T[(key[i]>>24)] << 24) ^
                    (T[(key[i+1]>>16) & 0xff] << 16) ^
                    (T[(key[i+2]>>8) & 0xff] << 8) ^
                    (T[(key[i+3]) & 0xff]);
}
    \end{lstlisting}
    \caption{Encrypt Function with one Byte-wide Table Lookup}
    \label{fig:one_byte_table_lookup}
\end{figure}

~

\begin{figure}[h!]
    \centering
    \begin{lstlisting}[xleftmargin=.02\textwidth,xrightmargin=.01\textwidth]
void encrypt_four(uint32_t *o, uint32_t *key, uint32_t l)
{
    for (int i = 0; i < l; i+=4)
        output[i] = (T[(key[i]>>24)] & 0xff000000) ^
                    (T[(key[i+1]>>16) & 0xff] & 0x00ff0000) ^
                    (T[(key[i+2]>>8) & 0xff] & 0x0000ff00) ^
                    (T[(key[i+3]) & 0xff] & 0x000000ff);
}
    \end{lstlisting}
    \caption{Encrypt Function with one Byte-wide Table Lookup}
    \label{fig:four_byte_table_lookup}
\end{figure}



% \newpage
\section{Detailed Experimental Results}
\label{sec:result-table}

Here we present the detailed experimental results.
Due to space limitation, we select the representative implementations of
AES, DES, and RSA in
mbed TLS 2.5,
OpenSSL 1.1.0f,  and
OpenSSL 1.1.1.  
%For RSA, we also include OpenSSL 1.0.2f and OpenSSL 1.0.2k.
The results are representative to other versions.
All the results will be made available in electronic format online
when the paper is published. %at \fixme{http://tinyurl}.

In all the tables presented in this appendix, the mark ``$*$'' means timeout,
which indicates more severe leakages. See \S\ref{loc:timeout} for the details.
Also note that we round the calculated numbers of leaked bits to include one digit
after the decimal point, so $0.0$ really means very small amount of leakage, but not exactly zero. See \S\ref{sssec:errest} for the details of error estimate.

  %as an integer.
  %Hence the `0' leakage does not equals to no leakage but some leakage size
 % between 0 and 0.5 bit, see \S\ref{sssec:errest} for the details.

\begin{table*}%[h]
\centering
\caption{Summary of all vulnerabilities in AES implemented by mbedTLS 2.5 with the amount of leak informationThe mark $*$ means timeout,which indicates more severe leakages (see \S
ef{loc:timeout}).}\label{tab:AESmbedTLS}
%\resizebox{\columnwidth}{!}{\begin{tabular}{clrrr}
\hline
\textbf{File} & \textbf{Line Num} & \textbf{Function} & \textbf{Leakedbits} & \textbf{Type} \\\hline
aes.c&536&mbedtls\_aes\_setkey\_enc&6 &DA\\
aes.c&536&mbedtls\_aes\_setkey\_enc&8 &DA\\
aes.c&536&mbedtls\_aes\_setkey\_enc&7 &DA\\
aes.c&536&mbedtls\_aes\_setkey\_enc&7 &DA\\
aes.c&729&mbedtls\_internal\_aes\_encrypt&3 &DA\\
aes.c&729&mbedtls\_internal\_aes\_encrypt&7 &DA\\
aes.c&729&mbedtls\_internal\_aes\_encrypt&3 &DA\\
aes.c&729&mbedtls\_internal\_aes\_encrypt&8 &DA\\
aes.c&729&mbedtls\_internal\_aes\_encrypt&3 &DA\\
aes.c&729&mbedtls\_internal\_aes\_encrypt&4 &DA\\
aes.c&729&mbedtls\_internal\_aes\_encrypt&7 &DA\\
aes.c&729&mbedtls\_internal\_aes\_encrypt&7 &DA\\
aes.c&729&mbedtls\_internal\_aes\_encrypt&8 &DA\\
aes.c&729&mbedtls\_internal\_aes\_encrypt&4 &DA\\
aes.c&729&mbedtls\_internal\_aes\_encrypt&8 &DA\\
aes.c&729&mbedtls\_internal\_aes\_encrypt&8 &DA\\
aes.c&729&mbedtls\_internal\_aes\_encrypt&7 &DA\\
aes.c&729&mbedtls\_internal\_aes\_encrypt&3 &DA\\
aes.c&729&mbedtls\_internal\_aes\_encrypt&3 &DA\\
aes.c&729&mbedtls\_internal\_aes\_encrypt&3 &DA\\
aes.c&730&mbedtls\_internal\_aes\_encrypt&3 &DA\\
aes.c&730&mbedtls\_internal\_aes\_encrypt&7 &DA\\
aes.c&730&mbedtls\_internal\_aes\_encrypt&3 &DA\\
aes.c&730&mbedtls\_internal\_aes\_encrypt&8 &DA\\
aes.c&730&mbedtls\_internal\_aes\_encrypt&3 &DA\\
aes.c&730&mbedtls\_internal\_aes\_encrypt&3 &DA\\
aes.c&730&mbedtls\_internal\_aes\_encrypt&8 &DA\\
aes.c&730&mbedtls\_internal\_aes\_encrypt&8 &DA\\
aes.c&730&mbedtls\_internal\_aes\_encrypt&7 &DA\\
aes.c&730&mbedtls\_internal\_aes\_encrypt&8 &DA\\
aes.c&730&mbedtls\_internal\_aes\_encrypt&7 &DA\\
aes.c&730&mbedtls\_internal\_aes\_encrypt&3 &DA\\
aes.c&730&mbedtls\_internal\_aes\_encrypt&8 &DA\\
aes.c&730&mbedtls\_internal\_aes\_encrypt&3 &DA\\
aes.c&730&mbedtls\_internal\_aes\_encrypt&4 &DA\\
aes.c&730&mbedtls\_internal\_aes\_encrypt&4 &DA\\
aes.c&733&mbedtls\_internal\_aes\_encrypt&4 &DA\\
aes.c&733&mbedtls\_internal\_aes\_encrypt&4 &DA\\
aes.c&733&mbedtls\_internal\_aes\_encrypt&4 &DA\\
aes.c&733&mbedtls\_internal\_aes\_encrypt&3 &DA\\
aes.c&733&mbedtls\_internal\_aes\_encrypt&3 &DA\\
aes.c&733&mbedtls\_internal\_aes\_encrypt&3 &DA\\
aes.c&733&mbedtls\_internal\_aes\_encrypt&3 &DA\\
aes.c&733&mbedtls\_internal\_aes\_encrypt&4 &DA\\
aes.c&733&mbedtls\_internal\_aes\_encrypt&3 &DA\\
aes.c&733&mbedtls\_internal\_aes\_encrypt&4 &DA\\
aes.c&733&mbedtls\_internal\_aes\_encrypt&4 &DA\\
aes.c&733&mbedtls\_internal\_aes\_encrypt&4 &DA\\
aes.c&733&mbedtls\_internal\_aes\_encrypt&3 &DA\\
aes.c&733&mbedtls\_internal\_aes\_encrypt&4 &DA\\
aes.c&733&mbedtls\_internal\_aes\_encrypt&4 &DA\\
aes.c&733&mbedtls\_internal\_aes\_encrypt&3 &DA\\
aes.c&735&mbedtls\_internal\_aes\_encrypt&2 &DA\\
aes.c&735&mbedtls\_internal\_aes\_encrypt&2 &DA\\
aes.c&735&mbedtls\_internal\_aes\_encrypt&2 &DA\\
aes.c&735&mbedtls\_internal\_aes\_encrypt&1 &DA\\
aes.c&741&mbedtls\_internal\_aes\_encrypt&1 &DA\\
aes.c&741&mbedtls\_internal\_aes\_encrypt&1 &DA\\
aes.c&747&mbedtls\_internal\_aes\_encrypt&2 &DA\\
aes.c&741&mbedtls\_internal\_aes\_encrypt&2 &DA\\
aes.c&753&mbedtls\_internal\_aes\_encrypt&2 &DA\\
aes.c&741&mbedtls\_internal\_aes\_encrypt&1 &DA\\
aes.c&747&mbedtls\_internal\_aes\_encrypt&1 &DA\\
aes.c&747&mbedtls\_internal\_aes\_encrypt&1 &DA\\
aes.c&753&mbedtls\_internal\_aes\_encrypt&2 &DA\\
aes.c&747&mbedtls\_internal\_aes\_encrypt&2 &DA\\
aes.c&753&mbedtls\_internal\_aes\_encrypt&2 &DA\\
aes.c&753&mbedtls\_internal\_aes\_encrypt&1 &DA\\
\hline
\end{tabular}
%}
\end{table*}
%% \begin{table*}%[h]
\centering
\caption{Summary of all vulnerabilities in AES implemented by mbedTLS 2.15.1 with the amount of leak informationThe mark $*$ means timeout,which indicates more severe leakages (see \S\ref{loc:timeout}).}\label{tab:AESmbedTLS}
%\resizebox{\columnwidth}{!}{
\begin{tabular}{clrrr}
\hline
\textbf{File} & \textbf{Line Num} & \textbf{Function} & \textbf{Leakedbits} & \textbf{Type} \\\hline
aes.c&595&mbedtls\_aes\_setkey\_enc&8 &DA\\
aes.c&595&mbedtls\_aes\_setkey\_enc&8 &DA\\
aes.c&595&mbedtls\_aes\_setkey\_enc&7 &DA\\
aes.c&595&mbedtls\_aes\_setkey\_enc&8 &DA\\
aes.c&860&mbedtls\_internal\_aes\_encrypt&3 &DA\\
aes.c&860&mbedtls\_internal\_aes\_encrypt&8 &DA\\
aes.c&860&mbedtls\_internal\_aes\_encrypt&4 &DA\\
aes.c&860&mbedtls\_internal\_aes\_encrypt&8 &DA\\
aes.c&860&mbedtls\_internal\_aes\_encrypt&3 &DA\\
aes.c&860&mbedtls\_internal\_aes\_encrypt&4 &DA\\
aes.c&860&mbedtls\_internal\_aes\_encrypt&7 &DA\\
aes.c&860&mbedtls\_internal\_aes\_encrypt&8 &DA\\
aes.c&860&mbedtls\_internal\_aes\_encrypt&8 &DA\\
aes.c&860&mbedtls\_internal\_aes\_encrypt&3 &DA\\
aes.c&860&mbedtls\_internal\_aes\_encrypt&7 &DA\\
aes.c&860&mbedtls\_internal\_aes\_encrypt&7 &DA\\
aes.c&860&mbedtls\_internal\_aes\_encrypt&8 &DA\\
aes.c&860&mbedtls\_internal\_aes\_encrypt&3 &DA\\
aes.c&860&mbedtls\_internal\_aes\_encrypt&4 &DA\\
aes.c&860&mbedtls\_internal\_aes\_encrypt&4 &DA\\
aes.c&861&mbedtls\_internal\_aes\_encrypt&4 &DA\\
aes.c&861&mbedtls\_internal\_aes\_encrypt&8 &DA\\
aes.c&861&mbedtls\_internal\_aes\_encrypt&4 &DA\\
aes.c&861&mbedtls\_internal\_aes\_encrypt&8 &DA\\
aes.c&861&mbedtls\_internal\_aes\_encrypt&3 &DA\\
aes.c&861&mbedtls\_internal\_aes\_encrypt&3 &DA\\
aes.c&861&mbedtls\_internal\_aes\_encrypt&7 &DA\\
aes.c&861&mbedtls\_internal\_aes\_encrypt&7 &DA\\
aes.c&861&mbedtls\_internal\_aes\_encrypt&7 &DA\\
aes.c&861&mbedtls\_internal\_aes\_encrypt&8 &DA\\
aes.c&861&mbedtls\_internal\_aes\_encrypt&8 &DA\\
aes.c&861&mbedtls\_internal\_aes\_encrypt&4 &DA\\
aes.c&861&mbedtls\_internal\_aes\_encrypt&8 &DA\\
aes.c&861&mbedtls\_internal\_aes\_encrypt&3 &DA\\
aes.c&861&mbedtls\_internal\_aes\_encrypt&4 &DA\\
aes.c&861&mbedtls\_internal\_aes\_encrypt&3 &DA\\
aes.c&864&mbedtls\_internal\_aes\_encrypt&3 &DA\\
aes.c&864&mbedtls\_internal\_aes\_encrypt&4 &DA\\
aes.c&864&mbedtls\_internal\_aes\_encrypt&3 &DA\\
aes.c&864&mbedtls\_internal\_aes\_encrypt&3 &DA\\
aes.c&864&mbedtls\_internal\_aes\_encrypt&4 &DA\\
aes.c&864&mbedtls\_internal\_aes\_encrypt&3 &DA\\
aes.c&864&mbedtls\_internal\_aes\_encrypt&4 &DA\\
aes.c&864&mbedtls\_internal\_aes\_encrypt&4 &DA\\
aes.c&864&mbedtls\_internal\_aes\_encrypt&3 &DA\\
aes.c&864&mbedtls\_internal\_aes\_encrypt&4 &DA\\
aes.c&864&mbedtls\_internal\_aes\_encrypt&3 &DA\\
aes.c&864&mbedtls\_internal\_aes\_encrypt&4 &DA\\
aes.c&864&mbedtls\_internal\_aes\_encrypt&3 &DA\\
aes.c&864&mbedtls\_internal\_aes\_encrypt&4 &DA\\
aes.c&864&mbedtls\_internal\_aes\_encrypt&3 &DA\\
aes.c&864&mbedtls\_internal\_aes\_encrypt&4 &DA\\
aes.c&866&mbedtls\_internal\_aes\_encrypt&1 &DA\\
aes.c&866&mbedtls\_internal\_aes\_encrypt&1 &DA\\
aes.c&866&mbedtls\_internal\_aes\_encrypt&1 &DA\\
aes.c&866&mbedtls\_internal\_aes\_encrypt&1 &DA\\
aes.c&872&mbedtls\_internal\_aes\_encrypt&1 &DA\\
aes.c&872&mbedtls\_internal\_aes\_encrypt&1 &DA\\
aes.c&878&mbedtls\_internal\_aes\_encrypt&1 &DA\\
aes.c&872&mbedtls\_internal\_aes\_encrypt&2 &DA\\
aes.c&884&mbedtls\_internal\_aes\_encrypt&2 &DA\\
aes.c&872&mbedtls\_internal\_aes\_encrypt&2 &DA\\
aes.c&878&mbedtls\_internal\_aes\_encrypt&2 &DA\\
aes.c&878&mbedtls\_internal\_aes\_encrypt&1 &DA\\
aes.c&884&mbedtls\_internal\_aes\_encrypt&2 &DA\\
aes.c&878&mbedtls\_internal\_aes\_encrypt&1 &DA\\
aes.c&884&mbedtls\_internal\_aes\_encrypt&2 &DA\\
aes.c&884&mbedtls\_internal\_aes\_encrypt&1 &DA\\
\hline
\end{tabular}
%}
\end{table*}
%% \begin{table}%[h]
\centering\tiny
\caption{Summary of all vulnerabilities in AES implemented by openssl 0.9.7 with the amount of leak information. The mark $*$ means timeout, which indicates more severe leakages (see \S\ref{loc:timeout}).}\label{tab:AESopenssl}
%\resizebox{\columnwidth}{!}{
\begin{tabular}{clrrr}
\hline
\textbf{File} & \textbf{Line Num} & \textbf{Function} & \textbf{Leakedbits} & \textbf{Type} \\\hline
aes\_core.c& 662&AES\_set\_encrypt\_key&5 &DA\\
aes\_core.c& 662&AES\_set\_encrypt\_key&3 &DA\\
aes\_core.c& 662&AES\_set\_encrypt\_key&3 &DA\\
aes\_core.c& 662&AES\_set\_encrypt\_key&4 &DA\\
aes\_core.c& 662&AES\_set\_encrypt\_key&3 &DA\\
aes\_core.c& 662&AES\_set\_encrypt\_key&3 &DA\\
aes\_core.c& 662&AES\_set\_encrypt\_key&4 &DA\\
aes\_core.c& 662&AES\_set\_encrypt\_key&3 &DA\\
aes\_core.c& 662&AES\_set\_encrypt\_key&4 &DA\\
aes\_core.c& 662&AES\_set\_encrypt\_key&3 &DA\\
aes\_core.c& 662&AES\_set\_encrypt\_key&4 &DA\\
aes\_core.c& 662&AES\_set\_encrypt\_key&4 &DA\\
aes\_core.c& 662&AES\_set\_encrypt\_key&4 &DA\\
aes\_core.c& 662&AES\_set\_encrypt\_key&3 &DA\\
aes\_core.c& 662&AES\_set\_encrypt\_key&4 &DA\\
aes\_core.c& 662&AES\_set\_encrypt\_key&4 &DA\\
aes\_core.c& 662&AES\_set\_encrypt\_key&3 &DA\\
aes\_core.c& 662&AES\_set\_encrypt\_key&3 &DA\\
aes\_core.c& 662&AES\_set\_encrypt\_key&3 &DA\\
aes\_core.c& 662&AES\_set\_encrypt\_key&4 &DA\\
aes\_core.c& 662&AES\_set\_encrypt\_key&4 &DA\\
aes\_core.c& 662&AES\_set\_encrypt\_key&3 &DA\\
aes\_core.c& 662&AES\_set\_encrypt\_key&4 &DA\\
aes\_core.c& 662&AES\_set\_encrypt\_key&3 &DA\\
aes\_core.c& 662&AES\_set\_encrypt\_key&3 &DA\\
aes\_core.c& 662&AES\_set\_encrypt\_key&3 &DA\\
aes\_core.c& 662&AES\_set\_encrypt\_key&4 &DA\\
aes\_core.c& 662&AES\_set\_encrypt\_key&3 &DA\\
aes\_core.c& 662&AES\_set\_encrypt\_key&4 &DA\\
aes\_core.c& 662&AES\_set\_encrypt\_key&3 &DA\\
aes\_core.c& 662&AES\_set\_encrypt\_key&4 &DA\\
aes\_core.c& 662&AES\_set\_encrypt\_key&4 &DA\\
aes\_core.c& 880&AES\_encrypt&4 &DA\\
aes\_core.c& 880&AES\_encrypt&10&DA\\
aes\_core.c& 880&AES\_encrypt&5 &DA\\
aes\_core.c& 880&AES\_encrypt&8 &DA\\
aes\_core.c& 886&AES\_encrypt&4 &DA\\
aes\_core.c& 886&AES\_encrypt&4 &DA\\
aes\_core.c& 886&AES\_encrypt&9 &DA\\
aes\_core.c& 886&AES\_encrypt&9 &DA\\
aes\_core.c& 892&AES\_encrypt&10&DA\\
aes\_core.c& 892&AES\_encrypt&5 &DA\\
aes\_core.c& 892&AES\_encrypt&8 &DA\\
aes\_core.c& 892&AES\_encrypt&4 &DA\\
aes\_core.c& 898&AES\_encrypt&9 &DA\\
aes\_core.c& 898&AES\_encrypt&10&DA\\
aes\_core.c& 898&AES\_encrypt&5 &DA\\
aes\_core.c& 898&AES\_encrypt&5 &DA\\
aes\_core.c& 910&AES\_encrypt&4 &DA\\
aes\_core.c& 910&AES\_encrypt&7 &DA\\
aes\_core.c& 910&AES\_encrypt&9 &DA\\
aes\_core.c& 910&AES\_encrypt&8 &DA\\
aes\_core.c& 916&AES\_encrypt&9 &DA\\
aes\_core.c& 916&AES\_encrypt&5 &DA\\
aes\_core.c& 916&AES\_encrypt&8 &DA\\
aes\_core.c& 916&AES\_encrypt&8 &DA\\
aes\_core.c& 922&AES\_encrypt&7 &DA\\
aes\_core.c& 922&AES\_encrypt&8 &DA\\
aes\_core.c& 922&AES\_encrypt&8 &DA\\
aes\_core.c& 922&AES\_encrypt&4 &DA\\
aes\_core.c& 928&AES\_encrypt&9 &DA\\
aes\_core.c& 928&AES\_encrypt&9 &DA\\
aes\_core.c& 928&AES\_encrypt&5 &DA\\
aes\_core.c& 928&AES\_encrypt&7 &DA\\
aes\_core.c& 940&AES\_encrypt&3 &DA\\
aes\_core.c& 940&AES\_encrypt&3 &DA\\
aes\_core.c& 940&AES\_encrypt&4 &DA\\
aes\_core.c& 947&AES\_encrypt&4 &DA\\
aes\_core.c& 947&AES\_encrypt&4 &DA\\
aes\_core.c& 947&AES\_encrypt&3 &DA\\
aes\_core.c& 954&AES\_encrypt&4 &DA\\
aes\_core.c& 954&AES\_encrypt&3 &DA\\
aes\_core.c& 954&AES\_encrypt&3 &DA\\
aes\_core.c& 961&AES\_encrypt&3 &DA\\
aes\_core.c& 961&AES\_encrypt&4 &DA\\
\hline
\end{tabular}
%}
\end{table}
%% \begin{table}%[h]
\centering\tiny\scriptsize
\caption{Summary of all vulnerabilities in AES implemented by openssl 1.0.2f with the amount of leak information. The mark $*$ means timeout, which indicates more severe leakages (see \S\ref{loc:timeout}).}\label{tab:AESopenssl}
%\resizebox{\columnwidth}{!}{
\begin{tabular}{lrlrr}
\hline
\textbf{File} & \textbf{Line No.} & \textbf{Function} & \textbf{Leaked Bits} & \textbf{Type} \\\hline
aes\_core.c& 654&private\_AES\_set\_encrypt\_key&4 &DA\\
aes\_core.c& 654&private\_AES\_set\_encrypt\_key&3 &DA\\
aes\_core.c& 672&private\_AES\_set\_encrypt\_key&4 &DA\\
aes\_core.c& 673&private\_AES\_set\_encrypt\_key&3 &DA\\
aes\_core.c& 694&private\_AES\_set\_encrypt\_key&4 &DA\\
aes\_core.c& 694&private\_AES\_set\_encrypt\_key&3 &DA\\
aes\_core.c& 695&private\_AES\_set\_encrypt\_key&3 &DA\\
aes\_core.c& 695&private\_AES\_set\_encrypt\_key&4 &DA\\
aes\_core.c& 643&private\_AES\_set\_encrypt\_key&3 &DA\\
aes\_core.c& 644&private\_AES\_set\_encrypt\_key&4 &DA\\
aes\_core.c& 661&private\_AES\_set\_encrypt\_key&3 &DA\\
aes\_core.c& 661&private\_AES\_set\_encrypt\_key&3 &DA\\
aes\_core.c& 661&private\_AES\_set\_encrypt\_key&4 &DA\\
aes\_core.c& 657&private\_AES\_set\_encrypt\_key&4 &DA\\
aes\_core.c& 664&private\_AES\_set\_encrypt\_key&3 &DA\\
aes\_core.c& 661&private\_AES\_set\_encrypt\_key&4 &DA\\
aes\_core.c& 661&private\_AES\_set\_encrypt\_key&4 &DA\\
aes\_core.c& 661&private\_AES\_set\_encrypt\_key&4 &DA\\
aes\_core.c& 661&private\_AES\_set\_encrypt\_key&4 &DA\\
aes\_core.c& 665&private\_AES\_set\_encrypt\_key&3 &DA\\
aes\_core.c& 658&private\_AES\_set\_encrypt\_key&3 &DA\\
aes\_core.c& 659&private\_AES\_set\_encrypt\_key&3 &DA\\
aes\_core.c& 661&private\_AES\_set\_encrypt\_key&3 &DA\\
aes\_core.c& 665&private\_AES\_set\_encrypt\_key&3 &DA\\
aes\_core.c& 661&private\_AES\_set\_encrypt\_key&3 &DA\\
aes\_core.c& 661&private\_AES\_set\_encrypt\_key&4 &DA\\
aes\_core.c& 661&private\_AES\_set\_encrypt\_key&4 &DA\\
aes\_core.c& 661&private\_AES\_set\_encrypt\_key&4 &DA\\
aes\_core.c& 661&private\_AES\_set\_encrypt\_key&4 &DA\\
aes\_core.c& 658&private\_AES\_set\_encrypt\_key&4 &DA\\
aes\_core.c& 661&private\_AES\_set\_encrypt\_key&4 &DA\\
aes\_core.c& 661&private\_AES\_set\_encrypt\_key&4 &DA\\
aes\_core.c& 661&private\_AES\_set\_encrypt\_key&4 &DA\\
aes\_core.c& 658&private\_AES\_set\_encrypt\_key&4 &DA\\
aes\_core.c& 661&private\_AES\_set\_encrypt\_key&4 &DA\\
aes\_core.c& 661&private\_AES\_set\_encrypt\_key&3 &DA\\
aes\_core.c& 661&private\_AES\_set\_encrypt\_key&3 &DA\\
aes\_core.c& 658&private\_AES\_set\_encrypt\_key&4 &DA\\
aes\_core.c& 661&private\_AES\_set\_encrypt\_key&4 &DA\\
aes\_core.c& 663&private\_AES\_set\_encrypt\_key&4 &DA\\
aes\_core.c& 801&AES\_encrypt&5 &DA\\
aes\_core.c& 801&AES\_encrypt&5 &DA\\
aes\_core.c& 802&AES\_encrypt&11&DA\\
aes\_core.c& 802&AES\_encrypt&10&DA\\
aes\_core.c& 878&AES\_encrypt&5 &DA\\
aes\_core.c& 878&AES\_encrypt&11&DA\\
aes\_core.c& 910&AES\_encrypt&4 &DA\\
aes\_core.c& 910&AES\_encrypt&10&DA\\
aes\_core.c& 910&AES\_encrypt&11&DA\\
aes\_core.c& 926&AES\_encrypt&11&DA\\
aes\_core.c& 916&AES\_encrypt&5 &DA\\
aes\_core.c& 918&AES\_encrypt&5 &DA\\
aes\_core.c& 918&AES\_encrypt&10&DA\\
aes\_core.c& 916&AES\_encrypt&10&DA\\
aes\_core.c& 916&AES\_encrypt&5 &DA\\
aes\_core.c& 922&AES\_encrypt&5 &DA\\
aes\_core.c& 922&AES\_encrypt&4 &DA\\
aes\_core.c& 922&AES\_encrypt&6 &DA\\
aes\_core.c& 928&AES\_encrypt&9 &DA\\
aes\_core.c& 884&AES\_encrypt&8 &DA\\
aes\_core.c& 881&AES\_encrypt&6 &DA\\
aes\_core.c& 880&AES\_encrypt&10&DA\\
aes\_core.c& 880&AES\_encrypt&5 &DA\\
aes\_core.c& 886&AES\_encrypt&10&DA\\
aes\_core.c& 888&AES\_encrypt&10&DA\\
aes\_core.c& 889&AES\_encrypt&10&DA\\
aes\_core.c& 901&AES\_encrypt&6 &DA\\
aes\_core.c& 892&AES\_encrypt&6 &DA\\
aes\_core.c& 892&AES\_encrypt&5 &DA\\
aes\_core.c& 892&AES\_encrypt&10&DA\\
aes\_core.c& 894&AES\_encrypt&4 &DA\\
aes\_core.c& 895&AES\_encrypt&9 &DA\\
aes\_core.c& 898&AES\_encrypt&4 &DA\\
aes\_core.c& 898&AES\_encrypt&3 &DA\\
aes\_core.c& 941&AES\_encrypt&4 &DA\\
aes\_core.c& 940&AES\_encrypt&4 &DA\\
aes\_core.c& 940&AES\_encrypt&3 &DA\\
aes\_core.c& 942&AES\_encrypt&4 &DA\\
aes\_core.c& 946&AES\_encrypt&5 &DA\\
aes\_core.c& 946&AES\_encrypt&3 &DA\\
aes\_core.c& 946&AES\_encrypt&3 &DA\\
aes\_core.c& 949&AES\_encrypt&4 &DA\\
aes\_core.c& 947&AES\_encrypt&4 &DA\\
aes\_core.c& 947&AES\_encrypt&3 &DA\\
aes\_core.c& 953&AES\_encrypt&4 &DA\\
aes\_core.c& 953&AES\_encrypt&3 &DA\\
aes\_core.c& 958&AES\_encrypt&4 &DA\\
aes\_core.c& 954&AES\_encrypt&3 &DA\\
\hline
\end{tabular}
%}
\end{table}
%% \begin{table}[h]
\centering
\caption{Summary of all vulnerabilities in AES implemented by openssl 1.0.2k with the amount of leak information}\label{tab:AESopenssl}
\resizebox{\columnwidth}{!}{\begin{tabular}{clrrr}
\hline
\textbf{File} & \textbf{Line Num} & \textbf{Function} & \textbf{Leakedbits} & \textbf{Type} \\\hline
aes\_core.c&654&private\_AES\_set\_encrypt\_key&4 &DA\\
aes\_core.c&654&private\_AES\_set\_encrypt\_key&4 &DA\\
aes\_core.c&672&private\_AES\_set\_encrypt\_key&4 &DA\\
aes\_core.c&673&private\_AES\_set\_encrypt\_key&4 &DA\\
aes\_core.c&694&private\_AES\_set\_encrypt\_key&3 &DA\\
aes\_core.c&694&private\_AES\_set\_encrypt\_key&3 &DA\\
aes\_core.c&695&private\_AES\_set\_encrypt\_key&3 &DA\\
aes\_core.c&695&private\_AES\_set\_encrypt\_key&3 &DA\\
aes\_core.c&643&private\_AES\_set\_encrypt\_key&4 &DA\\
aes\_core.c&644&private\_AES\_set\_encrypt\_key&3 &DA\\
aes\_core.c&661&private\_AES\_set\_encrypt\_key&4 &DA\\
aes\_core.c&661&private\_AES\_set\_encrypt\_key&4 &DA\\
aes\_core.c&661&private\_AES\_set\_encrypt\_key&3 &DA\\
aes\_core.c&657&private\_AES\_set\_encrypt\_key&4 &DA\\
aes\_core.c&664&private\_AES\_set\_encrypt\_key&3 &DA\\
aes\_core.c&661&private\_AES\_set\_encrypt\_key&3 &DA\\
aes\_core.c&661&private\_AES\_set\_encrypt\_key&4 &DA\\
aes\_core.c&661&private\_AES\_set\_encrypt\_key&4 &DA\\
aes\_core.c&661&private\_AES\_set\_encrypt\_key&3 &DA\\
aes\_core.c&665&private\_AES\_set\_encrypt\_key&4 &DA\\
aes\_core.c&658&private\_AES\_set\_encrypt\_key&3 &DA\\
aes\_core.c&659&private\_AES\_set\_encrypt\_key&4 &DA\\
aes\_core.c&661&private\_AES\_set\_encrypt\_key&3 &DA\\
aes\_core.c&665&private\_AES\_set\_encrypt\_key&4 &DA\\
aes\_core.c&661&private\_AES\_set\_encrypt\_key&4 &DA\\
aes\_core.c&661&private\_AES\_set\_encrypt\_key&3 &DA\\
aes\_core.c&661&private\_AES\_set\_encrypt\_key&3 &DA\\
aes\_core.c&661&private\_AES\_set\_encrypt\_key&4 &DA\\
aes\_core.c&661&private\_AES\_set\_encrypt\_key&3 &DA\\
aes\_core.c&658&private\_AES\_set\_encrypt\_key&3 &DA\\
aes\_core.c&661&private\_AES\_set\_encrypt\_key&3 &DA\\
aes\_core.c&661&private\_AES\_set\_encrypt\_key&4 &DA\\
aes\_core.c&661&private\_AES\_set\_encrypt\_key&3 &DA\\
aes\_core.c&658&private\_AES\_set\_encrypt\_key&3 &DA\\
aes\_core.c&661&private\_AES\_set\_encrypt\_key&3 &DA\\
aes\_core.c&661&private\_AES\_set\_encrypt\_key&4 &DA\\
aes\_core.c&661&private\_AES\_set\_encrypt\_key&3 &DA\\
aes\_core.c&658&private\_AES\_set\_encrypt\_key&3 &DA\\
aes\_core.c&661&private\_AES\_set\_encrypt\_key&4 &DA\\
aes\_core.c&663&private\_AES\_set\_encrypt\_key&4 &DA\\
aes\_core.c&801&AES\_encrypt&6 &DA\\
aes\_core.c&801&AES\_encrypt&5 &DA\\
aes\_core.c&802&AES\_encrypt&10&DA\\
aes\_core.c&802&AES\_encrypt&11&DA\\
aes\_core.c&878&AES\_encrypt&5 &DA\\
aes\_core.c&878&AES\_encrypt&10&DA\\
aes\_core.c&910&AES\_encrypt&5 &DA\\
aes\_core.c&910&AES\_encrypt&9 &DA\\
aes\_core.c&910&AES\_encrypt&11&DA\\
aes\_core.c&926&AES\_encrypt&12&DA\\
aes\_core.c&916&AES\_encrypt&5 &DA\\
aes\_core.c&918&AES\_encrypt&4 &DA\\
aes\_core.c&918&AES\_encrypt&11&DA\\
aes\_core.c&916&AES\_encrypt&11&DA\\
aes\_core.c&916&AES\_encrypt&5 &DA\\
aes\_core.c&922&AES\_encrypt&5 &DA\\
aes\_core.c&922&AES\_encrypt&4 &DA\\
aes\_core.c&922&AES\_encrypt&6 &DA\\
aes\_core.c&928&AES\_encrypt&9 &DA\\
aes\_core.c&884&AES\_encrypt&9 &DA\\
aes\_core.c&881&AES\_encrypt&5 &DA\\
aes\_core.c&880&AES\_encrypt&10&DA\\
aes\_core.c&880&AES\_encrypt&5 &DA\\
aes\_core.c&886&AES\_encrypt&10&DA\\
aes\_core.c&888&AES\_encrypt&10&DA\\
aes\_core.c&889&AES\_encrypt&9 &DA\\
aes\_core.c&901&AES\_encrypt&5 &DA\\
aes\_core.c&892&AES\_encrypt&5 &DA\\
aes\_core.c&892&AES\_encrypt&6 &DA\\
aes\_core.c&892&AES\_encrypt&8 &DA\\
aes\_core.c&894&AES\_encrypt&4 &DA\\
aes\_core.c&895&AES\_encrypt&9 &DA\\
aes\_core.c&898&AES\_encrypt&3 &DA\\
aes\_core.c&898&AES\_encrypt&4 &DA\\
aes\_core.c&941&AES\_encrypt&4 &DA\\
aes\_core.c&940&AES\_encrypt&4 &DA\\
aes\_core.c&940&AES\_encrypt&4 &DA\\
aes\_core.c&942&AES\_encrypt&4 &DA\\
aes\_core.c&946&AES\_encrypt&5 &DA\\
aes\_core.c&946&AES\_encrypt&3 &DA\\
aes\_core.c&946&AES\_encrypt&4 &DA\\
aes\_core.c&949&AES\_encrypt&4 &DA\\
aes\_core.c&947&AES\_encrypt&3 &DA\\
aes\_core.c&947&AES\_encrypt&4 &DA\\
aes\_core.c&953&AES\_encrypt&4 &DA\\
aes\_core.c&953&AES\_encrypt&4 &DA\\
aes\_core.c&958&AES\_encrypt&4 &DA\\
aes\_core.c&954&AES\_encrypt&3 &DA\\
\hline
\end{tabular}
}
\end{table}
\begin{table}[h!]
\centering\tiny\scriptsize
\renewcommand{\baselinestretch}{0.96}\selectfont
\caption{Leakages in AES implemented by openssl 1.1.0f}\label{tab:AESopenssl1.1.0f}
%\resizebox{\columnwidth}{!}{
\begin{tabular}{lrlrr}
\hline
\textbf{File} & \textbf{Line No.} & \textbf{Function} & \textbf{\# Leaked Bits} & \textbf{Type} \\\hline
aes\_core.c& 676&AES\_set\_encrypt\_key&3 &DA\\
aes\_core.c& 676&AES\_set\_encrypt\_key&5 &DA\\
aes\_core.c& 676&AES\_set\_encrypt\_key&4 &DA\\
aes\_core.c& 677&AES\_set\_encrypt\_key&3 &DA\\
aes\_core.c& 698&AES\_set\_encrypt\_key&4 &DA\\
aes\_core.c& 698&AES\_set\_encrypt\_key&4 &DA\\
aes\_core.c& 699&AES\_set\_encrypt\_key&4 &DA\\
aes\_core.c& 700&AES\_set\_encrypt\_key&3 &DA\\
aes\_core.c& 648&AES\_set\_encrypt\_key&4 &DA\\
aes\_core.c& 664&AES\_set\_encrypt\_key&4 &DA\\
aes\_core.c& 665&AES\_set\_encrypt\_key&3 &DA\\
aes\_core.c& 662&AES\_set\_encrypt\_key&3 &DA\\
aes\_core.c& 665&AES\_set\_encrypt\_key&3 &DA\\
aes\_core.c& 665&AES\_set\_encrypt\_key&4 &DA\\
aes\_core.c& 668&AES\_set\_encrypt\_key&3 &DA\\
aes\_core.c& 665&AES\_set\_encrypt\_key&4 &DA\\
aes\_core.c& 665&AES\_set\_encrypt\_key&4 &DA\\
aes\_core.c& 665&AES\_set\_encrypt\_key&3 &DA\\
aes\_core.c& 665&AES\_set\_encrypt\_key&4 &DA\\
aes\_core.c& 665&AES\_set\_encrypt\_key&3 &DA\\
aes\_core.c& 665&AES\_set\_encrypt\_key&3 &DA\\
aes\_core.c& 665&AES\_set\_encrypt\_key&3 &DA\\
aes\_core.c& 665&AES\_set\_encrypt\_key&4 &DA\\
aes\_core.c& 668&AES\_set\_encrypt\_key&4 &DA\\
aes\_core.c& 665&AES\_set\_encrypt\_key&3 &DA\\
aes\_core.c& 662&AES\_set\_encrypt\_key&4 &DA\\
aes\_core.c& 665&AES\_set\_encrypt\_key&3 &DA\\
aes\_core.c& 661&AES\_set\_encrypt\_key&3 &DA\\
aes\_core.c& 665&AES\_set\_encrypt\_key&3 &DA\\
aes\_core.c& 662&AES\_set\_encrypt\_key&3 &DA\\
aes\_core.c& 665&AES\_set\_encrypt\_key&4 &DA\\
aes\_core.c& 665&AES\_set\_encrypt\_key&4 &DA\\
aes\_core.c& 665&AES\_set\_encrypt\_key&4 &DA\\
aes\_core.c& 665&AES\_set\_encrypt\_key&3 &DA\\
aes\_core.c& 665&AES\_set\_encrypt\_key&3 &DA\\
aes\_core.c& 665&AES\_set\_encrypt\_key&3 &DA\\
aes\_core.c& 665&AES\_set\_encrypt\_key&3 &DA\\
aes\_core.c& 662&AES\_set\_encrypt\_key&3 &DA\\
aes\_core.c& 665&AES\_set\_encrypt\_key&4 &DA\\
aes\_core.c& 661&AES\_set\_encrypt\_key&4 &DA\\
aes\_core.c& 805&AES\_encrypt&5 &DA\\
aes\_core.c& 805&AES\_encrypt&7 &DA\\
aes\_core.c& 806&AES\_encrypt&11&DA\\
aes\_core.c& 806&AES\_encrypt&6 &DA\\
aes\_core.c& 882&AES\_encrypt&11&DA\\
aes\_core.c& 882&AES\_encrypt&10&DA\\
aes\_core.c& 797&AES\_encrypt&7 &DA\\
aes\_core.c& 797&AES\_encrypt&11&DA\\
aes\_core.c& 916&AES\_encrypt&5 &DA\\
aes\_core.c& 917&AES\_encrypt&12&DA\\
aes\_core.c& 924&AES\_encrypt&9 &DA\\
aes\_core.c& 914&AES\_encrypt&11&DA\\
aes\_core.c& 920&AES\_encrypt&7 &DA\\
aes\_core.c& 920&AES\_encrypt&12&DA\\
aes\_core.c& 920&AES\_encrypt&7 &DA\\
aes\_core.c& 920&AES\_encrypt&7 &DA\\
aes\_core.c& 929&AES\_encrypt&5 &DA\\
aes\_core.c& 932&AES\_encrypt&7 &DA\\
aes\_core.c& 934&AES\_encrypt&11&DA\\
aes\_core.c& 884&AES\_encrypt&11&DA\\
aes\_core.c& 885&AES\_encrypt&7 &DA\\
aes\_core.c& 886&AES\_encrypt&10&DA\\
aes\_core.c& 891&AES\_encrypt&5 &DA\\
aes\_core.c& 890&AES\_encrypt&9 &DA\\
aes\_core.c& 897&AES\_encrypt&6 &DA\\
aes\_core.c& 897&AES\_encrypt&11&DA\\
aes\_core.c& 896&AES\_encrypt&10&DA\\
aes\_core.c& 898&AES\_encrypt&5 &DA\\
aes\_core.c& 896&AES\_encrypt&11&DA\\
aes\_core.c& 899&AES\_encrypt&5 &DA\\
aes\_core.c& 902&AES\_encrypt&6 &DA\\
aes\_core.c& 902&AES\_encrypt&9 &DA\\
aes\_core.c& 910&AES\_encrypt&3 &DA\\
aes\_core.c& 910&AES\_encrypt&3 &DA\\
aes\_core.c& 944&AES\_encrypt&3 &DA\\
aes\_core.c& 944&AES\_encrypt&4 &DA\\
aes\_core.c& 944&AES\_encrypt&4 &DA\\
aes\_core.c& 944&AES\_encrypt&4 &DA\\
aes\_core.c& 950&AES\_encrypt&3 &DA\\
aes\_core.c& 952&AES\_encrypt&4 &DA\\
aes\_core.c& 951&AES\_encrypt&3 &DA\\
aes\_core.c& 951&AES\_encrypt&3 &DA\\
aes\_core.c& 961&AES\_encrypt&3 &DA\\
aes\_core.c& 957&AES\_encrypt&4 &DA\\
aes\_core.c& 957&AES\_encrypt&4 &DA\\
aes\_core.c& 958&AES\_encrypt&5 &DA\\
aes\_core.c& 958&AES\_encrypt&3 &DA\\
aes\_core.c& 960&AES\_encrypt&3 &DA\\
\hline
\end{tabular}
%}
\renewcommand{\baselinestretch}{1.0}\selectfont
\end{table}

\begin{table*}%[h]
\centering
\caption{Summary of all vulnerabilities in AES implemented by openssl 1.1.1 with the amount of leak informationThe mark $*$ means timeout,which indicates more severe leakages (see \S
ef{loc:timeout}).}\label{tab:AESopenssl}
%\resizebox{\columnwidth}{!}{\begin{tabular}{clrrr}
\hline
\textbf{File} & \textbf{Line Num} & \textbf{Function} & \textbf{Leakedbits} & \textbf{Type} \\\hline
aes\_core.c&665&AES\_set\_encrypt\_key&4 &DA\\
aes\_core.c&665&AES\_set\_encrypt\_key&3 &DA\\
aes\_core.c&665&AES\_set\_encrypt\_key&3 &DA\\
aes\_core.c&665&AES\_set\_encrypt\_key&4 &DA\\
aes\_core.c&665&AES\_set\_encrypt\_key&3 &DA\\
aes\_core.c&665&AES\_set\_encrypt\_key&3 &DA\\
aes\_core.c&665&AES\_set\_encrypt\_key&4 &DA\\
aes\_core.c&665&AES\_set\_encrypt\_key&3 &DA\\
aes\_core.c&665&AES\_set\_encrypt\_key&4 &DA\\
aes\_core.c&665&AES\_set\_encrypt\_key&4 &DA\\
aes\_core.c&665&AES\_set\_encrypt\_key&3 &DA\\
aes\_core.c&665&AES\_set\_encrypt\_key&3 &DA\\
aes\_core.c&665&AES\_set\_encrypt\_key&4 &DA\\
aes\_core.c&665&AES\_set\_encrypt\_key&4 &DA\\
aes\_core.c&665&AES\_set\_encrypt\_key&4 &DA\\
aes\_core.c&665&AES\_set\_encrypt\_key&4 &DA\\
aes\_core.c&665&AES\_set\_encrypt\_key&3 &DA\\
aes\_core.c&665&AES\_set\_encrypt\_key&4 &DA\\
aes\_core.c&665&AES\_set\_encrypt\_key&3 &DA\\
aes\_core.c&665&AES\_set\_encrypt\_key&4 &DA\\
aes\_core.c&665&AES\_set\_encrypt\_key&3 &DA\\
aes\_core.c&665&AES\_set\_encrypt\_key&4 &DA\\
aes\_core.c&665&AES\_set\_encrypt\_key&3 &DA\\
aes\_core.c&665&AES\_set\_encrypt\_key&4 &DA\\
aes\_core.c&665&AES\_set\_encrypt\_key&3 &DA\\
aes\_core.c&665&AES\_set\_encrypt\_key&4 &DA\\
aes\_core.c&665&AES\_set\_encrypt\_key&3 &DA\\
aes\_core.c&665&AES\_set\_encrypt\_key&4 &DA\\
aes\_core.c&665&AES\_set\_encrypt\_key&4 &DA\\
aes\_core.c&665&AES\_set\_encrypt\_key&4 &DA\\
aes\_core.c&665&AES\_set\_encrypt\_key&4 &DA\\
aes\_core.c&665&AES\_set\_encrypt\_key&4 &DA\\
aes\_core.c&665&AES\_set\_encrypt\_key&3 &DA\\
aes\_core.c&665&AES\_set\_encrypt\_key&3 &DA\\
aes\_core.c&665&AES\_set\_encrypt\_key&3 &DA\\
aes\_core.c&665&AES\_set\_encrypt\_key&4 &DA\\
aes\_core.c&665&AES\_set\_encrypt\_key&4 &DA\\
aes\_core.c&665&AES\_set\_encrypt\_key&4 &DA\\
aes\_core.c&665&AES\_set\_encrypt\_key&4 &DA\\
aes\_core.c&665&AES\_set\_encrypt\_key&3 &DA\\
aes\_core.c&884&AES\_encrypt&4 &DA\\
aes\_core.c&884&AES\_encrypt&7 &DA\\
aes\_core.c&884&AES\_encrypt&3 &DA\\
aes\_core.c&884&AES\_encrypt&7 &DA\\
aes\_core.c&890&AES\_encrypt&3 &DA\\
aes\_core.c&890&AES\_encrypt&4 &DA\\
aes\_core.c&890&AES\_encrypt&7 &DA\\
aes\_core.c&890&AES\_encrypt&7 &DA\\
aes\_core.c&896&AES\_encrypt&7 &DA\\
aes\_core.c&896&AES\_encrypt&3 &DA\\
aes\_core.c&896&AES\_encrypt&8 &DA\\
aes\_core.c&896&AES\_encrypt&4 &DA\\
aes\_core.c&902&AES\_encrypt&8 &DA\\
aes\_core.c&902&AES\_encrypt&8 &DA\\
aes\_core.c&902&AES\_encrypt&4 &DA\\
aes\_core.c&902&AES\_encrypt&3 &DA\\
aes\_core.c&914&AES\_encrypt&3 &DA\\
aes\_core.c&914&AES\_encrypt&8 &DA\\
aes\_core.c&914&AES\_encrypt&3 &DA\\
aes\_core.c&914&AES\_encrypt&7 &DA\\
aes\_core.c&920&AES\_encrypt&4 &DA\\
aes\_core.c&920&AES\_encrypt&4 &DA\\
aes\_core.c&920&AES\_encrypt&7 &DA\\
aes\_core.c&920&AES\_encrypt&7 &DA\\
aes\_core.c&926&AES\_encrypt&8 &DA\\
aes\_core.c&926&AES\_encrypt&3 &DA\\
aes\_core.c&926&AES\_encrypt&7 &DA\\
aes\_core.c&926&AES\_encrypt&4 &DA\\
aes\_core.c&932&AES\_encrypt&8 &DA\\
aes\_core.c&932&AES\_encrypt&7 &DA\\
aes\_core.c&932&AES\_encrypt&4 &DA\\
aes\_core.c&932&AES\_encrypt&3 &DA\\
aes\_core.c&944&AES\_encrypt&3 &DA\\
aes\_core.c&944&AES\_encrypt&4 &DA\\
aes\_core.c&944&AES\_encrypt&3 &DA\\
aes\_core.c&944&AES\_encrypt&4 &DA\\
aes\_core.c&951&AES\_encrypt&4 &DA\\
aes\_core.c&951&AES\_encrypt&4 &DA\\
aes\_core.c&951&AES\_encrypt&4 &DA\\
aes\_core.c&951&AES\_encrypt&3 &DA\\
aes\_core.c&958&AES\_encrypt&4 &DA\\
aes\_core.c&958&AES\_encrypt&3 &DA\\
aes\_core.c&958&AES\_encrypt&3 &DA\\
aes\_core.c&958&AES\_encrypt&4 &DA\\
aes\_core.c&965&AES\_encrypt&3 &DA\\
aes\_core.c&965&AES\_encrypt&3 &DA\\
aes\_core.c&965&AES\_encrypt&3 &DA\\
aes\_core.c&965&AES\_encrypt&4 &DA\\
\hline
\end{tabular}
%}
\end{table*}


\begin{table*}%[h]
\centering
\caption{Summary of all vulnerabilities in DES implemented by mbedTLS 2.5 with the amount of leak informationThe mark $*$ means timeout,which indicates more severe leakages (see \S\ref{loc:timeout}).}\label{tab:DESmbedTLS}
%\resizebox{\columnwidth}{!}{
\begin{tabular}{clrrr}
\hline
\textbf{File} & \textbf{Line Num} & \textbf{Function} & \textbf{Leakedbits} & \textbf{Type} \\\hline
des.c& 441&mbedtls\_des\_setkey&0 &DA\\
des.c& 438&mbedtls\_des\_setkey&0 &DA\\
des.c& 438&mbedtls\_des\_setkey&1 &DA\\
des.c& 439&mbedtls\_des\_setkey&1 &DA\\
des.c& 439&mbedtls\_des\_setkey&1 &DA\\
des.c& 440&mbedtls\_des\_setkey&1 &DA\\
des.c& 446&mbedtls\_des\_setkey&0 &DA\\
des.c& 446&mbedtls\_des\_setkey&0 &DA\\
des.c& 444&mbedtls\_des\_setkey&1 &DA\\
des.c& 444&mbedtls\_des\_setkey&0 &DA\\
des.c& 443&mbedtls\_des\_setkey&1 &DA\\
des.c& 443&mbedtls\_des\_setkey&1 &DA\\
des.c& 444&mbedtls\_des\_setkey&0 &DA\\
des.c& 445&mbedtls\_des\_setkey&0 &DA\\
des.c& 448&mbedtls\_des\_setkey&1 &DA\\
\hline
\end{tabular}
%}
\end{table*}
%% \begin{table}[h!]
\centering\tiny\scriptsize
\caption{Leakages in DES implemented by mbedTLS 2.15.1}\label{tab:DESmbedTLS2.15.1}
%\resizebox{\columnwidth}{!}{
\begin{tabular}{lrlrr}
\hline
\textbf{File} & \textbf{Line No.} & \textbf{Function} & \textbf{\# Leaked Bits} & \textbf{Type} \\\hline
des.c& 437&mbedtls\_des\_setkey&0 &DA\\
des.c& 434&mbedtls\_des\_setkey&1 &DA\\
des.c& 434&mbedtls\_des\_setkey&0 &DA\\
des.c& 435&mbedtls\_des\_setkey&0 &DA\\
des.c& 435&mbedtls\_des\_setkey&0 &DA\\
des.c& 436&mbedtls\_des\_setkey&0 &DA\\
des.c& 442&mbedtls\_des\_setkey&1 &DA\\
des.c& 442&mbedtls\_des\_setkey&0 &DA\\
des.c& 440&mbedtls\_des\_setkey&1 &DA\\
des.c& 440&mbedtls\_des\_setkey&0 &DA\\
des.c& 439&mbedtls\_des\_setkey&0 &DA\\
des.c& 439&mbedtls\_des\_setkey&0 &DA\\
des.c& 440&mbedtls\_des\_setkey&1 &DA\\
des.c& 441&mbedtls\_des\_setkey&1 &DA\\
des.c& 444&mbedtls\_des\_setkey&1 &DA\\
\hline
\end{tabular}
%}
\renewcommand{\baselinestretch}{1.0}\selectfont
\end{table}

%% \begin{table}%[h]
\centering\tiny
\caption{Summary of all vulnerabilities in DES implemented by openssl 0.9.7 with the amount of leak information. The mark $*$ means timeout, which indicates more severe leakages (see \S\ref{loc:timeout}).}\label{tab:DESopenssl}
%\resizebox{\columnwidth}{!}{
\begin{tabular}{clrrr}
\hline
\textbf{File} & \textbf{Line Num} & \textbf{Function} & \textbf{Leakedbits} & \textbf{Type} \\\hline
set\_key.c& 380&DES\_set\_key\_unchecked&6 &DA\\
set\_key.c& 380&DES\_set\_key\_unchecked&6 &DA\\
set\_key.c& 380&DES\_set\_key\_unchecked&8 &DA\\
set\_key.c& 380&DES\_set\_key\_unchecked&5 &DA\\
set\_key.c& 385&DES\_set\_key\_unchecked&2 &DA\\
set\_key.c& 385&DES\_set\_key\_unchecked&2 &DA\\
\hline
\end{tabular}
%}
\end{table}
%% \begin{table}[h!]
\centering\tiny\scriptsize
\renewcommand{\baselinestretch}{0.96}\selectfont
\caption{Leakages in DES implemented by openssl 1.0.2f}\label{tab:DESopenssl1.0.2f}
%\resizebox{\columnwidth}{!}{
\begin{tabular}{lrlrr}
\hline
\textbf{File} & \textbf{Line No.} & \textbf{Function} & \textbf{\# Leaked Bits} & \textbf{Type} \\\hline
set\_key.c& 409&DES\_set\_key\_unchecked&9 &DA\\
set\_key.c& 417&DES\_set\_key\_unchecked&8 &DA\\
set\_key.c& 418&DES\_set\_key\_unchecked&7 &DA\\
set\_key.c& 423&DES\_set\_key\_unchecked&5 &DA\\
set\_key.c& 420&DES\_set\_key\_unchecked&1 &DA\\
set\_key.c& 422&DES\_set\_key\_unchecked&2 &DA\\
set\_key.c& 422&DES\_set\_key\_unchecked&3 &DA\\
set\_key.c& 425&DES\_set\_key\_unchecked&0 &DA\\
\hline
\end{tabular}
%}
\renewcommand{\baselinestretch}{1.0}\selectfont
\end{table}

%% \begin{table}%[h]
\centering\tiny
\caption{Summary of all vulnerabilities in DES implemented by openssl 1.0.2k with the amount of leak information. The mark $*$ means timeout, which indicates more severe leakages (see \S\ref{loc:timeout}).}\label{tab:DESopenssl}
%\resizebox{\columnwidth}{!}{
\begin{tabular}{clrrr}
\hline
\textbf{File} & \textbf{Line Num} & \textbf{Function} & \textbf{Leakedbits} & \textbf{Type} \\\hline
set\_key.c& 409&DES\_set\_key\_unchecked&7 &DA\\
set\_key.c& 417&DES\_set\_key\_unchecked&9 &DA\\
set\_key.c& 418&DES\_set\_key\_unchecked&7 &DA\\
set\_key.c& 423&DES\_set\_key\_unchecked&5 &DA\\
set\_key.c& 420&DES\_set\_key\_unchecked&3 &DA\\
set\_key.c& 422&DES\_set\_key\_unchecked&4 &DA\\
set\_key.c& 422&DES\_set\_key\_unchecked&3 &DA\\
set\_key.c& 425&DES\_set\_key\_unchecked&0 &DA\\
\hline
\end{tabular}
%}
\end{table}
\begin{table}[h]
\centering
\caption{}\label{fig:}
\resizebox{\columnwidth}{!}{\begin{tabular}{clrrr}
\hline
\textbf{File} & \textbf{Line Num} & \textbf{Function} & \textbf{Leakedbits} & \textbf{Type} \\\hline
set\_key.c&351&DES\_set\_key\_unchecked&7 &DA\\
set\_key.c&353&DES\_set\_key\_unchecked&9 &DA\\
set\_key.c&361&DES\_set\_key\_unchecked&7 &DA\\
set\_key.c&362&DES\_set\_key\_unchecked&6 &DA\\
set\_key.c&362&DES\_set\_key\_unchecked&2 &DA\\
set\_key.c&364&DES\_set\_key\_unchecked&3 &DA\\
set\_key.c&364&DES\_set\_key\_unchecked&4 &DA\\
set\_key.c&365&DES\_set\_key\_unchecked&0 &DA\\
&&&&\\
\hline
\end{tabular}
}
\end{table}
\begin{table}%[h]
\centering\tiny
\caption{Summary of all vulnerabilities in DES implemented by openssl 1.1.1 with the amount of leak information. The mark $*$ means timeout, which indicates more severe leakages (see \S\ref{loc:timeout}).}\label{tab:DESopenssl}
%\resizebox{\columnwidth}{!}{
\begin{tabular}{clrrr}
\hline
\textbf{File} & \textbf{Line Num} & \textbf{Function} & \textbf{Leakedbits} & \textbf{Type} \\\hline
set\_key.c& 350&DES\_set\_key\_unchecked&5 &DA\\
set\_key.c& 350&DES\_set\_key\_unchecked&6 &DA\\
set\_key.c& 350&DES\_set\_key\_unchecked&7 &DA\\
set\_key.c& 350&DES\_set\_key\_unchecked&6 &DA\\
set\_key.c& 355&DES\_set\_key\_unchecked&2 &DA\\
set\_key.c& 355&DES\_set\_key\_unchecked&2 &DA\\
\hline
\end{tabular}
%}
\end{table}

\begin{table}%[h]
\centering\tiny
\caption{Summary of all vulnerabilities in RSA implemented by mbedTLS 2.5 with the amount of leak information. The mark $*$ means timeout, which indicates more severe leakages (see \S\ref{loc:timeout}).}\label{tab:RSAmbedTLS}
%\resizebox{\columnwidth}{!}{
\begin{tabular}{clrrr}
\hline
\textbf{File} & \textbf{Line Num} & \textbf{Function} & \textbf{Leakedbits} & \textbf{Type} \\\hline
bignum.c& 1617&mbedtls\_mpi\_exp\_mod&1 &CF\\
bignum.c& 861&mbedtls\_mpi\_cmp\_mpi&8 &CF\\
bignum.c& 862&mbedtls\_mpi\_cmp\_mpi&8 &CF\\
bignum.c& 1167&mpi\_mul\_hlp&*&\\
bignum.c& 828&mbedtls\_mpi\_cmp\_abs&9 &CF\\
bignum.c& 829&mbedtls\_mpi\_cmp\_abs&9 &CF\\
\hline
\end{tabular}
%}
\end{table}
%% \begin{table}%[h]
\centering\scriptsize
\caption{Summary of all vulnerabilities in RSA implemented by mbedTLS 2.15.1 with the amount of leak informationThe mark $*$ means timeout,which indicates more severe leakages (see \S\ref{loc:timeout}).}\label{tab:RSAmbedTLS}
%\resizebox{\columnwidth}{!}{
\begin{tabular}{clrrr}
\hline
\textbf{File} & \textbf{Line Num} & \textbf{Function} & \textbf{Leakedbits} & \textbf{Type} \\\hline
bignum.c& 855&mbedtls\_mpi\_cmp\_mpi&*&\\
rsa.c& 184&rsa\_check\_context.isra.0&0 &CF\\
bignum.c& 825&mbedtls\_mpi\_cmp\_abs&*&\\
bignum.c& 197&mbedtls\_mpi\_copy&*&\\
bignum.c& 1629&mbedtls\_mpi\_exp\_mod&1 &CF\\
bignum.c& 829&mbedtls\_mpi\_cmp\_abs&*&\\
bignum.c& 859&mbedtls\_mpi\_cmp\_mpi&*&\\
bignum.c& 873&mbedtls\_mpi\_cmp\_mpi&2 &CF\\
bignum.c& 874&mbedtls\_mpi\_cmp\_mpi&2 &CF\\
bignum.c& 840&mbedtls\_mpi\_cmp\_abs&8 &CF\\
bignum.c& 841&mbedtls\_mpi\_cmp\_abs&9 &CF\\
bignum.c& 1201&mbedtls\_mpi\_mul\_mpi&*&\\
\hline
\end{tabular}
%}
\end{table}
%% \begin{table}%[h]
\centering
\caption{Summary of all vulnerabilities in RSA implemented by openssl 0.9.7 with the amount of leak informationThe mark $*$ means timeout,which indicates more severe leakages (see \S\ref{loc:timeout}).}\label{tab:RSAopenssl}
%\resizebox{\columnwidth}{!}{
\begin{tabular}{clrrr}
\hline
\textbf{File} & \textbf{Line Num} & \textbf{Function} & \textbf{Leakedbits} & \textbf{Type} \\\hline
bn\_lib.c& 228&BN\_num\_bits&0 &CF\\
bn\_lib.c& 229&BN\_num\_bits&3 &DA\\
bn\_shift.c& 152&BN\_lshift&1 &CF\\
bn\_lib.c& 670&BN\_ucmp&*&\\
a\_gentm.c& 246&\_\_udivdi3&*&\\
bn\_div.c& 303&BN\_div&*&\\
bn\_add.c& 222&BN\_usub&4 &CF\\
bn\_add.c& 256&BN\_usub&*&\\
bn\_add.c& 256&BN\_usub&*&\\
bn\_gcd.c& 247&BN\_mod\_inverse&0 &CF\\
bn\_gcd.c& 268&BN\_mod\_inverse&7 &CF\\
bn\_gcd.c& 287&BN\_mod\_inverse&4 &CF\\
bn\_gcd.c& 291&BN\_mod\_inverse&*&\\
bn\_gcd.c& 272&BN\_mod\_inverse&12&CF\\
bn\_lib.c& 670&BN\_ucmp&*&\\
bn\_div.c& 303&BN\_div&*&\\
bn\_div.c& 307&BN\_div&*&\\
bn\_exp.c& 590&BN\_mod\_exp\_mont\_consttime&0 &CF\\
bn\_mont.c& 218&BN\_from\_montgomery&*&\\
bn\_asm.c& 691&bn\_sqr\_comba8&*&\\
bn\_asm.c& 696&bn\_sqr\_comba8&11&CF\\
bn\_asm.c& 696&bn\_sqr\_comba8&10&CF\\
bn\_asm.c& 704&bn\_sqr\_comba8&*&\\
bn\_asm.c& 709&bn\_sqr\_comba8&*&\\
bn\_asm.c& 709&bn\_sqr\_comba8&13&CF\\
bn\_asm.c& 710&bn\_sqr\_comba8&14&CF\\
bn\_asm.c& 714&bn\_sqr\_comba8&*&\\
bn\_asm.c& 714&bn\_sqr\_comba8&8 &CF\\
bn\_asm.c& 715&bn\_sqr\_comba8&*&\\
bn\_asm.c& 718&bn\_sqr\_comba8&14&CF\\
bn\_asm.c& 718&bn\_sqr\_comba8&11&CF\\
bn\_asm.c& 720&bn\_sqr\_comba8&*&\\
bn\_asm.c& 721&bn\_sqr\_comba8&*&\\
bn\_asm.c& 727&bn\_sqr\_comba8&*&\\
bn\_asm.c& 728&bn\_sqr\_comba8&*&\\
bn\_asm.c& 728&bn\_sqr\_comba8&*&\\
bn\_asm.c& 730&bn\_sqr\_comba8&10&CF\\
bn\_asm.c& 730&bn\_sqr\_comba8&9 &CF\\
bn\_asm.c& 731&bn\_sqr\_comba8&*&\\
bn\_asm.c& 732&bn\_sqr\_comba8&*&\\
bn\_asm.c& 733&bn\_sqr\_comba8&*&\\
bn\_asm.c& 737&bn\_sqr\_comba8&*&\\
bn\_asm.c& 738&bn\_sqr\_comba8&*&\\
bn\_asm.c& 738&bn\_sqr\_comba8&*&\\
bn\_asm.c& 740&bn\_sqr\_comba8&*&\\
bn\_asm.c& 740&bn\_sqr\_comba8&*&\\
bn\_asm.c& 746&bn\_sqr\_comba8&*&\\
bn\_asm.c& 748&bn\_sqr\_comba8&*&\\
bn\_asm.c& 748&bn\_sqr\_comba8&*&\\
bn\_asm.c& 692&bn\_sqr\_comba8&*&\\
bn\_asm.c& 698&bn\_sqr\_comba8&15&CF\\
bn\_asm.c& 698&bn\_sqr\_comba8&11&CF\\
bn\_asm.c& 705&bn\_sqr\_comba8&*&\\
bn\_asm.c& 707&bn\_sqr\_comba8&*&\\
bn\_asm.c& 707&bn\_sqr\_comba8&*&\\
bn\_asm.c& 716&bn\_sqr\_comba8&*&\\
bn\_asm.c& 719&bn\_sqr\_comba8&*&\\
bn\_asm.c& 720&bn\_sqr\_comba8&*&\\
bn\_asm.c& 721&bn\_sqr\_comba8&*&\\
bn\_asm.c& 722&bn\_sqr\_comba8&*&\\
bn\_asm.c& 726&bn\_sqr\_comba8&*&\\
bn\_asm.c& 727&bn\_sqr\_comba8&*&\\
bn\_asm.c& 714&bn\_sqr\_comba8&*&\\
bn\_asm.c& 715&bn\_sqr\_comba8&*&\\
bn\_asm.c& 716&bn\_sqr\_comba8&*&\\
bn\_asm.c& 721&bn\_sqr\_comba8&12&CF\\
bn\_asm.c& 722&bn\_sqr\_comba8&*&\\
bn\_asm.c& 726&bn\_sqr\_comba8&*&\\
bn\_asm.c& 732&bn\_sqr\_comba8&*&\\
bn\_asm.c& 733&bn\_sqr\_comba8&*&\\
bn\_asm.c& 737&bn\_sqr\_comba8&*&\\
bn\_asm.c& 741&bn\_sqr\_comba8&*&\\
bn\_asm.c& 742&bn\_sqr\_comba8&*&\\
bn\_asm.c& 742&bn\_sqr\_comba8&*&\\
bn\_asm.c& 746&bn\_sqr\_comba8&*&\\
bn\_asm.c& 749&bn\_sqr\_comba8&*&\\
bn\_asm.c& 700&bn\_sqr\_comba8&*&\\
bn\_asm.c& 700&bn\_sqr\_comba8&13&CF\\
bn\_asm.c& 704&bn\_sqr\_comba8&*&\\
bn\_asm.c& 726&bn\_sqr\_comba8&8 &CF\\
bn\_asm.c& 737&bn\_sqr\_comba8&*&\\
bn\_asm.c& 720&bn\_sqr\_comba8&*&\\
bn\_asm.c& 742&bn\_sqr\_comba8&*&\\
bn\_asm.c& 732&bn\_sqr\_comba8&*&\\
bn\_asm.c& 716&bn\_sqr\_comba8&*&\\
bn\_asm.c& 727&bn\_sqr\_comba8&*&\\
bn\_asm.c& 691&bn\_sqr\_comba8&10&CF\\
bn\_asm.c& 696&bn\_sqr\_comba8&*&\\
bn\_asm.c& 699&bn\_sqr\_comba8&*&\\
bn\_asm.c& 700&bn\_sqr\_comba8&*&\\
bn\_asm.c& 704&bn\_sqr\_comba8&*&\\
bn\_asm.c& 705&bn\_sqr\_comba8&*&\\
bn\_asm.c& 708&bn\_sqr\_comba8&*&\\
bn\_asm.c& 709&bn\_sqr\_comba8&*&\\
bn\_asm.c& 710&bn\_sqr\_comba8&*&\\
bn\_asm.c& 722&bn\_sqr\_comba8&10&CF\\
bn\_asm.c& 705&bn\_sqr\_comba8&14&CF\\
bn\_asm.c& 738&bn\_sqr\_comba8&6 &CF\\
bn\_asm.c& 746&bn\_sqr\_comba8&7 &CF\\
bn\_asm.c& 728&bn\_sqr\_comba8&10&CF\\
bn\_asm.c& 715&bn\_sqr\_comba8&13&CF\\
bn\_asm.c& 733&bn\_sqr\_comba8&11&CF\\
bn\_asm.c& 710&bn\_sqr\_comba8&10&CF\\
bn\_div.c& 303&BN\_div&1 &CF\\
bn\_div.c& 303&BN\_div&1 &CF\\
bn\_lib.c& 228&BN\_bn2bin&3 &CF\\
bn\_lib.c& 230&BN\_bn2bin&4 &DA\\
\hline
\end{tabular}
%}
\end{table}
%% \begin{table}%[h]
\centering\tiny\scriptsize
\caption{RSA implemented by openssl 1.0.2f}\label{tab:RSAopenssl}
%\caption{Summary of all vulnerabilities in RSA implemented by openssl 1.0.2f with the amount of leak information. The mark $*$ means timeout, which indicates more severe leakages (see \S\ref{loc:timeout}).}\label{tab:RSAopenssl}
%\resizebox{\columnwidth}{!}{
\begin{tabular}{clrrr}
\hline
\textbf{File} & \textbf{Line No.} & \textbf{Function} & \textbf{Leaked Bits} & \textbf{Type} \\\hline
bn\_lib.c& 199&BN\_num\_bits&*&\\
bn\_lib.c& 200&BN\_num\_bits&*&\\
bn\_lib.c& 201&BN\_num\_bits&*&\\
bn\_lib.c& 673&BN\_ucmp&*&\\
bio\_asn1.c& 482&\_\_udivdi3&8 &CF\\
bn\_div.c& 381&BN\_div&*&\\
bn\_div.c& 456&BN\_div&*&\\
bn\_gcd.c& 279&BN\_mod\_inverse&1 &CF\\
bn\_gcd.c& 302&BN\_mod\_inverse&5 &CF\\
bn\_gcd.c& 324&BN\_mod\_inverse&6 &CF\\
bn\_add.c& 255&BN\_usub&*&\\
bn\_gcd.c& 305&BN\_mod\_inverse&11&CF\\
bn\_gcd.c& 327&BN\_mod\_inverse&*&\\
bn\_lib.c& 203&BN\_num\_bits&*&\\
bn\_lib.c& 208&BN\_num\_bits&14&CF\\
bn\_lib.c& 209&BN\_num\_bits&*&\\
bn\_lib.c& 212&BN\_num\_bits&*&\\
bn\_gcd.c& 515&BN\_mod\_inverse&*&\\
bn\_div.c& 381&BN\_div&2 &CF\\
bn\_div.c& 439&BN\_div&14&CF\\
bn\_div.c& 385&BN\_div&10&CF\\
bn\_div.c& 381&BN\_div&1 &CF\\
bn\_div.c& 381&BN\_div&1 &CF\\
bn\_div.c& 469&BN\_div&0 &CF\\
bn\_exp.c& 676&BN\_mod\_exp\_mont\_consttime&1 &CF\\
bn\_exp.c& 796&BN\_mod\_exp\_mont\_consttime&0 &CF\\
bn\_mont.c& 262&BN\_from\_montgomery\_word&*&\\
bn\_mont.c& 263&BN\_from\_montgomery\_word&*&\\
bn\_mont.c& 264&BN\_from\_montgomery\_word&*&\\
bn\_mont.c& 266&BN\_from\_montgomery\_word&*&\\
bn\_mont.c& 276&BN\_from\_montgomery\_word&*&\\
bn\_mont.c& 276&BN\_from\_montgomery\_word&0 &DA\\
bn\_mont.c& 276&BN\_from\_montgomery\_word&0 &DA\\
bn\_mont.c& 275&BN\_from\_montgomery\_word&0 &CF\\
bn\_mont.c& 282&BN\_from\_montgomery\_word&*&\\
bn\_asm.c& 787&bn\_sqr\_comba8&*&\\
bn\_asm.c& 646&bn\_mul\_comba8&*&\\
bn\_mont.c& 201&BN\_from\_montgomery\_word&0 &CF\\
\hline
\end{tabular}
%}
\end{table}

%% \begin{table}[h!]
\centering\tiny\scriptsize
\caption{Leakages in RSA implemented by OpenSSL 1.0.2k}\label{tab:RSAOpenSSL1.0.2k}
%\resizebox{\columnwidth}{!}{
\begin{tabular}{lrlrr}
\hline
\textbf{File} & \textbf{Line No.} & \textbf{Function} & \textbf{\# Leaked Bits} & \textbf{Type} \\\hline
bn\_lib.c& 199&BN\_num\_bits&*&\\
bn\_lib.c& 200&BN\_num\_bits&10&CF\\
bn\_lib.c& 201&BN\_num\_bits&13&DA\\
bn\_shift.c& 168&BN\_lshift&4 &CF\\
bn\_lib.c& 673&BN\_ucmp&*&\\
bio\_asn1.c& 484&\_\_udivdi3&6 &CF\\
bn\_div.c& 381&BN\_div&*&\\
bn\_div.c& 456&BN\_div&*&\\
bn\_gcd.c& 279&BN\_mod\_inverse&0 &CF\\
bn\_gcd.c& 302&BN\_mod\_inverse&7 &CF\\
bn\_gcd.c& 324&BN\_mod\_inverse&9 &CF\\
bn\_add.c& 255&BN\_usub&*&\\
bn\_gcd.c& 327&BN\_mod\_inverse&13&CF\\
bn\_gcd.c& 305&BN\_mod\_inverse&14&CF\\
bn\_lib.c& 203&BN\_num\_bits&14&DA\\
bn\_lib.c& 208&BN\_num\_bits&14&CF\\
bn\_lib.c& 209&BN\_num\_bits&*&\\
bn\_lib.c& 212&BN\_num\_bits&*&\\
bn\_gcd.c& 515&BN\_mod\_inverse&*&\\
bn\_div.c& 381&BN\_div&8 &CF\\
bn\_div.c& 439&BN\_div&*&\\
bn\_div.c& 385&BN\_div&4 &CF\\
bn\_div.c& 381&BN\_div&0 &CF\\
bn\_div.c& 381&BN\_div&0 &CF\\
bn\_div.c& 469&BN\_div&1 &CF\\
bn\_exp.c& 716&BN\_mod\_exp\_mont\_consttime&1 &CF\\
bn\_exp.c& 836&BN\_mod\_exp\_mont\_consttime&0 &CF\\
bn\_mont.c& 262&BN\_from\_montgomery\_word&*&\\
bn\_mont.c& 263&BN\_from\_montgomery\_word&*&\\
bn\_mont.c& 264&BN\_from\_montgomery\_word&*&\\
bn\_mont.c& 266&BN\_from\_montgomery\_word&*&\\
bn\_mont.c& 276&BN\_from\_montgomery\_word&*&\\
bn\_mont.c& 282&BN\_from\_montgomery\_word&*&\\
bn\_mont.c& 201&BN\_from\_montgomery\_word&0 &CF\\
bn\_asm.c& 646&bn\_mul\_comba8&*&\\
bn\_asm.c& 787&bn\_sqr\_comba8&*&\\
\hline
\end{tabular}
%}
\renewcommand{\baselinestretch}{1.0}\selectfont
\end{table}

\begin{table}%[h]
\centering\tiny
\caption{RSA implemented by openssl 1.1.0f}\label{tab:RSAopenssl}
%\caption{Summary of all vulnerabilities in RSA implemented by openssl 1.1.0f with the amount of leak information. The mark $*$ means timeout,which indicates more severe leakages (see \S\ref{loc:timeout}).}\label{tab:RSAopenssl}
%\resizebox{\columnwidth}{!}{
\begin{tabular}{clrrr}
\hline
\textbf{File} & \textbf{Line Num} & \textbf{Function} & \textbf{Leakedbits} & \textbf{Type} \\\hline
bn\_lib.c& 143&BN\_num\_bits\_word&*&\\
bn\_lib.c& 144&BN\_num\_bits\_word&*&\\
bn\_lib.c& 145&BN\_num\_bits\_word&15&DA\\
bn\_lib.c& 1029&bn\_correct\_top&*&\\
bn\_lib.c& 639&BN\_ucmp&*&\\
ct\_b64.c& 164&\_\_udivdi3&5 &CF\\
bn\_div.c& 330&BN\_div&*&\\
bn\_gcd.c& 192&int\_bn\_mod\_inverse&0 &CF\\
bn\_gcd.c& 215&int\_bn\_mod\_inverse&7 &CF\\
bn\_gcd.c& 237&int\_bn\_mod\_inverse&7 &CF\\
bn\_gcd.c& 218&int\_bn\_mod\_inverse&13&CF\\
bn\_gcd.c& 240&int\_bn\_mod\_inverse&9 &CF\\
bn\_lib.c& 147&BN\_num\_bits\_word&*&\\
bn\_lib.c& 152&BN\_num\_bits\_word&14&CF\\
bn\_lib.c& 153&BN\_num\_bits\_word&*&\\
bn\_lib.c& 156&BN\_num\_bits\_word&*&\\
bn\_div.c& 384&BN\_div&15&CF\\
bn\_div.c& 330&BN\_div&11&CF\\
bn\_div.c& 334&BN\_div&3 &CF\\
bn\_exp.c& 622&BN\_mod\_exp\_mont\_consttime&1 &CF\\
bn\_exp.c& 741&BN\_mod\_exp\_mont\_consttime&0 &CF\\
bn\_mont.c& 138&BN\_from\_montgomery\_word&*&\\
bn\_mont.c& 139&BN\_from\_montgomery\_word&*&\\
bn\_mont.c& 140&BN\_from\_montgomery\_word&*&\\
bn\_mont.c& 142&BN\_from\_montgomery\_word&*&\\
bn\_mont.c& 152&BN\_from\_montgomery\_word&*&\\
bn\_asm.c& 733&bn\_sqr\_comba8&*&\\
bn\_asm.c& 592&bn\_mul\_comba8&*&\\
bn\_mont.c& 98&BN\_from\_montgomery\_word&0 &CF\\
bn\_div.c& 330&BN\_div&0 &CF\\
bn\_div.c& 330&BN\_div&0 &CF\\
\hline
\end{tabular}
%}
\end{table}

\begin{table}%[h]
\centering\tiny\scriptsize
%\caption{RSA implemented by openssl 1.1.1}\label{tab:RSAopenssl}
\caption{Summary of all vulnerabilities in RSA implemented by openssl 1.1.1 with the amount of leak information. The mark $*$ means timeout, which indicates more severe leakages (see \S\ref{loc:timeout}).}\label{tab:RSAopenssl}
%\resizebox{\columnwidth}{!}{
\begin{tabular}{clrrr}
\hline
\textbf{File} & \textbf{Line No.} & \textbf{Function} & \textbf{Leaked Bits} & \textbf{Type} \\\hline
rsa\_ossl.c& 399&rsa\_ossl\_private\_decrypt&0 &CF\\
bn\_lib.c& 555&BN\_ucmp&*&\\
bn\_gcd.c& 199&int\_bn\_mod\_inverse&0 &CF\\
bn\_gcd.c& 247&int\_bn\_mod\_inverse&14&CF\\
bn\_gcd.c& 225&int\_bn\_mod\_inverse&13&CF\\
ct\_b64.c& 168&\_\_udivdi3&0 &CF\\
bn\_div.c& 374&bn\_div\_fixed\_top&*&\\
bn\_lib.c& 955&bn\_correct\_top&2 &CF\\
ct\_b64.c& 168&\_\_memset\_sse2\_rep&0 &CF\\
ct\_b64.c& 168&\_\_memset\_sse2\_rep&0 &CF\\
ct\_b64.c& 168&\_\_memset\_sse2\_rep&0 &DA\\
ct\_b64.c& 168&\_\_memset\_sse2\_rep&0 &DA\\
bn\_exp.c& 317&BN\_mod\_exp\_mont&0 &CF\\
bn\_asm.c& 592&bn\_mul\_comba8&2 &CF\\
bn\_exp.c& 383&BN\_mod\_exp\_mont&0 &CF\\
bn\_lib.c& 453&BN\_bn2binpad&0 &DA\\
bn\_lib.c& 450&BN\_bn2binpad&0 &CF\\
rsa\_oaep.c& 180&RSA\_padding\_check\_PKCS1\_OAEP\_mgf1&0 &DA\\
rsa\_oaep.c& 180&RSA\_padding\_check\_PKCS1\_OAEP\_mgf1&0 &DA\\
rsa\_oaep.c& 176&RSA\_padding\_check\_PKCS1\_OAEP\_mgf1&0 &CF\\
string3.h& 90&SHA1\_Final&0 &CF\\
rsa\_oaep.c& 200&RSA\_padding\_check\_PKCS1\_OAEP\_mgf1&0 &CF\\
rsa\_oaep.c& 209&RSA\_padding\_check\_PKCS1\_OAEP\_mgf1&0 &CF\\
rsa\_oaep.c& 250&RSA\_padding\_check\_PKCS1\_OAEP\_mgf1&0 &CF\\
rsa\_oaep.c& 253&RSA\_padding\_check\_PKCS1\_OAEP\_mgf1&0 &CF\\
ct\_b64.c& 168&\_\_memset\_sse2\_rep&0 &DA\\
ct\_b64.c& 168&\_\_memset\_sse2\_rep&0 &DA\\
\hline
\end{tabular}
%}
\end{table}

