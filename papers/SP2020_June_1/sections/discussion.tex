\section{Discussions and Limitations}
In the section, we discuss \tool's limitations, usages, and
some future works.

\tool{} works on the native x86 execution traces. The design,
which is very precise in terms of true leakages compared to other
static source code method~\cite{197207,BacelarAlmeida:2013:FVS:2483313.2483334}, 
also suffers from limitations of dynamic approaches as well.
\tool{} is not sound and has coverage problem. Each time we only 
get one single execution trace. Therefore, we may neglect 
some side-channel vulnerabilities on other traces. However,
we argue that it is not a crucial problems for analyzing crypto
libraries. Because crypto libraries are designed to have the same
code coverage for various inputs. Our evaluation also confirms 
the above point. For symmetric encryptions during our evaluation, 
there is no secret-dependent control-flow transfers. RSA implementations
have several secret-dependent control-flow transfers. But after we 
manually check those leakages cites. We find most of them are useful
for bound checking, which do not leak much information and have 
negligible effects on the whole code coverage as well.

One of the motivations of \tool{} is that while recent works
have reported lots of tentative side-channel vulnerabilities,
most of them are unpatched by developers. Our evaluation result 
also confirms it. For RSA, the latest OpenSSL\@ only has one leakage
site that can leak more than 3 bits while there are 22 leakage sites
according to \tool{}. DES implementation of OpenSSL\@ has several 
sensitive leakages. But given the end life status of DES, it is 
still unpatched for the worth of engineering effort. At the early
stage of the project, we hope to find some sensitive leakages but are 
neglected by communities for years. But somehow every sensitive leakages
identified by \tool{} are known to people before. We think the main reason is
that we only test famous crypto algorithms in well-known crypto libraries. We will
apply \tool{} into other libraries and non-crypto libraries in the future. 

%\tool{} works on the native x86 instructions, while
%some existing works~\cite{197207,BacelarAlmeida:2013:FVS:2483313.2483334} 
%find side-channels vulnerabilities from source code level 
%or intermediate languages (e.g., VEX, REIL). Apart from the scalability
%issues for IR implementations, \tool{} is designed to work on the
%native x86 instructions for the following considerations. First, compliers can 
%introduce or mitigate side channels vulnerabilities. 
%For example, the GCC compiler may or may not translate the $!$ operator into conditional branches. 
%If the branch is secret-dependent, the attacker could learn some sensitive information.
%However, source-based methods fail to know how those the machine code looks like, which
%leads to false positives or false negatives.  
%Second, we find many crypto libraries have lots of inlined assembly code. In general, 
%it is hard to convert assembly code into source code
%or IR.


