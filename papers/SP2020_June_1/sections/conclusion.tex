\section{Conclusion}
In this paper, we present \tool{} for identifying and quantifying 
memory-based side-channel leakages. We show that \tool{} is effective
at finding and quantifying the side-channel leakages. With the new
definition of information leakage that imitates an real side-channel attacker, 
the number of leaked bits is useful to justify the understand 
the sensitive level of side-channel vulnerabilities. 
The evaluation results confirm our deisgn goal and show
\tool{} is useful to estimate the amount of leaked information in 
real-world applications. 